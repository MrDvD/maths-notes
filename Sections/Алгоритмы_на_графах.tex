\section{Алгоритмы на графах}

\subsection{Обход графа}

{\bold Обход в глубину} {\ital (Depth-First Search, DFS)} --- метод, при котором граф обходят сначала по детям, потом по сиблингам.
% ---
\begin{theorem}
{\bold Алгоритмы:}
% ---
\begin{list*}
\item поиск эйлерового пути, цикла;
\item тест ацикличности;
\item поиск мостов, точек сочленения;
\item топологическая сортировка;
\item построение компонент связности:
\begin{list*}[2]
\item обыкновенных {\ital\color{desc} (грядки, водостоки)};
\item сильных {\ital\color{desc} (алгоритм Косараджу, конденсация)}.
\end{list*}
\item 2-SAT;
\item алгоритм Куна.
\end{list*}
\end{theorem}
% ---
{\bold Обход в ширину} {\ital (Breadth-First Search, BFS)} --- метод, при котором граф обходят сначала по сиблингам, потом по детям.
% ---
\begin{theorem}
{\bold Алгоритмы:}
% ---
\begin{list*}
\item алгоритм Кана;
\item проверка графа на двудольность;
\item поиск кратчайших рёберных путей.
\end{list*}
\end{theorem}

Нет циклов $\implies$ не требуется массив {\ital visited}.

Также были задачи на потоки. Рассмотреть их!

\subsection{Поиск эйлерового пути}

{\bold Алгоритм} поиска эйлерового {\ital цикла}:

\begin{code}{python}
if is_eulerian(graph): #1
  ecirc = graph.get_ecirc(v) #2
else:
  ecirc = -1

def get_ecirc(v): # DFS
  ecirc = list()
  for n in edges[v]:
    if edges[v][n] > 0:
      edges[v][n] -= 1
      ecirc += get_ecirc(n)
  return ecirc + [v] #3
\end{code}

\begin{list*}[][\#]
\item Проверить граф на {\ital эйлеровость}.
\item Начать с любой вершины $v$.
\item Если дан орграф, ответ нужно {\ital инвертировать}.
\end{list*}

{\bold Алгоритм} поиска эйлерового {\ital пути}:

\begin{code}{python}
if is_semi_eulerian(graph): #1
  epath = graph.get_epath(v) #2
else:
  epath = -1

def get_epath(v): # DFS
  epath = list()
  for n in edges[v]:
    if edges[v][n] > 0:
      edges[v][n] -= 1
      epath += get_epath(n)
  return epath + [v] #3
\end{code}

\begin{list*}[][\#]
\item Проверить граф на {\ital полуэйлеровость}.
\item Начать с вершины нечётной степени $v$.

{\bold НО:} если дан орграф, выбрать вершину большей {\ital исхо"=дящей} валентности.
\item Если дан орграф, ответ нужно {\ital инвертировать}.
\end{list*}

Применяется в {\bold задачах}:

\begin{list*}
\item китайский почтальон;
\item домино, восстановление строки.
\end{list*}

\subsection{Топологическая сортировка}

{\bold Алгоритм} топологической сортировки через {\ital DFS}:

\begin{code}{python}
def topo_sort(v): # DFS
  vertices[v].visited = 1
  stack = list()
  for n in vertices[v].adjacent:
    if not vertices[v].visited:
      stack += topo_sort(n)
  return stack + [v]
\end{code}

\subsection{Алгоритм Кана}

{\bold Алгоритм} топологической сортировки через {\ital BFS}:

\begin{code}{python}
def topo_sort(queue=deque()):
  for v in vertices: #1
    if vertices[v].indegree == 0:
      queue.append(v)
  order = list()
  while queue:
    q = queue.popleft()
    order.append(q)
    for n in vertices[q].adjacent_out:
      vertices[n].indegree -= 1
      if vertices[n].indegree == 0: #2
        queue.append(n)
  return order 
\end{code}

\begin{list*}[][\#]
\item Найти {\ital корни} графа, добавить их в очередь.
\item Если какая либо из соседних вершин стала {\ital корнем}.
\end{list*}

Применяется в {\bold задачах}:

\begin{list*}
\item лексикографическая сортировка;
\item топологическое маркирование;
\item тест ацикличности.
\end{list*}

\subsection{Алгоритм Косараджу}

{\bold Алгоритм} поиска компонент сильной связности:

\begin{code}{python}
reverse = graph.transpose() #1
order = reverse.topo_sort() #2
mark = 0
for v in order[::-1]: #3
  if not graph.vertices[v].visited:
    mark_component(v, mark)
    mark += 1

def mark_component(v, mark): # DFS
  vertices[v].visited = 1
  vertices[v].mark = mark
  for n in vertices[v].adjacent:
    if not vertices[n].visited:
      mark_component(n, mark)
\end{code}

\begin{list*}[][\#]
\item Построить {\ital транспонированный} граф {\ital reverse}.
\item Применить {\ital топологическую сортировку} к {\ital reverse}.
\item Применить {\ital DFS} на {\ital graph} в обратном порядке топологи"=ческой сортировки:
% ---
$$\text{цикл поиска в глубину}\equiv\text{сильная компонента связности}$$
\end{list*}

\subsection{2-SAT}

{\bold Алгоритм} решения:

\begin{list*}[][\#]
\item Перевести CNF в INF.
\item Построить граф импликаций.
\item Найти компоненты сильной связности в графе.
\item Проверить, что для любой вершины $x$ графа справедливо:
$$c[x]\neq c[\lnot x]$$
\item Если требуется вывести ответ, то воспользоваться формулой:
$$x=\left\{\begin{aligned}
&\text{true}, &c[x]\less c[\lnot x]\\
&\text{false}, &c[x]\greater c[\lnot x]
\end{aligned}\right.$$
\end{list*}

\subsection{Поиск мостов}

{\bold Алгоритм} поиска мостов на неорграфе:

\begin{code}{python}
def get_bridges(v, parent=-1): # DFS
  vertices[v].visited = 1
  time += 1
  vertices[v].ord = time #1
  vertices[v].min = time
  bridges = set()
  for n in vertices.adjacent:
    if n == parent: #2
      continue
    if vertices[n].visited: #3
      vertices[v].min = min(vertices[v].min, \
                            vertices[n].ord)
    else: #4
      bridges |= get_bridges(n, v) #5
      vertices[v].min = min(vertices[v].min, \
                            vertices[n].min)
      if vertices[n].min > vertices[v].ord and \
         parent != -1: #6
           bridges.add((v, n))
  return bridges
\end{code}

\begin{list*}[][\#]
\item Запись текущего порядка захода в вершину.
\item Если $\langle v,n\rangle$ --- ребро в {\ital обратную сторону}.
\item Если $\langle v,n\rangle$ --- {\ital обратное ребро}.
\item Если $\langle v,n\rangle$ --- {\ital ребро дерева}.
\item Мост может быть отмечен {\ital несколько раз} за алгоритм.
\item Критерий моста:
% ---
$$[\langle v,neigh\rangle\text{ --- мост}]=\left\{\begin{aligned}
&\text{true}, &\text{min}[neigh]\greater\text{ord}[v]\\
&\text{false}, &\text{min}[neigh]\leq\text{ord}[v]
\end{aligned}\right.$$
\end{list*}

\subsection{Поиск точек сочленения}

{\bold Алгоритм} поиска точек сочленения на неорграфе:

\begin{code}{python}
def get_cuts(v, parent=-1): # DFS
  vertices[v].visited = 1
  time += 1
  vertices[v].ord = time #1
  vertices[v].min = time
  children, cuts = 0, set()
  for n in vertices.adjacent:
    if n == parent: #2
      continue
    if vertices[n].visited: #3
      vertices[v].min = min(vertices[v].min, \
                            vertices[n].ord)
    else: #4
      children += 1 
      cuts |= get_cuts(n, v) #5
      vertices[v].min = min(vertices[v].min, \
                            vertices[n].min)
      if vertices[n].min >= vertices[v].ord and \
         parent != -1: #6
           cuts.add(v)
  if children > 1 and parent == -1:
    cuts.add(v)
  return cuts
\end{code}

\begin{list*}[][\#]
\item Запись текущего порядка захода в вершину.
\item Если $\langle v,n\rangle$ --- ребро в {\ital обратную сторону}.
\item Если $\langle v,n\rangle$ --- {\ital обратное ребро}.
\item Если $\langle v,n\rangle$ --- {\ital ребро дерева}.
\item Вершина может быть отмечена {\ital несколько раз} за алгоритм.
\item Критерий точки сочленения:
% ---
$$[v\text{ --- т. сочленения}]=\left\{\begin{aligned}
&\text{true}, &\text{min}[neigh]\geq\text{ord}[v]\\
&\text{false}, &\text{min}[neigh]\less\text{ord}[v]
\end{aligned}\right.$$
\end{list*}

\subsection{Алгоритм Куна}

{\bold Алгоритм} поиска максимального паросочетания на двудольном графе:

\begin{code}{python}
for v in vertices.part: #1
  if has_ichain(v):
    null_all_visited(vertices)

matching = dict() # empty == None
def has_ichain(v): # DFS
  if vertices[v].visited:
    return False
  vertices[v].visited = 1
  for n in vertices[v].adjacent:
    if matching[n] == None or \
       has_ichain(matching[n]): #2
         matching[n] = v
         return True
  return False
\end{code}

\begin{list*}[][\#]
\item Рассматриваются все вершины {\ital одной доли} двудольного графа.
\item Если пары нет или если есть {\ital увеличивающаяся цепь}.
\end{list*}

\subsection{Кратчайший путь}

{\bold Волновой алгоритм} --- да.

{\bold Алгоритм Дейкстры} ищёт кратчайшие пути из заданной вершины до всех остальных вершин взвешенного графа {\ital\color{desc} (вес рёбер неотрицательный)}:

\begin{code}{python}
def dijkstra(): # tabulation
  vertices[1].cost = 0 #1
  for _ in vertices:
    v = (None, inf)
    for i in vertices:
      if not vertices[i].visited and \
         vertices[i].min < v[1]:
           v = (i, vertices[i].min)
    if v[0] == None: #2
      break
    vertices[v[0]].visited = 1
    for n in vertices[v[0]].adjacent:
      curr_cost = vertices[v[0]].min + \
                  edges[n][v[0]].cost
      if curr_cost < vertices[n].min:
        vertices[n].min = curr_cost
        vertices[n].parent = v[0] #3
\end{code}

\begin{list*}[][\#]
\item База динамики; по умолчанию вес рёбер --- $\infty$.
\item Если добраться до вершины невозможно.
\item Параметр для составления сертификата решения.
\end{list*}

{\bold Алгоритм Беллмана-Форда} --- да.

\subsection{Алгоритм Прима}

{\bold Алгоритм} поиска минимального остовного дерева {\ital\color{desc} (по структуре идентичен алгоритму Дейкстры)}:

\begin{code}{python}
def prim(): # greedy
  vertices[1].min = 0 #1
  for _ in vertices:
    v = (None, inf)
    for i in vertices:
      if not vertices[i].visited and \
         vertices[i].min < v[1]:
           v = (i, vertices[i].min)
    if v[0] == None: #2
      break
    vertices[v[0]].visited = 1
    if vertices[v[0]].parent != None: #3
      add_safe_edge(v[0], vertices[v[0]].parent)
    for n in vertices[v[0]].adjacent:
      curr_cost = edges[v[0]][n]
      if curr_cost < vertices[n].min:
        vertices[n].min = curr_cost
        vertices[n].parent = v[0]
\end{code}

\begin{list*}[][\#]
\item База динамики; по умолчанию вес рёбер --- $\infty$.
\item Если добраться до вершины невозможно.
\item Вывести {\ital безопасное ребро}, если обработано хотя бы две вершины.
\end{list*}
