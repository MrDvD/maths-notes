\section{Теория алгоритмов}

\subsection{Динамическое программирование}

{\bold Динамическое программирование} --- метод решения задач на оптимизацию {\ital по принципу оптимальности}:

{\centering «оптимальная структура имеет оптимальные\\
подструктуры» {\ital\color{desc} (Р. Беллман)}\par}

\subsection{Уравнение Беллмана}

Введём задачу на оптимизацию вида:

\begin{tabularcx}{0pt}{3pt}{>{\centering\arraybackslash}m{39mm}r @{ --- } L}{\textwidth}
\multirow{3}{*}{\vspace*{-6pt}$\displaystyle\optm{d\in\Delta}\{H(d)\}$} & $d$ & выбор;\\
& $\Delta$ & допустимое множество;\\
& $H$ & целевая функция одной переменной.
\end{tabularcx}

{\ital Оптимум} --- оптимальное значение целевой функции {\ital\color{desc} (выбор $d^*$ оптимизирует $H$)}:
% ---
$$H^*:=H(d^*)\quad\quad d^*:=\text{arg }\optm{d\in\Delta}\{H(d)\}$$
% ---
Пусть $H$ --- целевая функция нескольких переменных.

Оптимум такой задачи можно найти либо {\ital полным перебо"=ром}, либо {\ital последовательным принятием решений}:
% ---
$$\begin{aligned}
H^*&=\optm{\langle d_1,\dots,d_n\rangle\in\Delta}\{H(d_1,\dots,d_n)\}\\
&=\optm{d_1\in D_1}\{\optm{d_2\in D_2}\{\dots\{\optm{d_n\in D_n}\{h(d_1,\dots,d_n)\}\}\dots\}\}\\
&=\optm{d_1\in D_1}\{H(d_1,d^*_2(d_1),\dots,d^*_n(d_1))\}
\end{aligned}$$

\begin{tabularcx}{0pt}{3pt}{r @{ --- } L}{\textwidth}
$\Delta=D_1\times\dots\times D_n$ & пространство решений;\\
$D_n(d_1,\dots,d_{n-1})$ & множество решений, которое зависит от~предыдущих $\langle d_1,\dots, d_{n-1}\rangle$ решений;\\
$d^*_i(d_1,\dots,d_{i-1})$ & локальный выбор $d$, оптимизирующий $H$.
\end{tabularcx}

\subsection{Распределение ресурсов}

В задаче на {\ital оптимальное распределение ресурсов} требуется разделить ограниченное число ресурсов на множество их потребителей, у которых есть стоимость.

Общая формула:
% ---
$$f(k,m)=\min_{d\in\{0,\dots,m\}}\{C(k,d)+f(k+1,m-d)\}$$
