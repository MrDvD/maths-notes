\section{Теория множеств}

\subsection{Открытое множество}

{\ital $\varepsilon$-окрестность} точки $x_0\in X$ метрического пространства
$\langle X,d\rangle$ --- такое множество точек $x\in X$, что $d(x_0,x)\less\varepsilon$.

Упрощённая запись $\{x\mid d(x_0,x)\less\varepsilon\}=:U_\varepsilon(x_0)$.

Особые случаи:
% ---
\begin{align*}
U_\varepsilon(+\infty)&:=(1/\varepsilon;+\infty)\\
U_\varepsilon(-\infty)&:=(-\infty;-1/\varepsilon)
\end{align*}
% ---
{\ital Проколотой} называется $\varepsilon$-окрестность точки $x_0$ без неё:\\[-8pt]
% ---
$$\overset{\circ}{U}_\varepsilon(x_0):=U_\varepsilon(x_0)\backslash\{x_0\}$$
% ---
{\ital Правосторонней (левосторонней)} называется $\varepsilon$-окрестность точки $x_0$
без левой (правой) половины:
% ---
$$U_{\varepsilon+}(x_0):=[x_0;\varepsilon)\quad\quad U_{\varepsilon-}(x_0):=(\varepsilon;x_0]$$

\subsection{Ограниченное множество}

Множество $M$ ограничено {\ital сверху}, если
% ---
$$\forall m\in M\ \exists C\in\mathbb{R}\colon m\leq C.$$
% ---
{\bold Точной} {\ital (минимальной}, англ. {\ital supremum)} называется такая
{\ital верхняя} граница множества $M$ --- $\sup M$, что
% ---
$$\forall\varepsilon\greater 0\ \exists m\in M\colon m\in U_{\varepsilon-}(\sup M).$$
% ---
Множество $M$ ограничено {\ital снизу}, если
% ---
$$\forall m\in M\ \exists C\in\mathbb{R}\colon m\geq C.$$
% ---
{\bold Точной} {\ital (максимальной}, англ. {\ital infimum)} называется такая
{\ital нижняя} граница множества $M$ --- $\inf M$, что
% ---
$$\forall\varepsilon\greater 0\ \exists m\in M\colon m\in U_{\varepsilon+}(\inf M).$$

\subsection{Принцип Кантора}

Последовательность вложенных отрезков содержит точки $\xi$, которые принадлежат им всем:
% ---
$$\forall n\in\mathbb{N}\ \exists\xi\in[a_n;b_n]\subset[a_{n-1};b_{n-1}]$$
% ---
Если $n\to\infty$, $(b_n-a_n)\to 0$, то $\xi$ единственна:
% ---
$$\lim_{n\to\infty}a_n=\sup\{a_n\}=\lim_{n\to\infty}b_n=\inf\{b_n\}=\xi$$
% ---
{\bold Доказательство.} По теореме Вейерштрасса:
% ---
$$\lim_{n\to\infty}a_n=\sup\{a_n\}\quad\quad\lim_{n\to\infty}b_n=\inf\{b_n\}$$
% ---
Значит, $\forall(n\in\mathbb{N},\ \xi\in[\sup\{a_n\};\inf\{b_n\}])\ \xi\in[a_n;b_n]$.
$\qedw$

Если $\inf\{b_n\}=\sup\{a_n\}$, то $\xi$ единственна:
% ---
$$0=\inf\{b_n\}-\sup\{a_n\}=\lim_{n\to\infty}b_n-\lim_{n\to\infty}a_n=\lim_{n\to\infty}
(b_n-a_n)\qedb$$

\subsection{Локальный экстремум}

{\ital Локальный} {\bold максимум} функции $f$ --- такая точка $x_0$, что
% ---
$$\exists\delta\greater 0\colon\sup U_\delta(x_0)=f(x_0).$$ 
% ---
{\ital Локальный} {\bold минимум} функции $f$ --- такая точка $x_0$, что
% ---
$$\exists\delta\greater 0\colon\inf U_\delta(x_0)=f(x_0).$$
% ---
 Их объединяют в точки {\ital локального} {\bold экстремума}.
 
{\ital Критической} называется такая точка $x_0$, что
% ---
$$\begin{sqcases}
f'(x_0)=0\text{ \ital\color{desc}(стационарна)}\\
f'(x_0)=\text{undefined}
\end{sqcases}$$
