\section{Интегральное исчисление}

\subsection{Неопределённый интеграл}

{\bold Первообразная} для функции $f$ на множестве $X$ --- такая функция $F$, что:
% ---
$$\forall x\in X\ F'(x)=f(x)$$
% ---
\begin{theorem}
Если у функции $f$ есть первообразная $F$, то для любой константы $C$ функция $F+C$ тоже первообразная, причём других нет.
\end{theorem}
% ---
{\bold Доказательство.} По определению первообразной:
% ---
$$F'=f$$
% ---
По дистрибуции производной:
% ---
$$(F+C)'=f\implies F+C\text{ --- первообразная для }f\qedw$$
% ---
Пусть $\Phi$ --- другая первообразная для $f$:
% ---
$$\begin{cases*}
\Phi'=f\\
F'=f
\end{cases*}\implies
\Phi'-F'=(\Phi-F)'=f-f=0$$
% ---
По условию постоянства функции:
% ---
$$\Phi-F=C\iff\Phi=F+C\qedb$$
% ---
{\bold Неопределённый интеграл} --- множество всех первообраз"=ных функции $f$:
% ---
$$F(x)+C=:\int f(x)\diff x$$
% ---
\begin{tabularx}{\textwidth}{r @{ --- } L}
$f$ & подынтегральная функция;\\
$f(x)\diff x$ & подынтегральное выражение;\\
$x$ & переменная интегрирования;\\
$C$ & постоянная интегрирования.
\end{tabularx}

\subsection{Свойства неопределённого интеграла}

Операция интегрирования {\ital аддитивна}, а также:
% ---
$$\begin{aligned}
\int F'(x)\diff x&=F(x)+C & &\diff\left(\int F(x)\diff x\right)=F(x)\diff x\\
\left(\int F(x)\diff x\right)'&=F(x)+C & &\int kF(x)\diff x=k\int F(x)\diff x,\ k\neq 0\\
\end{aligned}$$
% ---
\begin{theorem}
{\bold Интегрирование по частям.} Пусть $u\diff v$ --- подынтегральная функция. Тогда справедливо:
% ---
$$\int u\diff v=uv-\int v\diff u$$
\end{theorem}

{\bold Доказательство.} По «дистрибуции» производной:
% ---
$$(uv)'=u'v+uv'$$
% ---
По определению интеграла:
% ---
$$\int(u'v+uv')\diff x=uv+C\iff\int u'v\diff x+\int uv'\diff x=uv+C$$
% ---
По определению дифференциала:
% ---
$$\begin{cases*}
\diff u=u'\diff x\\
\diff v=v'\diff x
\end{cases*}\implies
\int v\diff u+\int u\diff v=uv+C\iff$$
$$\iff\int u\diff v=uv-\int v\diff u\qedb$$
% ---
\begin{theorem}
{\bold Инвариантность.} Смена переменной интегрирования на другую дифференцируемую функцию является {\ital равносильным} переходом:
% ---
$$\int f(x)\diff x=F(x)+C\iff\int f(\varphi(x))\diff\varphi(x)=F(\varphi(x))+C$$
\end{theorem}

\subsection{Дифференциальное уравнение}

{\bold Дифференциальным} называется уравнение с неизвестной функцией под знаком {\ital производной} или {\ital дифференциала}.

\begin{theorem}
{\bold Метод Фурье.} Решение дифференциального уравнения $y'=\varphi(x)\psi(y)$ удовлетворяет условию: {\ital\color{desc}(Ж. Фурье)}
% ---
$$\left[\begin{aligned}
&\int\frac{\diff y}{\psi(y)}=\int\varphi(x)\diff x, & \psi(y)&\neq 0\\
&\ y=y_0, & \psi(y_0)&=0
\end{aligned}\right.$$
\end{theorem}

{\bold Доказательство.} По условию:
% ---
$$y'=\varphi(x)\psi(y)\iff\frac{y'}{\psi(y)}=\varphi(x),\ \psi\neq 0$$
% ---
Возьмём интеграл от обеих частей уравнения:
% ---
$$\frac{y'\diff x}{\psi(y)}=\varphi(x)\diff x\implies\int\frac{\diff y}{\psi(y)}=\int\varphi(x)\diff x\qedw$$
% ---
По условию:
% ---
$$\psi(y_0)=0\implies (y_0)'=0\implies 0=0\qedb$$

\subsection{Площадь плоской фигуры}

{\bold Вложеннной} {\ital в фигуру} $F$ называется такая фигура $P$, которая целиком лежит внути $F$:
% ---
$$S_*(F)\text{ --- {\ital внутренняя площадь }}F$$
% ---
{\bold Объемлющей} {\ital фигуру} $F$ называется такая фигура $Q$, которая целиком содержит $F$:
% ---
$$S^*(F)\text{ --- {\ital внешняя площадь }}F$$
% ---
{\bold Квадрируемой} называется такая фигура $F$, у которой множества $S_*(F)$ и $S^*(F)$ имеют единую точную границу:
% ---
$$\sup S_*(F)=\inf S^*(F)=S(F)\text{ --- {\ital площадь }}F$$
% ---
{\bold Спрямляемой} называется кривая с {\ital конечной} длиной.

\subsection{Свойства квадрируемости}

\begin{theorem}
{\bold Критерий квадрируемости.} Фигура $F$ квадрируема тогда и только тогда, когда:
% ---
$$\forall\varepsilon\greater 0\ \exists\ P\subseteq F\subseteq Q\colon S(Q)-S(P)\less\varepsilon$$
\end{theorem}
% ---
{\bold Доказательство $\implies$}. Зафиксируем $\varepsilon\greater 0$.

По определению точных границ множеств $S(P),\ S(Q)$:
% ---
\begin{gather*}\left\{\begin{aligned}
&\forall\varepsilon/2\greater 0\ \exists\ P\subseteq F\colon S(F)-S(P)\less\varepsilon/2\\
&\forall\varepsilon/2\greater 0\ \exists\ Q\supseteq F\colon S(Q)-S(F)\less\varepsilon/2
\end{aligned}\right.\implies\\
\implies S(Q)-S(P)\less\varepsilon\qedb
\end{gather*}
% ---
{\bold Доказательство $\impliedby$}. По определению точных верхних границ множеств $S(P),\ S(Q)$:
% ---
$$S(Q)-S(P)\less\varepsilon\implies 0\leq\underset{Q\supseteq F}{\inf}S(Q)-\underset{P\subseteq F}{\sup}S(P)\less\epsilon$$
% ---
По определению квадрируемой фигуры:
% ---
\begin{gather*}
\inf S^*(F)-\sup(S_*(F))=0\iff\inf S^*(F)=\sup S_*(F)\implies\\
\implies F\text{ квадрируема}\qedb\end{gather*}
% ---
\begin{theorem}
{\bold Признак квадрируемости.} Если граница фигуры $F$ --- спрямляемая кривая, то $F$ квадрируема.
\end{theorem}
% ---
{\bold Доказательство.} По условию:
% ---
$$S^*(F)-S_*(F)\text{ --- S фигуры, объемлющей границу }F$$
% ---
По определению точных границ множеств $S^*(F),\ S_*(F)$:
% ---
$$\inf S^*(F)-\sup S_*(F)=S_\text{\tinyt гр}=0\implies F\text{ квадрируема}\qedb$$
% ---
\begin{theorem}
{\bold Аддитивность.} Пусть $F_1$ и $F_2$ квадрируемы, причём $F_1\cup F_2=F,\ F_1\cap F_2=\emptyset$. Тогда $F$ {\ital тоже} квадрируема.
\end{theorem}
% ---
{\bold Доказательство.} По условию:
% ---
$$F=F_1\cup F_2\implies S_\text{\tinyt гр}\leq S_{\text{\tinyt гр}1}+S_{\text{\tinyt гр}2}$$
% ---
По критерию квадрируемости:
% ---
$$S_{\text{\tinyt гр}1}+S_{\text{\tinyt гр}2}=0\implies S_\text{\tinyt гр}=0\implies F\text{ квадрируема}\qedb$$
% ---
\begin{theorem}
Пересечение квадрируемых фигур {\ital квадрируемо}.
\end{theorem}
% ---
Доказательство {\ital аналогично} предыдущему свойству.

\subsection{Определённый интеграл}

{\bold Разбиение} отрезка $[a;b]$ --- конечное упорядоченное мно"=жество $X\subseteq[a;b]$, причём $a,b\in X$.

{\bold Частичным} называется отрезок, составленный из {\ital соседних} элементов разбиения:
% ---
$$\Delta x_k=x_k-x_{k-1}\text{ --- {\ital длина} частичного отрезка }[x_{k-1};x_k]$$
% ---
{\bold Интегральная сумма} функции $f$ на $[a;b]$ имеет вид:
% ---
$$\sum_{k=1}^{n}f(\xi_k)\Delta x_k,\ \xi_k\in [x_{k-1};x_k]$$

{\bold Нижней} {\bold(верхней)} называется такая интегральная сумма, в~которой $\xi_k$ {\ital минимизирует (максимизирует)} значение $f$ на~частичном отрезке:
% ---
$$\sum_{k=1}^n\underline{f}\Delta x_k\text{ --- {\ital нижняя} сумма}\quad\sum_{k=1}^n\overline{f}\Delta x_k\text{ --- {\ital верхняя} сумма}$$
% ---
{\bold Определённый интеграл} --- предел интегральной суммы:
% ---
$$\lim_{n\to+\infty}\sum_{k=1}^nf(\xi_k)\Delta x_k=:\int_a^bf(x)\diff x$$
% ---
{\bold Криволинейная трапеция} --- подграфик неотрицательной и непрерывной функции на $[a;b]$.
% ---
\begin{theorem}
{\bold Геометрический смысл.} Пусть $f$ задаёт криволинейную трапецию $T$ на $[a;b]$. Тогда её площадь равна:
% ---
$$S(T)=\int_a^bf(x)\diff x$$
\end{theorem}
% ---
{\bold Доказательство.} По признаку квадрируемости:
% ---
$$f\in\mathbb{C}[a;b]\implies T\text{ квадрируема}$$
% ---
По определению интегральных сумм:
% ---
$$S^*(T)\text{ --- {\ital верхняя} сумма;}\quad S_*(T)\text{ --- {\ital нижняя} сумма}$$
% ---
По определению квадрируемости:
% ---
\begin{gather*}
S^*(T)-S_*(T)\less\varepsilon\implies\inf S^*(T)=\sup S_*(T)=S(T)\implies\\
\implies\lim_{n\to+\infty}\sum_{k=1}^nf(\xi_k)\Delta x_k=S(T)
\end{gather*}
% ---
По определению определённого интеграла:
% ---
$$S(T)=\int_a^bf(x)\diff x\qedb$$
% ---
\begin{theorem}
{\bold Оценка определённого интеграла.} Пусть функция $f$ на отрезке $[a;b]$ принимает значения из $[m;M]$. Тогда:
% ---
$$m(b-a)\leq\int_a^bf(x)\diff x\leq M(b-a)$$
\end{theorem}
% ---
Очевидным является {\ital геометрическое} доказательство через площади фигур.

\subsection{Интеграл с переменным пределом}

\begin{theorem}
{\bold Теорема о среднем.} Пусть $f\in\mathbb{C}[a;b]$. Тогда верно:
% ---
$$\exists\xi\in[a;b]\colon\int_a^bf(x)\diff x=f(\xi)(b-a)$$
\end{theorem}
% ---
{\bold Доказательство.} По оценке определённого интеграла:
% ---
\begin{gather*}
m\leq f(x)\leq M\implies m(b-a)\leq\int_a^bf(x)\diff x\leq M(b-a)\implies\\
\implies m\leq\frac{\int_a^bf(x)\diff x}{b-a}\leq M,\ a\neq b
\end{gather*}
% ---
По теореме о промежуточном значении:
% ---
$$\exists\xi\in[a;b]\colon f(\xi)=\frac{\int_a^bf(x)\diff x}{b-a}\implies\int_a^bf(x)\diff x=f(\xi)(b-a)\qedb$$
% ---
{\bold Интеграл с переменным верхним пределом} --- функция вида:
% ---
$$S(x)=\int_a^xf(t)\diff t,\ x\in[a;b]$$
% ---
\begin{theorem}
Пусть функция $f$ непрерывна в окрестности точки $t=x$. Тогда в $x$ функция $S(x)$ дифференцируема, причём
% ---
$$S'(x)=f(x).$$
\end{theorem}
% ---
{\bold Доказательство.} По геометрическому смыслу определён"=ного интеграла:
% ---
\begin{gather*}
\Delta S=S(x+\Delta x)-S(x)=\int_a^{x+\Delta x}f(t)\diff t-\int_a^xf(t)\diff t=\\
=\int_a^xf(t)\diff t+\int_x^{x+\Delta x}f(t)\diff t-\int_a^xf(t)\diff t=\int_x^{x+\Delta x}f(t)\diff t
\end{gather*}
% ---
По теореме о среднем:
% ---
$$\exists\xi\in[x;x+\Delta x]\colon\int_x^{x+\Delta x}f(t)\diff t=f(\xi)(x+\Delta x-x)=f(\xi)\Delta x$$
% ---
По предельному переходу:
% ---
\begin{gather*}
\Delta S=f(\xi)\Delta x\iff\frac{\Delta S}{\Delta x}=f(\xi)\implies\lim_{\Delta x\to 0}\frac{\Delta S}{\Delta x}=\lim_{\Delta x\to 0}f(\xi)\implies\\
\implies S'(x)=f(x)\qedb
\end{gather*}

\subsection{Формула Ньютона—Лейбница}

\begin{theorem}
Пусть $F$ --- первообразная для функции $f$. Тогда верно:
% ---
$$\int_a^bf(x)\diff x=\left.F(x)\right\rvert_a^b=F(b)-F(a)$$
\end{theorem}
% ---
{\bold Доказательство.} По свойству интеграла с переменным верхним пределом:
% ---
$$S'(x)=f(x)\implies S(x)=F(x)+C\text{ --- первообразные}$$
% ---
По определению определённого интеграла:
% ---
$$S(a)=\int_a^af(t)\diff t=0\implies C=-F(a)\implies S(x)=F(x)-F(a)$$
% ---
По условию:
% ---
$$x=b\implies S(b)=F(b)-F(a)\iff\int_a^bf(t)\diff t=F(b)-F(a)\qedb$$

\subsection{Свойства определённого интеграла}

Скоро...
