\section{Интегральное исчисление}

\subsection{Неопределённый интеграл}

{\bold Первообразная} для функции $f$ на множестве $X$ --- такая функция $F$, что:
% ---
$$\forall x\in X\ F'(x)=f(x)$$
% ---
\begin{theorem}
Если у функции $f$ есть первообразная $F$, то для любой константы $C$ функция $F+C$ тоже первообразная, причём других нет.
\end{theorem}
% ---
{\bold Доказательство.} По определению первообразной:
% ---
$$F'=f$$
% ---
По дистрибуции производной:
% ---
$$(F+C)'=f\implies F+C\text{ --- первообразная для }f\qedw$$
% ---
Пусть $\Phi$ --- другая первообразная для $f$:
% ---
$$\begin{cases*}
\Phi'=f\\
F'=f
\end{cases*}\implies
\Phi'-F'=(\Phi-F)'=f-f=0$$
% ---
По условию постоянства функции:
% ---
$$\Phi-F=C\iff\Phi=F+C\qedb$$
% ---
{\bold Неопределённый интеграл} --- множество всех первообраз"=ных функции $f$:
% ---
$$F(x)+C=:\int f(x)\diff x$$
% ---
\begin{tabularx}{\textwidth}{r @{ --- } L}
$f$ & подынтегральная функция;\\
$f(x)\diff x$ & подынтегральное выражение;\\
$x$ & переменная интегрирования;\\
$C$ & постоянная интегрирования.
\end{tabularx}

\subsection{Свойства}

Операция интегрирования {\ital дистрибутивна} относительно {\ital сложения}, а также:
% ---
$$\begin{aligned}
\int F'(x)\diff x&=F(x)+C & &\diff\left(\int F(x)\diff x\right)=F(x)\diff x\\
\left(\int F(x)\diff x\right)'&=F(x)+C & &\int kF(x)\diff x=k\int F(x)\diff x,\ k\neq 0\\
\end{aligned}$$
% ---
\begin{theorem}
{\bold Интегрирование по частям.} Пусть $u\diff v$ --- подынтегральная функция. Тогда справедливо:
% ---
$$\int u\diff v=uv-\int v\diff u$$
\end{theorem}

{\bold Доказательство.} По «дистрибуции» производной:
% ---
$$(uv)'=u'v+uv'$$
% ---
По определению интеграла:
% ---
$$\int(u'v+uv')\diff x=uv+C\iff\int u'v\diff x+\int uv'\diff x=uv+C$$
% ---
По определению дифференциала:
% ---
$$\begin{cases*}
\diff u=u'\diff x\\
\diff v=v'\diff x
\end{cases*}\implies
\int v\diff u+\int u\diff v=uv+C\iff$$
$$\iff\int u\diff v=uv-\int v\diff u\qedb$$
% ---
\begin{theorem}
{\bold Инвариантность.} Смена переменной интегрирования на другую дифференцируемую функцию является {\ital равносильным} переходом:
% ---
$$\int f(x)\diff x=F(x)+C\iff\int f(\varphi(x))\diff\varphi(x)=F(\varphi(x))+C$$
\end{theorem}
