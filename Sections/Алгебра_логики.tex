\section{Алгебра логики}

\subsection{Определение}

{\bold Алгебра логики} --- алгебраическая структура, которая образована двухэлементным множеством $\{0;1\}$.

{\bold Высказывание} --- повествовательное предложение, о кото"=ром можно сказать в данный момент, что оно {\ital истинно} или {\ital ложно}.

{\bold Логическая связка} --- операция алгебры логики:
% ---
\begin{list*}[][\#]
\item{\ital Инверсия} «$\lnot$» --- логическое {\ital «не»}.
\item{\ital Конъюнкция} «$\land$» --- логическое {\ital «и»}.
\item{\ital Дизъюнкция} «$\lor$» --- логическое {\ital «или»}.
\item{\ital Строгая дизъюнкция} «$\dot{\lor}$» --- логическое {\ital «искл. или»}.
\item{\ital Импликация} «$\rightarrow$» --- логическое {\ital «$\implies$»}.
\item{\ital Эквиваленция} «$\equiv$» --- логическое {\ital «$\iff$»}.
\end{list*}

\subsection{Свойства}

Конъюнкция и дизъюнкция {\ital коммутативны}, {\ital ассоциативны} и {\ital дистрибутивны} относительно друг друга.
% ---
\begin{theorem}
{\bold Идемпотентность.}
% ---
$$A\land A=A\qquad A\lor A=A$$
\end{theorem}
% ---
\begin{theorem}
{\bold Закон противоречия} и {\bold исключённого третьего.}
% ---
$$A\land\overline{A}=0\qquad A\lor\overline{A}=1$$
\end{theorem}
% ---
\begin{theorem}
{\bold Закон поглощения.}
% ---
$$\begin{aligned}
&A\land 1=A &\quad &A\land 0=0\\
&A\lor 1=1 &\quad &A\lor 0=A
\end{aligned}$$
\end{theorem}
% ---
\begin{theorem}
{\bold Закон де Мóргана.}
% ---
$$\overline{A\land B}=\overline{A}\lor\overline{B}\qquad\overline{A\lor B}=\overline{A}\land\overline{B}$$
\end{theorem}

\subsection{Нормальная форма}

{\bold Терм} --- компонент логической функции:

\begin{list*}
\item{\bold макстерм} --- переменные прямой и инверсной форм связаны {\ital дизъюнкцией};
\item{\bold минтерм} --- переменные прямой и инверсной форм связаны {\ital конъюнкцией};
\end{list*}

{\bold Ранг} {\ital терма} --- число переменных, которые в него входят.

{\bold Нормальная дизъюнктивная форма} {\ital (DNF)} --- дизъюнк"=ция минтермов любого ранга.

{\bold Нормальная конъюнктивная форма} {\ital (CNF)} --- конъюнк"=ция макстермов любого ранга:
% ---
\begin{theorem}
\begin{tabularcx}{0pt}{0pt}{C@{\hspace*{-16pt}}C}{\textwidth}
$\begin{aligned}
A\dot{\lor}B&=(A\lor B)\land(\overline{A}\lor\overline{B})\\
A\equiv B&=(\overline{A}\lor B)\land(A\lor\overline{B})
\end{aligned}$\hspace*{-18pt} & $A\rightarrow B=\overline{A}\lor B$
\end{tabularcx}
\end{theorem}
% ---
{\bold Нормальная импликативная форма} {\ital (INF)} --- конъюнк"=ция макстермов любого ранга, которые заменены имплика"=цией:
% ---
$$A\lor B=(\overline{A}\rightarrow B)\land(\overline{B}\to A)$$
