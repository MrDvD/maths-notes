\section{Аналитическая геометрия}

\subsection{Коллинеарность}

{\bold Коллинеарными} называются:
% ---
\begin{tabularcx}{3pt}{3pt}{@{--- } L}{\textwidth}
{\ital точки}, которые лежат на одной прямой;\\
{\ital векторы}, которые лежат на одной прямой или на~параллельных прямых.
\end{tabularcx}
% ---
\begin{theorem}
{\bold Критерий коллинеарности} двух векторов:
% ---
$$\overrightarrow{a}=\lambda\overrightarrow{b},\ \lambda\in\mathbb{R}\iff\begin{cases}
x_a=\lambda x_b\\
y_a=\lambda y_b
\end{cases}$$ 
\end{theorem}
% ---
В частности, нулевой вектор коллинеарен {\ital любому} вектору:
% ---
$$\overrightarrow{0}=0\overrightarrow{a}$$
% ---
\begin{theorem}
{\bold Уравнение секущей} по двум известным точкам:
% ---
$$A\langle x_a,y_a\rangle,\ B\langle x_b,y_b\rangle\implies\frac{x-x_a}{x_b-x_a}=\frac{y-y_a}{y_b-y_a}$$
\end{theorem}
% ---
{\bold Доказательство.} Пусть $\overrightarrow{AX},\ \overrightarrow{AB}$ --- коллинеарные векторы.

По критерию коллинеарности двух векторов:
% ---
$$\left\{\begin{aligned}
x-x_a=\lambda(x_b-x_a)\\
y-y_a=\lambda(y_b-y_a)
\end{aligned}\right.\iff
\lambda=\frac{x-x_a}{x_b-x_a}=\frac{y-y_a}{y_b-y_a}\qedb$$
