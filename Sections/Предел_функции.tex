\section{Предел функции}

\subsection{Предел}

{\bold Предел} функции $f\colon X\to Y$ в точке $x_0\in X$ {\ital по Коши} --- такое
$a\in Y$, что {\ital\color{desc} (О.Л. Коши)}
% ---
$$\forall\varepsilon\greater 0\ \exists\delta\greater 0\colon\forall x\in\underbrace{
\overset{\circ}{U}_\delta(x_0)\subseteq D_f}_{\text{\tinyt I}},\ \underbrace{f(x)\in U_\varepsilon(a)}_{\text{\tinyt II}}.$$\\
% ---
\begin{tabularc}{0pt}{0pt}{>{\raggedleft\arraybackslash}p{.03\linewidth} @{ --- } 
>{\raggedright\arraybackslash}p{.91\linewidth}}{n}
I & функция $f$ определена в какой-либо проколотой $\delta$-окрестности точки $x_0$;
\\[18pt]
II & функция $f$ имеет образ в какой-либо проколотой $\varepsilon$-окрестности точки 
$a$.  
\end{tabularc}

{\bold Предел} функции $f\colon X\to Y$ в точке $x_0\in X$ {\ital по Гейне} --- такое
$a\in Y$, что {\ital\color{desc} (Э. Гейне)}
% ---
$$\forall\{x_n\}\in D_f\colon\lim_{n\to\infty}x_n=x_0\ (x_n\neq x_0)\implies\lim_{n\to
\infty}f(x_n)=a.$$
% ---
Упрощённая запись $\forall x\in X\ \lim_{x\to x_0}f(x)=a$ или $x\to x_0$, $f(x)\to a$.

\subsection{Критерий Коши}

Сходимость $\iff$ выполнение {\ital условия Коши}:\\[-8pt]
% ---
$$\forall\varepsilon\greater 0\ \exists\delta\greater 0\colon\forall x',x''\in\overset
{\circ}{U}_\delta(x_0)\ \abs{f(x')-f(x'')}\less\varepsilon$$
% ---
{\bold Доказательство $\implies$.} По определению предела:\\[-8pt]
% ---
$$\forall\varepsilon\greater 0\ \exists\delta\greater 0\colon\overset{\circ}{U}_\delta
(x_0)\subseteq D_f,\ U_{\varepsilon/2}(a)\cap E_f\neq\emptyset$$\\[-6pt]
% ---
Пусть $x',x''\in\overset{\circ}{U}_\delta(x_0)$; по неравенству треугольника:
% ---
$$\abs{f(x')-f(x'')}\leq\abs{f(x')-a}+\abs{f(x'')-a}\less\varepsilon/2+\varepsilon/2=
\varepsilon\qedb$$
% ---
{\bold Доказательство $\impliedby$.} По условию Коши:
% ---
$$\exists\{x_n\}\in D_f\colon\lim_{n\to\infty}x_n=x_0,\ x_n\neq x_0$$
% ---
Последовательности $\{f(x_n)\}$ фундаментальны $\implies$ сходятся.

По фундаментальности и сходимости к одной точке $x_0$:
% ---
$$\lim_{x\to x_0}f(x)=a\qedb$$

\subsection{Предел композиции функций}

Пусть $f\colon X\to Y$, $g\colon Y\to Z$. Тогда:
% ---
$$\begin{cases}
\lim_{x\to x_0}f(x)=y_0\\
\lim_{x\to x_0}g(x)=z_0
\end{cases}\hspace*{-12pt}\iff\begin{cases}
\lim_{x\to x_0}(g\circ f)(x)=z_0\\
f(x)\neq y_0
\end{cases}$$
% ---
{\bold Доказательство.} Пусть $g\circ f=\varphi$; по определению предела:\\[-9pt]
% ---
$$\begin{cases}
\forall\varepsilon\greater 0\ \exists\delta\greater 0\colon\overset{\circ}{U}_\delta(y_0)
\subseteq D_g,\ U_\varepsilon(z_0)\cap E_g\neq\emptyset\\
\forall\delta\greater 0\ \exists\sigma\greater 0\colon\overset{\circ}{U}_\sigma(x_0)
\subseteq D_f,\ U_\delta(y_0)\cap E_f\neq\emptyset
\end{cases}$$\\[-6pt]
% ---
Из $\overset{\circ}{U}_\delta(y_0)\cap U_\delta(y_0)=\overset{\circ}{U}_\delta(y_0)$ 
следует:\\[-14pt]
% ---
$$\begin{cases}
\forall\varepsilon\greater 0\ \exists\sigma\greater 0\colon\overset{\circ}{U}_\sigma(x_0)
\subseteq D_f,\ U_\varepsilon(\varphi(x))\cap E_g\neq\emptyset\\
y\neq y_0\iff f(x)\neq y_0
\end{cases}\hspace*{-12pt}\iff$$
$$\lim_{x\to x_0}\varphi(x)=z_0,\ f(x)\neq y_0.\qedb$$

\subsection{Бесконечно малая функция}

Функция $g$ {\ital бесконечно мала} относительно $f$ при $x\to\ x_0$, если
% ---
$$g(x)=\varepsilon(x)f(x):=\underset{x\to\ x_0}{o(f)},\quad\quad\lim_{x\to x_0}\varepsilon
(x)=0.$$
% ---
Верно следующее утверждение:
% ---
$$\lim_{x\to\ x_0}f(x)=a\iff f(x)=a+\underset{x\to x_0}{o(x)}$$
% ---
{\bold Доказательство $\implies$.} По определению предела:
% ---
$$\forall\varepsilon\greater 0\ \exists\delta\greater 0\colon 0\less\abs{x-x_0}\less
\delta,\ \abs{f(x)-a}\less\varepsilon$$
% ---
По теореме о промежуточной функции:
% ---
$$0\leq\abs{f(x)-a}\less\varepsilon\implies\lim_{x\to x_0}(f(x)-a)=0\iff$$
$$f(x)-a=\underset{x\to x_0}{o(x)}\iff f(x)=a+\underset{x\to x_0}{o(x)}\qedb$$
% ---
{\bold Доказательство $\impliedby$.} По условию:
% ---
$$f(x)=a+o(x)\iff\abs{f(x)-a}=\abs{o(x)}$$
% ---
По определению бесконечно малой функции:
% ---
$$\forall\varepsilon\greater 0\ \exists\delta\greater 0\colon 0\less\abs{x-x_0}\less
\delta,\ \abs{o(x)}\less\varepsilon$$
% ---
По определению предела:
% ---
$$\abs{f(x)-a}\less\varepsilon\iff\lim_{x\to x_0}f(x)=a\qedb$$

\subsection{Односторонний предел}

{\bold Правосторонним} называется предел функции, который определён в терминах
правосторонних $\varepsilon$-окрестностей {\ital (неубывающих 
последовательностей)}:
% ---
$$\lim_{x\to x_0+0}f(x)=a\quad\text{или}\quad x\to x_0+0,\ f(x)\to a$$
% ---
{\bold Левосторонним} называется предел функции, который определён в терминах
левосторонних $\varepsilon$-окрестностей {\ital (невозрастающих 
последовательностей)}.
% ---
$$\lim_{x\to x_0-0}f(x)=a\quad\text{или}\quad x\to x_0-0,\ f(x)\to a$$
% ---
Сущестование предела равносильно существованию {\ital равных} односторонних пределов:
% ---
$$\lim_{x\to x_0}f(x)\iff\lim_{x\to x_0+0}f(x)=\lim_{x\to x_0-0}f(x)$$

\subsection{Непрерывность}

Пусть $\forall\varepsilon\greater 0\ U_\varepsilon(x_0)\subseteq D_f$. Тогда:

\begin{tabularc}{0pt}{0pt}{r @{ --- } l}{n}
$x-x_0=:\Delta x$ & {\ital приращение аргумента} в точке $x_0$;\\
$f(x)-f(x_0)=:\Delta f$ & {\ital приращение функции} в точке $x_0$.
\end{tabularc}

Функция $f$ {\ital непрерывна} в точке $x_0$, если
% ---
$$\lim_{x\to x_0}f(x)=f(x_0)\quad\text{или}\quad\Delta x\to 0,\ \Delta f\to 0.$$
% ---
Определения {\ital непрерывности справа} и {\ital слева} точки $x_0$ связаны
с правосторонними и левосторонними пределами.

Непрерывными в точке $x_0$ являются {\ital сумма}, {\ital произведение} и~{\ital 
композиция} непрерывных в ней функций.

\newpage
\subsection{Теорема Вейерштрасса}

Пусть $f\in C[a;b]$. Тогда в некоторых точках отрезка функция достигает своих точных 
верхней и нижней границ на $[a;b]$.

{\bold Доказательство.} Пусть $\sup f([a;b])=:M$, $\inf f([a;b])=:m$.

По определению точных верхней и нижней границ:
% ---
$$\forall x\in[a;b]\ f(x)\in[m;M]$$
% ---
По принципу компактности отрезка:
% ---
$$\lim_{n\to\infty}f(x_n)=M\quad\quad\lim_{k\to\infty}x_{n_k}=\xi$$
% ---
По определению непрерывности:
% ---
$$\lim_{k\to\infty}f(x_{n_k})=f(\xi)\implies f(\xi)=M\qedb$$

\subsection{Теорема о промежуточном значении}

Пусть $f$ непрерывна на промежутке $X\ni a,b$. Тогда:
% ---
$$\forall c\in[f(a);f(b)]\ \exists\xi\in[a;b]\colon c=f(\xi)$$
% ---
{\bold Доказательство.} По принципу Кантора:
% ---
$$\forall n\in\mathbb{N}\ \exists\xi\in[a_n;b_n]\subset[a_{n-1};b_{n-1}]\subseteq X
\implies$$
$$n\to\infty,\ a_n,b_n\to\xi$$
% ---
По определению непрерывности функции на промежутке:
% ---
$$n\to\infty,\ f(a_n),f(b_n)\to f(\xi)$$
% ---
По теореме о промежуточной функции:
% ---
$$f(a_n)\leq c\leq f(b_n)\implies c=f(\xi)\qedb$$
