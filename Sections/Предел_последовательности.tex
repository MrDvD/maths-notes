\section{Предел последовательности}

% ОПРЕДЕЛЕНИЕ ПОДПОСЛЕДОВАТЕЛЬНОСТИ

\subsection{Предел}

{\bold Предел} последовательности $\{x_n\}$ --- такое $a$, что
% ---
$$\forall\varepsilon\greater 0\ \exists N\colon\forall n\greater N\ x_n\in U_
\varepsilon(a).$$
% ---
Упрощённая запись $\lim_{n\to\infty}x_n=a$ или $n\to\infty$, $x_n\to a$.

Этот оператор {\ital дистрибутивен} относительно {\ital сложения}, {\ital умножения}.

{\ital Частичным} называется предел подпоследовательности.

\subsection{Свойства предела}

Сходимость $\implies$ ограниченность.

{\bold Доказательство.} Пусть $\lim_{n\to\infty}x_n=a$. По определению:
% ---
$$\forall\varepsilon\greater 0\ \exists N\colon\forall n\greater N\ x_n\in U_
\varepsilon(a)$$
% ---
По «дистрибуции» модуля относительно сложения:
% ---
$$\abs{x_n}=\abs{x_n-a+a}\leq\abs{x_n-a}+\abs{a}\less\varepsilon+\abs{a}$$
% ---
Положим, что $\forall m\leq N\ L=\max(\abs{\{x_m\}},\varepsilon+\abs{a})\implies\abs{x_n}
\leq L$.$\qedb$

Пусть $n\to\infty$, $x_n\to a$, $y_n\to b$. Выполняя {\ital предельный переход}, 
при $x_n\leq y_n$ или $x_n\less y_n$ сохраняется неравенство $a\leq b$.

{\bold Доказательство.} По определению предела:
% ---
$$\forall\varepsilon\greater 0\ \exists N\colon\forall n\greater N\ x_n\in U_
\varepsilon(a),\ y_n\in U_\varepsilon(b)$$
% ---
Следовательно,
% ---
$$+\begin{cases}
x_n\leq y_n\\
a-x_n\less\varepsilon\\
y_n-b\less\varepsilon
\end{cases}\hspace*{-12pt}\iff
\begin{cases}
y_n-x_n\geq 0\\
y_n-x_n\less 2\varepsilon+b-a
\end{cases}\hspace*{-12pt}\iff
\frac{a-b}{2}\less\varepsilon$$
% ---
Так как $\varepsilon$ --- сколь угодно малое положительное число, то $a-b\leq 0\iff a\leq 
b$.$\qedw$

При $x_n\less y_n$ доказательство аналогично.$\qedb$

Пусть $n\to\infty$, $x_n,y_n\to a$. Тогда при $\forall\{z_n\}\colon x_n\leq z_n\leq y_n$ 
справедливо $z_n\to a$. {\ital\color{desc}(теорема о промежуточной функции)}

{\bold Доказательство.} По определению предела:
% ---
$$\forall\varepsilon\greater 0\ \exists N\colon\forall n\greater N\ x_n,y_n\in U_
\varepsilon(a)$$
% ---
Следовательно,
% ---
$$a-\varepsilon\less x_n\leq z_n\leq y_n\less a+\varepsilon\implies z_n\in U_
\varepsilon(a)\implies\lim_{n\to\infty}z_n=a.\qedb$$

\subsection{Условие Коши}

Последовательность $\{x_n\}$ удовлетворяет {\ital условию Коши} {\ital\color{desc} 
(является фундаментальной)}, если
% ---
$$\forall\varepsilon\greater 0\ \exists N\colon\forall n,m\greater N\ \abs{x_n-x_m}\less
\varepsilon.$$
% ---
Фундаментальность $\implies$ ограниченность.

{\bold Доказательство.} По условию Коши:
% ---
$$\forall\varepsilon\greater 0\ \exists N\colon\forall n,m\greater N\ \abs{x_n-x_m}\less
\varepsilon$$
% ---
По «дистрибуции» модуля относительно сложения:
% ---
$$\begin{cases}
\abs{x_n-x_m}\less\varepsilon\\
\abs{x_n}=\abs{x_n-x_m+x_m}
\end{cases}\hspace*{-12pt}\iff
\begin{cases}
\abs{x_n-x_m}\less\varepsilon\\
\abs{x_n}\leq\abs{x_n-x_m}+\abs{x_m}
\end{cases}\hspace*{-12pt}\iff$$
$$\abs{x_n}\less\varepsilon+\abs{x_m}$$
% ---
Положим, что $\forall k\leq N\ L=\max(\abs{\{x_k\}},\varepsilon+\abs{x_m})\implies
\abs{x_n}\leq L$.$\qedb$

\subsection{Принцип компактности отрезка}

Ограниченность $\implies$ частичная сходимость:
% ---
$$\forall\{x_n\}\in[a;b]\ \exists\{n_k\}\kern-4pt\uparrow\colon\lim_{k\to\infty}x_{n_k}=
\xi$$
% ---
{\bold Доказательство.} По принципу Кантора:
% ---
$$\forall k\in\mathbb{N}\ \exists!\xi\in[a_k;b_k]\subset[a_{k-1};b_{k-1}]\iff$$
$$\lim_{k\to\infty}a_k=\lim_{k\to\infty}b_n=\xi$$
% ---
Образуем подпоследовательность:
% ---
$$\{x_{n_k}\mid \{n_k\}\kern-4pt\uparrow,\ x_{n_k}\in[a_k;b_k]\}$$
% ---
По теореме о промежуточной функции:
% ---
$$a_k\leq x_{n_k}\leq b_k\implies \lim_{k\to\infty}x_{n_k}=\xi\qedb$$
% ---
Частичный предел фундаментальной последовательности является её пределом.

{\bold Доказательство.} Пусть $\{x_n\}$ фундаментальна $\implies$ она ограничена.

По принципу компактности отрезка $\lim_{k\to\infty}x_{n_k}=a$.

По условию Коши:
% ---
$$\forall\varepsilon/2\greater 0\ \exists N\colon\forall n,m\greater N\ \abs{x_n-x_m}
\less\varepsilon/2$$
% ---
Зафиксируем $n$. При $x_m=x_{n_k}\greater N$ перейдём к пределу:
% ---
$$\abs{x_n-a}\leq\varepsilon/2\less\varepsilon\iff\lim_{k\to\infty}x_n=a\qedb$$

\subsection{Критерий Коши}

Сходимость $\iff$ фундаментальность.

{\bold Доказательство $\implies$.} По определению предела:
% ---
$$\forall\varepsilon\greater 0\ \exists N\colon\forall n\greater N\ x_n\in U_
{\varepsilon/2}(a)$$ 
% ---
Значит, $\forall n,m\greater N\ \abs{x_n-x_m}=\abs{(x_n-a)+(a-x_m)}$.

По «дистрибуции» модуля относительно сложения:
% ---
$$\abs{x_n-x_m}\leq\abs{x_n-a}+\abs{x_m-a}\less \varepsilon/2+\varepsilon/2=\varepsilon
\qedb$$
% ---
{\bold Доказательство $\impliedby$.} Пусть $\{x_n\}$ фундаментальна $\implies$ она 
ограничена $\implies$ по принципу компактности отрезка она частично сходится к $c$
$\implies$ по условию Коши и принципу компактности отрезка она сходится к $c$.$\qedb$

\subsection{Теорема Вейершстрасса}

Монотонность $\implies$ сходимость:
% ---
$$\begin{sqcases}
\forall\{x_n\}\kern-4pt\nearrow\lim_{n\to\infty}x_n=\sup\{x_n\}\\
\forall\{y_n\}\kern-4pt\searrow\lim_{n\to\infty}y_n=\inf\{y_n\}
\end{sqcases}$$
% ---
{\bold Доказательство.} По определению точной верхней границы:
% ---
$$\forall n\in\mathbb{N}\ x_n\leq\sup\{x_n\}$$
% ---
Так как последовательность неубывает, то
% ---
$$\forall\epsilon\greater 0\ \exists N\colon\forall n\greater N\ x_n\in U_\epsilon(\sup
\{x_n\})\implies$$
$$\lim_{n\to\infty}x_n=\sup\{x_n\}.\qedw$$
% ---
Для $\{y_n\}\kern-4pt\searrow$ доказательство аналогично.$\qedb$
