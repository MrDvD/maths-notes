\section{Алгебра логики}

\subsection{Определение}

{\bold Алгебра логики} --- алгебраическая структура, которая образована двухэлементным множеством $\{0;1\}$.

{\bold Высказывание} --- повествовательное предложение, о кото"=ром можно сказать в данный момент, {\ital истинно} ли оно или {\ital ложно}.

{\bold Предикат} {\ital (одноместный)} --- функция переменной $x$, которое принимает значение из множества $\{0;1\}$.

\begin{theorem}
Высказывание --- {\ital нуль-}местный предикат.
\end{theorem} 

{\bold Логическая связка} --- операция алгебры логики:
% ---
\begin{list*}[][\#]
\item{\ital Инверсия} «$\lnot$» --- логическое {\ital «не»}.
\item{\ital Конъюнкция} «$\land$» --- логическое {\ital «и»}.
\item{\ital Дизъюнкция} «$\lor$» --- логическое {\ital «или»}.
\item{\ital Строгая дизъюнкция} «$\dot{\lor}$» --- логическое {\ital «искл. или»}.
\item{\ital Импликация} «$\rightarrow$» --- логическое {\ital «$\implies$»}.
\item{\ital Эквиваленция} «$\equiv$» --- логическое {\ital «$\iff$»}.
\end{list*}

{\bold Квантор} --- логическая операция:
% ---
\begin{list*}
\item{\ital квантор всеобщности} «$\forall$»
\item{\ital квантор существования} «$\exists$»
\end{list*}

\subsection{Свойства}

Конъюнкция и дизъюнкция {\ital коммутативны}, {\ital ассоциативны} и {\ital дистрибутивны} относительно друг друга.
% ---
\begin{theorem}
{\bold Идемпотентность.} {\ital\color{desc} (Тавтология)}
% ---
$$A\land A=A\qquad A\lor A=A$$
\end{theorem}
% ---
\begin{theorem}
{\bold Инволютивность.}
% ---
$$\bar{\bar{A}}=A$$
\end{theorem}
% ---
\begin{theorem}
{\bold Закон противоречия} и {\bold исключённого третьего.}
% ---
$$A\land\overline{A}=0\qquad A\lor\overline{A}=1$$
\end{theorem}
% ---
\begin{theorem}
{\bold Закон поглощения.}
% ---
$$\begin{gathered}\begin{aligned}
&A\land 1=A &\quad &A\land 0=0\\
&A\lor 1=1 &\quad &A\lor 0=A
\end{aligned}\\
A\land (A\lor B)=A\qquad A\lor (A\land B)=A\end{gathered}$$
\end{theorem}
% ---
\begin{theorem}
{\bold Закон де Мóргана.} {\ital\color{desc} (Двойственность)}
% ---
$$\overline{A\land B}=\overline{A}\lor\overline{B}\qquad\overline{A\lor B}=\overline{A}\land\overline{B}$$
\end{theorem}
% ---
\begin{theorem}
{\bold Закон склеивания.}
% ---
$$\begin{aligned}
&(A\lor B)\land(A\lor \overline{B})=A\\
&(A\land B)\lor(A\land \overline{B})=A\\
\end{aligned}$$
\end{theorem}
% ---
\begin{theorem}
{\bold Закон сокращения.} {\ital\color{desc} (Порецкий)}
% ---
$$\begin{aligned}
&A\lor(\bar{A}\land B)=A\lor B\\
&A\land(\bar{A}\lor B)=A\land B
\end{aligned}$$
\end{theorem}
% ---
\begin{theorem}
{\bold Силлогизм.}
% ---
$$(A\to B)\land(B\to C)=A\to C$$
\end{theorem}

\subsection{Булева функция}

{\bold Булева функция} --- функция вида:
% ---
$$f\colon\{0,1\}^n\to\{0,1\}$$
% ---
Причём справедливы утверждения:
% ---
$$\begin{gathered}
\abs{\{0,1\}^n}=2^n\\
\abs{\{f\mid f\colon\{0,1\}^n\to\{0,1\}\}}=2^{2^n}
\end{gathered}$$
% ---
\begin{theorem}
{\bold Определение.} Примем сокращения:
% ---
\begin{list*}[][\#]
\item $x_1\Delta x_2=\overline{x_1\to x_2}$ --- запрет $x_1$ по $x_2$
\item $x_1\downarrow x_2=\overline{x_1\lor x_2}$ --- стрелка Пирса {\ital (функция Вебба)}
\item $x_1\mid x_2=\overline{x_1\land x_2}$ --- штрих Шеффера
\end{list*}
\end{theorem}
% ---
У {\bold вырожденной} булевой функции по аргументу $x_i$ значение {\ital не зависит} от $x_i$ при любом наборе аргументов:
% ---
$$f(x_1,\dots,0,\dots,x_n)=f(x_1,\dots,1,\dots,x_n)$$
% ---
К таким бинарным функциям относятся {\ital 6 отображений}:
% ---
\begin{list*}
\item {\bold тавтология}, логическая единица
\item {\bold противоречие}, логический нуль
\item {\bold повтор} $x_1$ и $x_2$
\item {\bold отрицание} $x_1$ и $x_2$
\end{list*}

\subsection{Нормальная форма}

{\bold Элементарная конъюнкция} {\ital (конъюнктивный терм)} --- конъюнкция попарно различимых переменных или их отрицаний.

{\bold Элементарная дизъюнкция} {\ital (дизъюнктивный терм)} --- дизъюнкция попарно различимых переменных или их отрицаний.

{\bold Ранг терма} --- число переменных, которые в него входят.

{\bold Дизъюнктивная нормальная форма} {\ital (ДНФ)} --- дизъюнкция конъюнктивных термов.

{\bold Конъюнктивная нормальная форма} {\ital (КНФ)} --- конъюнкция дизъюнктивных термов:
% ---
\begin{theorem}
\begin{tabularcx}{0pt}{0pt}{C@{\hspace*{-16pt}}C}{\textwidth}
$\begin{aligned}
A\dot{\lor}B&=(A\lor B)\land(\overline{A}\lor\overline{B})\\
A\equiv B&=(\overline{A}\lor B)\land(A\lor\overline{B})
\end{aligned}$\hspace*{-18pt} & $A\rightarrow B=\overline{A}\lor B$
\end{tabularcx}
\end{theorem}
% ---
{\bold Импликативная нормальная форма} {\ital (INF)} --- конъюнкция дизъюнктивных термов любого ранга, которые заменены импликацией:
% ---
$$A\lor B=(\overline{A}\rightarrow B)\land(\overline{B}\to A)$$
% ---
{\bold Конституента единицы} {\ital (нуля)} --- конъюнктивный {\ital (дизъюнктивный)} терм максимального ранга.

\begin{theorem}
{\bold Свойство.} Конституента единицы {\ital (нуля)} принимает значение единицы {\ital (нуля)} на одном и только на одном наборе аргументов.
\end{theorem}

В {\bold канонической {\ital (совершенной)} КНФ} все входящие дизъюнктивные термы --- конституенты нуля.

\begin{theorem}
{\bold Теорема.} Всякая булева функция, кроме 1, имеет\\ единственную ККНФ.
\end{theorem}

В {\bold канонической {\ital (совершенной)} ДНФ} все входящие конъюнктивные термы --- конституенты единицы.

\begin{theorem}
{\bold Теорема.} Всякая булева функция, кроме 0, имеет\\ единственную КДНФ.
\end{theorem}

\subsection{Импликанта}

{\bold Импликанта} $f$ --- такая булева функция $g$, что:
% ---
$$\forall X\colon g(X)=1\implies f(X)=1$$
% ---
{\bold Элементарная импликанта} --- такая импликанта, которая выражена элементарной конъюнкцией.

{\bold Простая {\ital (собственная)} импликанта} --- конъюнктивный импликантный терм, никакая собственная часть которого не~является импликантой.

{\bold Существенная импликанта} --- простая импликанта, если она и только она покрывает конституенту единицы. 

{\bold Полная система импликант} --- множество импликант данной функции, у которой все конституенты единицы покрыты хотя бы одной импликантой.

{\bold Приведённая система импликант} --- полная система импликант, никакое подмножество которой не является полной системой. 

{\bold Сокращённая ДНФ} --- дизъюнкция полной системы простых импликант данной функции.

{\bold Тупиковая ДНФ} --- дизъюнкция приведённой системы простых импликант данной функции.

{\bold Минимальная ДНФ} --- такая СДНФ, в которой число входящих переменных минимально.

\subsection{Имплицента}

{\bold Имплицента} $f$ --- такая булева функция $g$, что:
% ---
$$\forall X\colon g(X)=0\implies f(X)=0$$
% ---
{\bold Элементарная имплицента} --- такая имплицента, которая выражена элементарной дизъюнкцией.

{\bold Простая {\ital (собственная)} имплицента} --- дизъюнктивный имплицентный терм, никакая собственная часть которого не~является имплицентой.

{\bold Существенная имплицента} --- простая имплицента, если она и только она покрывает конституенту нуля. 

{\bold Полная система имплицент} --- множество имплицент данной функции, у которой все конституенты нуля покрыты хотя бы одной имплицентой.

{\bold Приведённая система имплицент} --- полная система имплицент, никакое подмножество которой не является полной системой. 

{\bold Сокращённая КНФ} --- конъюнкция полной системы простых имплицент данной функции.

{\bold Тупиковая КНФ} --- конъюнкция приведённой системы простых имплицент данной функции.

{\bold Минимальная КНФ} --- такая СКНФ, в которой число входящих переменных минимально.

\subsection{Способы задания функции}

Пусть $f\colon\{0,1\}^n\to\{0,1\}$.

К {\ital основным способам} задания функции относятся:
% ---
\begin{list*}
\item аналитическая форма
\item таблица истинности
\end{list*}
% ---
{\bold Числовая форма} --- перечисление множества десятичных эквивалентов набора аргументов, на котором функция равна единице {\ital (нулю)}:
% ---
$$\begin{gathered}
f^3(X)=\lor(0,2,6,7)\sim\text{КДНФ}\\
f^3(X)=\land(1,3,4,5)\sim\text{ККНФ}
\end{gathered}$$
% ---
{\bold Символическая форма} --- десятичный эквивалент упорядоченного набора значений функции:
% ---
$$f(X)=f^3_{105}\text{ {\ital\color{desc}(чтение с начала таблицы)}}$$ 
% ---
{\bold Графическая форма} --- отображение множества векторов из области определения во множество вершин $n$-мерного {\ital булевого куба}.

{\bold Существенной} называется такая вершина булевого куба, в~чьих координатах функция {\ital истинна}.

{\bold $\symbf{0}$-куб} --- объект, который соответствует элементарной импликанте и покрывает существенную вершину.

Для {\ital кубов} определены следующие отношения:
% ---
\begin{list*}
\item {\bold отношение соседства}: кубы отличаются лишь {\ital одной} координатой:
\begin{list*}[2]
\item {\ital свободная координата {\color{desc}(X)}} различается с соседом
\item {\ital связанная координата} совпадает с соседом
\end{list*}
\item {\bold операция склеивания}: соседние вершины образуют $1$-куб
\item {\bold отношение покрытия}: куб является частью куба большей размерности
\end{list*}
% ---
{\bold Кубический комплекс} $K^n(f)$ --- множество всех $n$-кубов функции $f$:
% ---
$$K(f)=\bigcup_{i}K^i(f)$$
% ---
{\bold Покрытие} $C(f)$ --- подмножество $K(f)$, которое покрывает все существенные вершины функции $f$:
% ---
$$C(f)\sim\text{полная система элементарных импликант}$$
% ---
{\bold Максимальный куб} --- куб кубического комплекса, который {\ital нельзя склеить} с другими кубами:
% ---
$$\text{максимальный куб}\sim\text{простая импликанта}$$
% ---
{\bold Множество максимальных кубов} $Z(f)$:
% ---
$$Z(f)\sim\text{полная система простых импликант}$$
% ---
{\bold Ядро покрытия} $T(f)$ --- множество максимальных кубов, без которых покрытие не будет образовано:
% ---
$$T(f)\sim\text{множество существенных импликант}$$

\subsection{Минимизация функции}

\begin{theorem}
{\bold Метод Квайна-Мак-Класски.}
% ---
\begin{list*}[][\#]
\item Выписать и упорядочить $0$-кубы функции по числу единиц.
\item Склеить все возможные кубы в $1$-кубы.
\item Повторять операции, пока не образуется $K^n(f)=\emptyset$.
\item Получить множество $Z(f)$, из которого образовать минимальное покрытие. 
\end{list*}
\end{theorem}

\subsection{Функциональная полнота системы}

Система булевых функций {\bold функционально полная}, если любую булева функция выражается композицией функций системы.

{\bold Замкнутый класс} --- это...

{\bold Предполный класс} --- это...

Класс {\bold сохраняющих ноль} функций $T_0$ --- для $f$ верно:
% ---
$$f(0,\dots,0)=0$$
% ---
К ним относятся:
% ---
$$f^2_0,\ f^2_1,\ f^2_2,\ f^2_3,\ f^2_4,\ f^2_5,\ f^2_6,\ f^2_7$$
% ---
Класс {\bold сохраняющих единицу} функций $T_1$ --- для $f$ верно:
% ---
$$f(1,\dots,1)=1$$
% ---
К ним относятся:
% ---
$$f^2_1,\ f^2_3,\ f^2_5,\ f^2_7,\ f^2_9,\ f^2_{11},\ f^2_{13},\ f^2_{15}$$
% ---
Класс {\bold самодвойственных функций} $S$ --- для $f$ верно:
% ---
$$\forall x_i\colon f(x_1,\dots,x_n)=\overline{f(\overline{x_1},\dots,\overline{x_n})}$$
% ---
К ним относятся:
% ---
$$x_1,\ x_2,\ \overline{x_1},\ \overline{x_2}$$
% ---
Класс {\bold линейных функций} $L$ --- $f$ представляется {\ital линейным} полиномом Жегалкина:
% ---
$$f(x_1,\dots,x_n)=a_0\oplus a_1x_1\oplus\dots\oplus a_nx_n,\quad a_i\in\{0,1\}$$
% ---
К ним относятся:
% ---
$$0,\ 1,\ x_1,\ x_2,\ \overline{x_1},\ \overline{x_2},\ \oplus,\ \sim$$
% ---
Класс {\bold монотонных функций} $M$ --- для $f$ верно:
% ---
$$\forall X_i\prec X_j\colon f(X_i)\leq f(X_j)$$
% ---
К ним относятся:
% ---
$$0,\ 1,\ x_1,\ x_2,\ \land,\ \lor$$
% ---
https://www.mathnet.ru/links/06feed3dd44de28a750d3d92ac80d6ec/da399.pdf

Частичный порядок

\begin{theorem}
{\bold Теорема Поста-Яблонского.} Система булевых функций {\ital функционально полная}$\iff$она не содержится полностью ни в одном из классов $T_0$, $T_1$, $S$, $L$, $M$.
\end{theorem}

\begin{theorem}
{\bold Булевый базис} --- функционально полная система:
% ---
$$\{\neg,\land,\lor\}$$
\end{theorem}

{\bold Доказательство.} ...

\begin{theorem}
{\bold Конструктивный подход.} Если функционально полная система булевых функций есть композиция набора других функций он тоже {\ital функционально полный}.
\end{theorem}

\begin{theorem}
{\bold Универсальный базис} --- функционально полная система:
% ---
$$\{\downarrow\}\qquad\{\mid\}$$
\end{theorem}

{\bold Доказательство.} По конструктивному подходу для стрелки Пирса из булевого базиса:
% ---
$$\begin{gathered}
\overline{x}=x\downarrow x\\
x_1\land x_2=(x_1\downarrow x_1)\downarrow(x_2\downarrow x_2)\\
x_1\lor x_2=(x_1\downarrow x_2)\downarrow(x_1\downarrow x_2)
\end{gathered}$$
% ---
Значит, $\{\downarrow\}$ --- функционально полный базис.$\qedw$

По конструктивному подходу для штриха Шиффера из~булевого базиса:
% ---
$$\begin{gathered}
\overline{x}=x\mid x\\
x_1\land x_2=(x_1\mid x_2)\mid(x_1\mid x_2)\\
x_1\lor x_2=(x_1\mid x_1)\mid(x_2\mid x_2)
\end{gathered}$$
% ---
Значит, $\{\mid\}$ --- функционально полный базис.$\qedb$

\subsection{Синтез схем}

{\bold Цена схемы} --- количество оборудования, которое использовано в схеме:
% ---
$$S=\sum s_in_i$$
% ---
\begin{list*}
\item $s_i$ --- цена $i$-го компонента
\item $n_i$ --- количество $i$-го компонента
\end{list*}
% ---
{\bold Цена схемы по Квайну} $S_Q$ --- суммарное число входов у~логических элементов.

{\bold Цена $\symbf{n}$-куба} --- число его связанных координат.

{\bold Минимальной} называется такая нормальная форма, которая {\ital минимизирует} цену схемы.

{\bold Минимальным} называется такое покрытие, которое состоит из {\ital минимальных} нормальных форм.

{\bold Цена покрытия} $S^a$ --- сумма цен кубов, которые входят в~покрытие.

{\bold Цена покрытия} $S^b$ --- сумма $S^a$ и количества кубов, которые входят в покрытие.

\begin{theorem}
{\bold Теорема.} Связь цены схемы по Квайну с ценой покрытия:
% ---
$$S^a\leq S_Q\leq S^b$$
\end{theorem}
