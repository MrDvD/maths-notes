\section{Алгебра логики}

\subsection{Определение}

{\bold Алгебра логики} --- алгебраическая структура, которая образована двухэлементным множеством $\{0;1\}$.

{\bold Высказывание} --- повествовательное предложение, о кото"=ром можно сказать в данный момент, {\ital истинно} ли оно или {\ital ложно}.

{\bold Предикат} {\ital (одноместный)} --- функция переменной $x$, которое принимает значение из множества $\{0;1\}$.

\begin{theorem}
Высказывание --- {\ital нуль-}местный предикат.
\end{theorem} 

{\bold Логическая связка} --- операция алгебры логики:
% ---
\begin{list*}[][\#]
\item{\ital Инверсия} «$\lnot$» --- логическое {\ital «не»}.
\item{\ital Конъюнкция} «$\land$» --- логическое {\ital «и»}.
\item{\ital Дизъюнкция} «$\lor$» --- логическое {\ital «или»}.
\item{\ital Строгая дизъюнкция} «$\dot{\lor}$» --- логическое {\ital «искл. или»}.
\item{\ital Импликация} «$\rightarrow$» --- логическое {\ital «$\implies$»}.
\item{\ital Эквиваленция} «$\equiv$» --- логическое {\ital «$\iff$»}.
\end{list*}

{\bold Квантор} --- логическая операция:
% ---
\begin{list*}
\item{\ital квантор всеобщности} «$\forall$»
\item{\ital квантор существования} «$\exists$»
\end{list*}

\subsection{Свойства}

Конъюнкция и дизъюнкция {\ital коммутативны}, {\ital ассоциативны} и {\ital дистрибутивны} относительно друг друга.
% ---
\begin{theorem}
{\bold Идемпотентность.} {\ital\color{desc} (Тавтология)}
% ---
$$A\land A=A\qquad A\lor A=A$$
\end{theorem}
% ---
\begin{theorem}
{\bold Инволютивность.}
% ---
$$\bar{\bar{A}}=A$$
\end{theorem}
% ---
\begin{theorem}
{\bold Закон противоречия} и {\bold исключённого третьего.}
% ---
$$A\land\overline{A}=0\qquad A\lor\overline{A}=1$$
\end{theorem}
% ---
\begin{theorem}
{\bold Закон поглощения.}
% ---
$$\begin{gathered}\begin{aligned}
&A\land 1=A &\quad &A\land 0=0\\
&A\lor 1=1 &\quad &A\lor 0=A
\end{aligned}\\
A\land (A\lor B)=A\qquad A\lor (A\land B)=A\end{gathered}$$
\end{theorem}
% ---
\begin{theorem}
{\bold Закон де Мóргана.} {\ital\color{desc} (Двойственность)}
% ---
$$\overline{A\land B}=\overline{A}\lor\overline{B}\qquad\overline{A\lor B}=\overline{A}\land\overline{B}$$
\end{theorem}
% ---
\begin{theorem}
{\bold Закон склеивания.}
% ---
$$\begin{aligned}
&(A\lor B)\land(A\lor \overline{B})=A\\
&(A\land B)\lor(A\land \overline{B})=A\\
\end{aligned}$$
\end{theorem}
% ---
\begin{theorem}
{\bold Закон сокращения.} {\ital\color{desc} (Порецкий)}
% ---
$$\begin{aligned}
&A\lor(\bar{A}\land B)=A\lor B\\
&A\land(\bar{A}\lor B)=A\land B
\end{aligned}$$
\end{theorem}
% ---
\begin{theorem}
{\bold Силлогизм.}
% ---
$$(A\to B)\land(B\to C)=A\to C$$
\end{theorem}

\subsection{Булева функция}

{\bold Булева функция} --- функция вида:
% ---
$$f\colon\{0,1\}^n\to\{0,1\}$$
% ---
Причём справедливы утверждения:
% ---
$$\begin{gathered}
\abs{\{0,1\}^n}=2^n\\
\abs{\{f\mid f\colon\{0,1\}^n\to\{0,1\}\}}=2^{2^n}
\end{gathered}$$
% ---
У {\bold вырожденной} булевой функции по аргументу $x_i$ значение {\ital не зависит} от $x_i$ при любом наборе аргументов:
% ---
$$f(x_1,\dots,0,\dots,x_n)=f(x_1,\dots,1,\dots,x_n)$$
% ---
К таким бинарным функциям относятся {\ital 6 отображений}:
% ---
\begin{list*}
\item {\bold тавтология}, логическая единица
\item {\bold противоречие}, логический нуль
\item {\bold повтор} $x_1$ и $x_2$
\item {\bold отрицание} $x_1$ и $x_2$
\end{list*}

\subsection{Способы задания функции}

\begin{theorem}
{\bold Способы задания функции.}
% ---
\begin{list*}
\item аналитическая форма
\item таблицей истинности
\item графическая форма
\item таблично-графическая форма
\item числовая форма
\item символическая форма
\end{list*}
\end{theorem}

При {\bold графическом} задании булевых функции от $n$ аргументов её область определения отображают в виде $n$-мерного {\ital единичного куба}.

{\bold Существенной} называется такая вершина $n$-куба, в чьих координатах функция {\ital истинна}.

Такие вершины образуют {\bold $\symbf{0}$-кубы}, между которыми определены следующие отношения:
% ---
\begin{list*}
\item {\bold отношение соседства}: $0$-кубы отличаются лишь {\ital одной} координатой
\item {\bold операция склеивания}: соседние вершины образуют $1$-куб
\end{list*}

\subsection{Нормальная форма}

{\bold Элементарная конъюнкция} {\ital (конъюнктивный терм)} --- конъюнкция попарно различимых переменных или их отрицаний.

{\bold Элементарная дизъюнкция} {\ital (дизъюнктивный терм)} --- дизъюнкция попарно различимых переменных или их отрицаний.

{\bold Ранг терма} --- число переменных, которые в него входят.

{\bold Дизъюнктивная нормальная форма} {\ital (ДНФ)} --- дизъюнкция конъюнктивных термов.

{\bold Конъюнктивная нормальная форма} {\ital (КНФ)} --- конъюнкция дизъюнктивных термов:
% ---
\begin{theorem}
\begin{tabularcx}{0pt}{0pt}{C@{\hspace*{-16pt}}C}{\textwidth}
$\begin{aligned}
A\dot{\lor}B&=(A\lor B)\land(\overline{A}\lor\overline{B})\\
A\equiv B&=(\overline{A}\lor B)\land(A\lor\overline{B})
\end{aligned}$\hspace*{-18pt} & $A\rightarrow B=\overline{A}\lor B$
\end{tabularcx}
\end{theorem}
% ---
{\bold Импликативная нормальная форма} {\ital (INF)} --- конъюнкция дизъюнктивных термов любого ранга, которые заменены импликацией:
% ---
$$A\lor B=(\overline{A}\rightarrow B)\land(\overline{B}\to A)$$
% ---
{\bold Конституента единицы} {\ital (нуля)} --- конъюнктивный {\ital (дизъюнктивный)} терм максимального ранга.

\begin{theorem}
{\bold Свойство.} Конституента единицы {\ital (нуля)} принимает значение единицы {\ital (нуля)} на одном и только на одном наборе аргументов.
\end{theorem}

В {\bold канонической {\ital (совершенной)} КНФ} все входящие дизъюнктивные термы --- конституенты нуля.

\begin{theorem}
{\bold Теорема.} Всякая булева функция, кроме 1, имеет\\ единственную СКНФ.
\end{theorem}

В {\bold канонической {\ital (совершенной)} ДНФ} все входящие конъюнктивные термы --- конституенты единицы.

\begin{theorem}
{\bold Теорема.} Всякая булева функция, кроме 0, имеет\\ единственную СДНФ.
\end{theorem}

\begin{theorem}
{\bold Следствие.} Любая булева функция может быть записана в {\ital булевом базисе} $\{\land,\lor,\neg\}$.
\end{theorem}
