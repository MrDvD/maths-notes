\section{Информатика}

\subsection{Кодирование информации}

{\bold Код} ---

{\bold Кодирование} ---

блочный, свёрточный
равномерный, неравномерный
разделимый, неразделимый

\subsection{Помехоустойчивый код}

{\bold Помехоустойчивый код} обнаруживает ошибки в кодовых словах, которые возникают при передаче.

{\bold Самокорректирующийся код} автоматически исправляет ошибки в кодовых словах, которые возникают при передаче.

{\bold Контрольная сумма} --- значение, которое рассчитано для данных и используется для проверки их целостности при хранении, передаче.

хеш-функция?
CRC, MD5

\subsection{Код Хэмминга}

{\bold Код Хэмминга} --- реализация блочного равномерного разделимого самокорректирующегося кода, которая исправляет одиночные битовые ошибки, возникшие при передаче или хранении данных.

{\bold Синдром ошибок} --- вектор контрольных сумм информационных и проверочных разрядов.
