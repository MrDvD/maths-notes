\section{Структуры данных}

\subsection{Стек}

\begin{theorem}
{\bold Задача.} Вычислить область самого большого прямоугольника в гистограмме, который находится на общей базовой линии.
\end{theorem}

{\bold Идея.} Пока гистограмма {\ital строго возрастает}, добавлять столбцы в стек.

Если высота $i$-го столбца $\leq$ верхнему элементу стека:
% ---
\begin{list*}[][\#]
\item Убирать из стека столбцы, пока гистограмма не станет {\ital строго возрастающей}.
\item Считать площадь прямоугольника от $i-1$ до убранного столбца $x$ {\ital\color{desc}(брать высоту как локальный минимум)}.
\end{list*}
% ---
Если гистограмма {\ital не стала} строго возрастающей, {\ital  вернуть} последний убранный столбец в стек.
