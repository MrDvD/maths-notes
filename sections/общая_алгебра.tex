\section{Общая алгебра}

\subsection{Алгебраическая операция}

$n$-местная {\bold алгебраическая операция} $\ast$ --- отображение:
% ---
$$\ast\colon X_1\times\dots\times X_n\to Y$$
% ---
Операция $\ast$ {\bold внутренняя} для множества $X$ {\ital (множество $X$ {\boldital замкнуто} относительно операции $\ast$)}, если:
% ---
$$X_i\subseteq X\land Y\subseteq X$$
% ---
Иначе операция $\ast$ {\bold внешняя} для множества $X$ {\ital (множество $X$ {\boldital не замкнуто} относительно операции $\ast$)}.

{\bold Нейтральным} от-но $\ast$ наз-ся такой элемент $e\in X$, что:
% ---
$$\forall x\in X\ e\ast x=x\ast e=x$$
% ---
\begin{theorem}
Если нейтральный элемент $e$ от-но $\ast$ существует, то он {\ital единственный}.
\end{theorem}
% ---
{\bold Поглощающим} от-но $\ast$ наз-ся такой элемент $w\in X$, что:
% ---
$$\forall x\in X\ w\ast x=x\ast w=w$$
% ---
\begin{theorem}
Если поглощающий элемент $w$ от-но $\ast$ существует, то он {\ital единственный}.
\end{theorem}
% ---
{\bold Симметричным} {\ital (противоположным, обратным)} к~элементу $x\in X$ от-но $\ast$ наз-ся такой элемент $x^{-1}\in X$, что:
% ---
$$x\ast x^{-1}=x^{-1}\ast x=e\text{ {\ital\color{desc}(нейтр.)}}$$
% ---
\begin{theorem}
Если симметричный элемент $x^{-1}$ к элементу $x$ от-но ассоциативной $\ast$ существует, то он {\ital единственный}.
\end{theorem}
% ---
\begin{theorem}
{\bold Теорема.} Если $y\colon X\to Y$ и $x\colon Y\to X$ --- отображения, то $y$ 
{\ital инъективно}, а $x$ {\ital сюръективно}.
\end{theorem}

{\bold Доказательство.} По условию, множество $X$ накладывается на себя.
Значит, $f$ {\ital всюду определено}.\par

Так как $g$ функционально, то
% ---
$$\forall x_1,x_2\in X\ \exists y\in E_f\colon x_1fy,\ x_2fy\iff x_1=x_2,$$
% ---
то есть $f$ {\ital инъективно}.$\qedw$\par

Когда $X$ накладывается на себя, то
% ---
$$\forall x\in E_g\ \exists y\in D_g\colon x\mapsto y,$$
% ---
то есть $g$ {\ital сюръективно}.$\qedb$

\subsection{Свойства операций}

Пусть $\ast\colon X^2\to X$, $\circ\colon X^2\to X$ --- алгебраические операции.

Операция $\ast$ {\bold коммутативна}, когда
% ---
$$\forall x,y\in X\ x\ast y=y\ast x$$
% ---
Операция $\ast$ {\bold ассоциативна}, когда
% ---
$$\forall x,y,z\in X\ (x\ast y)\ast z=x\ast(y\ast z)$$
% ---
Операция $\ast$ {\bold дистрибутивна} относительно $\circ$, когда
% ---
$$\forall x,y,z\in X\ 
\begin{cases*}
&x\ast(y\circ z)=(x\ast y)\circ (x\ast z)\text{ {\ital\color{desc} (дистр. справа)}}\\
&(x\circ y)\ast z=(x\ast z)\circ(y\ast z)\text{ {\ital\color{desc} (дистр. слева)}}
\end{cases*}$$
% ---
\begin{theorem}
{\bold Теорема.} Пусть $e$ --- нейтральный элемент от-но операции $\ast$, дистрибутивной от-но операции $\circ$. Тогда $e$ --- {\ital поглощающий элемент} от-но $\circ$.
\end{theorem}

{\ital (Работает ли теорема выше в обратную сторону?)}

\subsection{Алгебраическая структура}

{\bold Алгебраическая структура} {\ital (система)} --- множество $X$ с~введёнными на нём 
алгебраическими операциями:
% ---
$$(X,\ast_1,\dots,\ast_n)$$
% ---
Для {\bold голоморфных} структур $(M,\circ)$, $(N,\ast)$ существует отображение $f$ вида:
% ---
$$f\colon M\to N\iff\forall x,y\in M\ f(x\circ y)=f(x)\ast f(y)$$
% ---
Для {\bold изоморфных} структур $(M,\circ)$, $(N,\ast)$ существует изоморфизм $f$ вида:
% ---
$$\begin{gathered}
f\colon M\xrightarrow{\sim} N\iff\forall x,y\in M\ f(x\circ y)=f(x)\ast f(y)\\
(M,\circ)\simeq (N,\ast)
\end{gathered}$$

\subsection{Виды структур}

{\bold Магма} {\ital (группоид)} --- алгебраическая структура $(X,\ast)$ с~внутренней бинарной операцией $\ast$.

{\bold Группа} --- магма, для которой справедливо:
% ---
\begin{list*}[][\#]
\item Ассоциативность $\ast$ --- для {\bold полугруппы}.
\item $\exists$ нейтральный элемент от-но $\ast$ --- для {\bold моноида}.
\item $\forall x\in G\ \exists$ симметричный элемент от-но $\ast$.
{\color{desc}\item Коммутативность $\ast$ --- для {\bold абелевой} группы.}
\end{list*}
% ---
{\bold Кольцо} --- такая алгебраическая структура $(X,+,\cdot)$, что:
% ---
\begin{list*}[][\#]
\item Абелева группа относительно {\ital сложения} $+$.
\item Дистрибутивность {\ital умножения} относительно {\ital сложения}.
{\color{desc}\item Ассоциативность {\ital умножения} $\cdot$ --- для {\bold ассоциативного кольца}.}
{\color{desc}\item Коммутативность {\ital умножения} --- для {\bold коммутативного кольца}.}
{\color{desc}\item $\exists$ нейтральный элемент от-но $\cdot$ --- для {\bold кольца с~единицей}.}
{\color{desc}\item Группа относительно {\ital умножения} --- для {\bold поля}.}
\end{list*}
% ---


{\bold Подгруппа} --- подмножество $H$ аддитивной абелевой группы $G$, которое замкнуто по сложению на всей группе $G$:
% ---
$$+\colon H\times G\to H$$
% ---
{\bold Идеал} --- подмножество $I$ кольца $A$, которое замкнуто по умножению на всём кольце $A$:
% ---
$$\cdot\colon I\times A\to I$$

\subsection{Числовые системы}

{\bold Система натуральных чисел} --- коммутативная аддитивная и мультипликативная полугруппа $\langle\mathbb{N};\ +,\times\rangle$.\par

{\bold Система целых чисел} --- коммутативное кольцо $\langle\mathbb{Z};\ +,\times\rangle$.
\par

{\bold Система рациональных чисел} --- упорядоченное поле $\langle\mathbb{Q};\ +,\times
\rangle$.\par

{\bold Система действительных чисел} --- непрерывное упорядо"=ченное поле $\langle\mathbb
{R};\ +,\times\rangle$.

\subsection{Действительные числа}

{\bold Система действительных чисел} --- алгебраическая структура $\langle\mathbb{R};\ +,\cdot,\leq\rangle$, в которой выполняется следующая {\ital аксиоматика}.
% ---
\begin{theorem}
{\bold Аксиомы сложения.} {\ital Сложение} --- алгебраическая операция $+\colon\mathbb{R}^2\to\mathbb{R}$.

Свойства:
% ---
\begin{list*}[][\#]
\item коммутативность
\item ассоциативность
\item существование нейтрального элемента
\item существование противоположного {\ital (симметричного)} элемента
\end{list*}
% ---
Следствия из аксиом:
% ---
\begin{list*}[][\#]
\item Единственность нуля {\ital (нейтрального элемента)}.
\item Единственность противоположного {\ital (симметричного)} элемента.
\item Умножение на $-1$ порождает противоположный элемент.
\end{list*}
\end{theorem}

\begin{theorem}
{\bold Аксиомы умножения.} {\ital Умножение} --- алгебраическая операция $\cdot\colon\mathbb{R}^2\to\mathbb{R}$.

Свойства:
% ---
\begin{list*}[][\#]
\item коммутативность
\item ассоциативность
\item существование нейтрального элемента
\item существование обратного {\ital (симметричного)} элемента
\end{list*}
% ---
Следствия из аксиом:
% ---
\begin{list*}[][\#]
\item Единственность единицы {\ital (нейтрального элемента)}.
\item Единственность обратного {\ital (симметричного)} элемента.
\item Нуль --- поглощающий элемент.
\end{list*}
\end{theorem}

\begin{theorem}
{\bold Связь сложения и умножения.} {\ital\color{desc} (Дистрибутивность $\cdot$ от-но $+$)}
% ---
$$\forall x,y,z\in\mathbb{R}\ x\cdot(y+z)=x\cdot y+x\cdot z$$
\end{theorem}

\begin{theorem}
{\bold Аксиомы порядка.} {\ital Неравенство} --- алгебраическая операция $\leq\colon\mathbb{R}^2\to\mathbb{R}$.

Свойства:
% ---
\begin{list*}[][\#]
\item рефлексивность
\item антисимметричность
\item транзитивность
\item {\ital линейная упорядоченность:}
% ---
$$\forall x,y\in\mathbb{R}\ (x\leq y)\lor(y\leq x)$$
\end{list*}
\end{theorem}

\begin{theorem}
{\bold Связь сложения и порядка.}
% ---
$$\forall x,y,z\in\mathbb{R}\ x\leq y\iff x+z\leq y+z$$
% ---
{\bold Следствие.}
% ---
$$(x\leq y)\land(z\leq k)\implies x+z\leq y+k$$
\end{theorem}

\begin{theorem}
{\bold Связь умножения и порядка.}
% ---
$$\forall x,y,z\in\mathbb{R}\ (0\leq x)\land(0\leq y)\implies 0\leq x\cdot y$$
% ---
{\bold Следствия.}
% ---
$$\begin{aligned}
0\greater x\land 0\greater y&\implies 0\less xy\\
0\greater x\land 0\less y&\implies 0\greater xy\\
x\less y\land z\greater 0&\implies xz\less yz\\
x\less y\land z\less 0&\implies xz\greater yz
\end{aligned}$$
\end{theorem}

\begin{theorem}
{\bold Аксиома полноты.} Пусть $X,Y\subset\mathbb{R}$ и $X\neq\emptyset$, $Y\neq\emptyset$. 
% ---
$$(\forall x\in X,\ y\in Y\ x\leq y)\implies\exists c\in\mathbb{R}\ x\leq c\leq y$$
\end{theorem}

\begin{theorem}
{\bold Лемма.} {\ital\color{desc}(Сравнение нуля и единицы)}
% ---
$$0\less 1$$
\end{theorem}

Доказательство леммы следует из {\ital аксиом порядка}.

\subsection{Подмножества $\mathbb{R}$}

Для {\bold индуктивного} множества $X\subset\mathbb{R}$ справедливо:
% ---
$$\forall x\in X\ (x+1)\in X$$
% ---
Примеры индуктивных множеств: $\mathbb{N}$, $\mathbb{Z}$, $\mathbb{Q}$, $\mathbb{I}$, $\mathbb{R}$.

\begin{theorem}
{\bold Принцип математической индукции.} Пусть $X\subset\mathbb{N}$:
% ---
$$(\underbrace{1\in X}_{1}\land\underbrace{\forall x\in X\ (x+1)\in X}_{2})\implies X=\mathbb{N}$$
% ---
\begin{list*}[][\#]
\item Базис индукции.
\item Индукционный шаг.
\end{list*}
\end{theorem}
% ---
{\bold Целые числа} $\mathbb{Z}$ --- объединение $\mathbb{N}$ с множеством чисел, противоположных к $\mathbb{N}$, и нулём.

{\bold Рациональные числа} $\mathbb{Q}$ --- множество чисел вида:
% ---
$$m\cdot n^{-1},\quad m,n\in\mathbb{Q}$$
% ---
{\bold Иррациональные числа} $\mathbb{I}$ --- множество вида:
% ---
$$\mathbb{I}=\mathbb{R}\backslash\mathbb{Q}$$

\subsection{Расширенная числовая прямая}

{\bold Расширенная числовая прямая} --- расширение множества действительных 
чисел:
% ---
$$\bar{\mathbb{R}}=\mathbb{R}\cup\{\pm\infty\}$$
% ---
Свойства бесконечности:
% ---
$$\begin{gathered}
(\pm\infty)+(\pm\infty)=\pm\infty\\
(\pm\infty)\cdot(\pm\infty)=+\infty\\
(+\infty)\cdot(-\infty)=-\infty\\
x+(\pm\infty)=(\pm\infty)+x=\pm\infty,\quad x\in\mathbb{R}\\
x\cdot(\pm\infty)=(\pm\infty)\cdot x=\begin{cases*}
&\pm\infty,\quad x\greater 0\\
&\mp\infty,\quad x\less 0
\end{cases*}\\
\frac{x}{0}=\begin{cases*}
&+\infty,\quad x\greater 0\\
&-\infty,\quad x\less 0
\end{cases*}\\
\frac{x}{\infty}=0,\quad x\in\mathbb{R}
\end{gathered}$$

\newpage
\subsection{Комплексные числа}

{\bold Система комплéксных чисел} --- непрерывное поле:
% ---
$$\langle\mathbb{C}=\mathbb{R}^2;\ +,\cdot\rangle$$
% ---
{\bold Комплексное число} --- элемент $z=(a,b)\in\mathbb{C}$.
% ---
\begin{theorem}
{\bold Свойство.} Множество $\mathbb{C}$ включает в себя множество $\mathbb{R}$:
% ---
$$\forall a\in\mathbb{R}\ (a,0)\leftrightarrow a\implies\mathbb{R}\subset\mathbb{C}$$
\end{theorem}
% ---
{\bold Мнимая единица} --- комплексное число вида:
% ---
$$(0,1)\leftrightarrow i,\quad i^2=-1$$
% ---
{\bold Плоскость комплексных чисел} --- декартова система координат $Oab$ с биекцией вида:
% ---
$$z=(a,b)\leftrightarrow M(a,b)$$

\begin{list*}
\item $Oa$ есть действительная ось {\ital\color{desc}($a=\ren{z}$)}
\item $Ob$ есть мнимая ось {\ital\color{desc}($b=\imn{z}$)}
\end{list*}
% ---
\begin{theorem}
{\bold Свойство.} Комплексные числа {\ital равны}, если соответственно равны их действительные и мнимые части:
% ---
$$(a_1,b_1)=(a_2,b_2)\iff a_1=a_2\land b_1=b_2$$
\end{theorem}
% ---
\begin{theorem}
{\bold Формы записи.} Пусть $z=(a,b)\in\mathbb{C}$.
% ---
\begin{list*}
\item алгебраическая форма: $z=a+bi$
\item тригонометрическая форма: $z=\abs{z}(\cos\varphi+i\sin\varphi)$
\end{list*}
\end{theorem}

\subsection{Операции комплексных чисел}

\begin{theorem}
{\bold Сложение и умножение.} Пусть $z_1=(a,b)$, $z_2=(c,d)$:
% ---
$$\begin{aligned}
&z_1+z_2=(a,b)+(c,d)=(a+c,b+d)\\
&z_1\cdot z_2=(a,b)\cdot(c,d)=(ac-bd,ad+bd)
\end{aligned}$$
% ---
Cвойства операций {\ital индуцируются} из множества дейст"=вительных чисел $\mathbb{R}$. 
\end{theorem}

\begin{theorem}
{\bold Сопряжение} --- операция смены знака $\imn{z}$:
% ---
$$z=(a,b)\to\bar{z}=(a,-b)$$
% ---
Свойства:
% ---
\begin{list*}
\item дистрибутивность относительно {\ital сложения} $+$
\item дистрибутивность относительно {\ital умножения} $\cdot$
\end{list*}
\end{theorem}
% ---
\begin{theorem}
{\bold Модуль} комплексного числа $z$ --- расстояние $\rho(O;M)$:
% ---
$$\abs{z}=r=\abs{OM}=\sqrt{a^2+b^2}=\sqrt{z\bar{z}}$$
% ---
Свойства:
% ---
\begin{list*}
\item дистрибутивность относительно {\ital умножения} $\cdot$ {\ital (деления $\colon$)}
\end{list*}
\end{theorem}
% ---
\begin{theorem}
{\bold Аргумент} комплексного числа --- угол, образованный $\overrightarrow{OM}$ с~действительной осью:
% ---
$$\arg z=\varphi=\arctg\frac{b}{a},\quad\varphi\in(-\pi;\pi]$$
% ---
Свойства:
% ---
\begin{list*}
\item $\arg(z_1\cdot z_2)=\arg z_1+\arg z_2$
\item $\arg(z_1\slash z_2)=\arg z_1-\arg z_2$
\item $\arg z^n=n\arg z,\ n\in\mathbb{Z}$
\end{list*}
\end{theorem}
% ---
\begin{theorem}
{\bold Формула Муавра.} Возведение в степень числа $z\in\mathbb{C}$:
% ---
$$z^n=\abs{z}^n(\cos n\varphi+i\sin n\varphi),\quad n\in\mathbb{Z}$$
% ---
{\bold Следствие.} Корень $n$-й степени из числа $z\in\mathbb{C}$:
% ---
$$\sqrt[n]{z}=\sqrt[n]{\abs{z}}\left(\cos\frac{\varphi+2\pi k}{n}+i\sin\frac{\varphi+2\pi k}{n}\right),\quad k\in\mathbb{Z}$$
\end{theorem}
% ---
\begin{theorem}
{\bold Квадратный корень.} Корень из числа $z\in\mathbb{C}$:
% ---
$$\sqrt{z}=\pm\left(\sqrt{\frac{\abs{z}+a}{2}}+\sgnb{b}i\sqrt{\frac{\abs{z}-a}{2}}\right)$$
\end{theorem}

\subsection{Метрическое пространство}

{\ital Метрическое пространство} --- алгебраическая структура $\langle M;\ d\rangle$,
где $d$ --- метрика.\par

Метрика $d$ множества $M$ --- функция $d\colon M\times M\to R^+_0$, которая
определяет {\ital расстояние} между его двумя элементами.\par

Например, {\ital евклидова метрика} использует теорему Пифагора в $n$-мерном
пространстве:
% ---
$$d(x,y)=\sqrt{\sum^{n}_{k=1}(x_k-y_k)^2}$$
% ---
Для метрического пространства $\langle M;\ d\rangle,\ x,y,z\in M$ выполня"=ются
следующие {\ital аксиомы}:\par

--- $d(x,y)=0\iff x=y$ --- {\ital тождество};\\
--- $d(x,y)=d(y,x)$ --- {\ital симметрия};\\
--- $d(x,y)\leq d(x,z) + d(y,z)$ --- {\ital «неравенство треугольника»}.

\subsection{Длина отрезка}

{\bold Расстояние} между точками $A\langle a\rangle$ и $B\langle b\rangle$ ---
% ---
$$\abs{\overrightarrow{AB}}=\abs{a-b}\implies AB^2=(a-b)(\bar{a}-\bar{b})$$
% ---
{\ital Уравнение окружности} с центром $A\langle a\rangle$ радиуса $r$ ---
% ---
$$(z-a)(\bar{z}-\bar{a})=r^2$$

\subsection{Скалярное произведение векторов}

{\ital Скалярное произведение радиус-векторов} ---
% ---
$$2\overrightarrow{OA}\cdot\overrightarrow{OB}=a\bar{b}+\bar{a}b$$
% ---
{\bold Доказательство.} Пусть $A\langle a\rangle$, $B=\langle b\rangle$, $a=x_1+y_1i$, $b=x_2+y_2i$. Тогда:
% ---
$$a\bar{b}+\bar{a}b=(x_1+y_1i)(x_2-y_2i)+(x_1-y_1i)(x_2+y_2i)=$$
$$2(x_1x_2+y_1y_2)=2\overrightarrow{OA}\cdot\overrightarrow{OB}\qedb$$

Пусть $A\langle a\rangle$, $B\langle b\rangle$, $C\langle c\rangle$, $D\langle d\rangle$ --- четыре различные точки. Тогда {\ital скалярное произведение произвольных векторов} ---
% ---
$$2\overrightarrow{AB}\cdot\overrightarrow{CD}=(b-a)(\bar{d}-\bar{c})+(\bar{b}-\bar{a})(d-c)$$
% ---
{\bold Доказательство.} По условию:
% ---
$$2\overrightarrow{AB}\cdot\overrightarrow{CD}=2(\overrightarrow{OB}-\overrightarrow{OA})(\overrightarrow{OD}-\overrightarrow{OC})=$$
$$2(\overrightarrow{OB}\cdot\overrightarrow{OD}-\overrightarrow{OB}\cdot\overrightarrow{OC}-\overrightarrow{OA}\cdot\overrightarrow{OD}+\overrightarrow{OA}\cdot\overrightarrow{OC})=$$
$$\cancel{2}\cdot\frac{1}{\cancel{2}}(b\bar{d}+\bar{b}d-b\bar{c}-\bar{b}c-a\bar{d}-\bar{a}d+a\bar{c}+\bar{a}c)=$$
$$(a-b)(\bar{c}-\bar{d})+(\bar{a}-\bar{b})(c-d)\qedb$$

\subsection{Коллинеарность}

{\bold Коллинеарными} называются:
% ---
\begin{tabularcx}{3pt}{3pt}{@{--- } L}{\textwidth}
{\ital точки}, которые лежат на одной прямой;\\
{\ital векторы}, которые лежат на одной прямой или на~параллельных прямых.
\end{tabularcx}
% ---
\begin{theorem}
{\ital Критерий коллинеарности} точек $A$, $B$ с $O$:
% ---
$$\frac{a}{b}=\overline{\left(\frac{a}{b}\right)}\quad\text{или}\quad
\begin{sqcases*}
&a=0\\
&b=0
\end{sqcases*}$$
\end{theorem}
% ---
{\bold Доказательство.} Очевидно, что:
% ---
$$\left[\begin{aligned}
&\arg a-\arg b=0\\
&\arg a-\arg b=\pm\pi
\end{aligned}\right.\implies
\arg\frac{a}{b}=0;\pm\pi$$
% ---
По определению аргумента комплексного числа:
% ---
$$\frac{a}{b}\text{ --- действительное число}\implies\frac{a}{b}=\overline{\left(\frac{a}{b}\right)}\qedb$$
% ---
\begin{theorem}
{\ital Критерий коллинеарности} векторов $\overrightarrow{AB}$, $\overrightarrow{CD}$:
% ---
$$\frac{b-a}{d-c}=\overline{\left(\frac{b-a}{d-c}\right)}\quad\text{или}\quad
\left[\begin{aligned}
\overrightarrow{AB}=\overrightarrow{0}\\
\overrightarrow{CD}=\overrightarrow{0}\\
\end{aligned}\right.$$
\end{theorem}
% ---
{\bold Доказательство.} По определению комплексных чисел:
% ---
$$\overrightarrow{AB}\sim b-a,\ \overrightarrow{CD}\sim d-c$$
% ---
По критерию коллинеарности двух точек с $O$:
% ---
$$\frac{b-a}{d-c}=\overline{\left(\frac{b-a}{d-c}\right)}\qedb$$
% ---
Если $A$, $B$, $C$, $D$ лежат на одной окружности, то:
% ---
$$\overrightarrow{AB}\parallel\overrightarrow{CD}\iff\frac{b}{d}=\frac{a}{c}$$
% ---
\begin{theorem}
{\ital Критерий коллинеарности} трёх точек:
% ---
$$\frac{b-a}{c-a}=\overline{\left(\frac{b-a}{c-a}\right)}\quad\text{или}\quad
\left[\begin{aligned}
&\overrightarrow{AB}=\overrightarrow{0}\\
&\overrightarrow{AC}=\overrightarrow{0}
\end{aligned}\right.$$
\end{theorem}
% ---
{\bold Доказательство.} Очевидно, что:
% ---
$$\overrightarrow{AB}\parallel\overrightarrow{AC}\iff A,B,C\text{ коллинеарны}$$
% ---
По критерию коллинеарности векторов:
% ---
$$\frac{b-a}{c-a}=\overline{\left(\frac{b-a}{c-a}\right)}\qedb$$
% ---
\begin{theorem}
{\ital Уравнение секущей} $AB$:
% ---
$$(\bar{a}-\bar{b})z+(b-a)\bar{z}+a\bar{b}-b\bar{a}=0$$
\end{theorem}
% ---
{\bold Доказательство.} Нет и не будет: раздел будет снесён.
