\section{Модульная арифметика}

\subsection{Конгруэнтность}

Два целых числа {\bold конгруэнтны} {\ital\color{desc} (сравнимы)} по модулю $m$, когда их разность кратна $m$ {\ital\color[HTML]{888888} (К.Ф. Гаусс)}:
% ---
$$a\equiv b\modb{m}\iff m\mid(a-b)\iff a=b+mk,\ k\in\mathbb{Z}$$
% ---
Отношение конгруэнтности {\ital транзитивно}, поэтому числа образуют {\ital систему остаточных классов} $\symbf{Z}_m$ по модулю {\ital m}. Например, $\symbf{Z}_3$:
% ---
$$\{\ldots, -6, -3, \symbf{0}, 3, 6,\ldots\}\text{ класс }r_0$$
$$\{\ldots, -5, -2, \symbf{1}, 4, 7,\ldots\}\text{ класс }r_1$$
$$\{\ldots, -4, -1, \symbf{2}, 5, 8,\ldots\}\text{ класс }r_2$$

\subsection{Свойства сравнения}

Конгруэнтные числа можно {\ital складывать}, {\ital перемножать} и~передавать {\ital многочлену} $f\in\mathbb{Z}[x]$:
% ---
$$\begin{cases}
\begin{aligned}
a\equiv b&\modb{m}\\
c\equiv d&\modb{m}\\
\end{aligned}
\end{cases}
\hspace*{-12pt}\implies
\begin{cases}
\begin{aligned}
&a + c\equiv b + d &\modb{m}\\
&ac\equiv bd &\modb{m}\\
&f(a)\equiv f(b) &\modb{m}
\end{aligned}
\end{cases}$$
% ---
Конгруэнтные числа можно {\ital умножать {\color{desc}(делить)}} на одно число с {\ital увеличением {\color{desc}(сокращением)}} модуля:
% ---
\begin{align*}
a\equiv b\modb{m}&\iff ad\equiv bd\modb{md}\\
ad\equiv bd\modb{m}&\iff a\equiv b\modb{\frac{m}{(m,d)}}
\end{align*}
% ---
Из транзитивности делимости следует:
% ---
$$a\equiv b\modb{m},\ n\mid m\implies a\equiv b\modb{n}$$

\subsection{Признаки делимости}

Для вывода признаков делимости лучше использовать {\ital деся"=тичное представление} числа $\overline{a_1a_2\dots a_n}$:
% ---
$$\overline{a_1a_2\dots a_n}=\sum^{n-1}_{i=0}a_{n-i}10^i$$

\begin{itemize}
\item[---]При модуле $m=2^k; 5^k; 10^k$ одночлены $a_{n-i}10^i\equiv a_{n-i}0=$ $=0$
$(i\geq k)$. Значит, число $\overline{a_1a_2\dots a_n}$ кратно {\ital m}, когда
последние {\ital k} цифры кратны {\ital m}:
$$\overline{a_1a_2\dots a_n}\equiv 0\iff
\overline{a_{n-k+1}\dots a_{n-1}a_n}\equiv 0$$

\item[---]При модуле $m=3; 9$ одночлены $a_{n-i}10^i\equiv a_{n-i}1^i=a_{n-i}$.
Значит, число $\overline{a_1a_2\dots a_n}$ кратно {\ital m}, когда сумма цифр\\
кратна {\ital m}:
$$\overline{a_1a_2\dots a_n}\equiv 0\iff
a_1+a_2+\dots+a_n\equiv 0$$

\item[---]При модуле $m=11$ одночлены $a_{n-i}10^i\equiv a_{n-i}(-1)^i$.
Значит, число $\overline{a_1a_2\dots a_n}$ кратно 11, когда знакочереду-\\ющаяся
сумма цифр кратна 11:
$$\overline{a_1a_2\dots a_n}\equiv 0\iff
a_1-a_2+\dots-a_n\equiv 0$$

\item[---]При модуле $m=7$ вычтем из числа {\ital n} последнюю цифру; останется
$\lfloor n/10\rfloor$. Последняя цифра равна $n-10\lfloor n/10\rfloor$.
Вычтем из числа удвоенную последнюю цифру:
$$\left\lfloor\frac{n}{10}\right\rfloor -2(n-10\left\lfloor\frac{n}{10}
\right\rfloor)\equiv 0\iff 21\left\lfloor\frac{n}{10}\right\rfloor-2n\equiv 0$$
Одночлен $21\lfloor n/10\rfloor\equiv 0$. Значит, число $\overline{a_1a_2\dots a_n}$
кратно 7, когда удвоенная разность последней цифры числа\\
и самого числа без этой цифры кратна 7:
$$\overline{a_1a_2\dots a_n}\equiv 0\iff\overline{a_1a_2\dots a_{n-1}}-2a_n\equiv 0$$
\end{itemize}

\subsection{Функция Эйлера}

Функция $\phi(m)$ считает количество положительных целых чисел, меньших {\ital m}
и взаимно простых с ним {\ital\color[HTML]{888888} (для малых\\ и простых m
целесообразно перебрать вручную)}:
% ---
$$\phi(m)=m\prod^{}_{p\mid m}\left(1-\frac{1}{p}\right)$$

\begin{tabularc}{0pt}{0pt}{r @{ --- } l}{n}
$p$ & простой делитель $m$;\\
$1/p$ & часть чисел, кратных $p$;\\
$1-1/p$ & часть чисел, взаимно простых с $p$.
\end{tabularc}

Функция Эйлера {\ital мультипликативна {\color{desc}(только для взаимно простых натуральных чисел)}}.

\subsection{Теорема Эйлера}

\begin{theorem}
{\bold Теорема.} Пусть $a\in\mathbb{Z}$, $(a,m)=1$. Тогда: {\ital\color{desc}(Л. Эйлер)}
% ---
$$a^{\phi(m)}\equiv 1\modb{m},\quad a\not\equiv 0\modb{m}$$
\end{theorem}
% ---
{\bold Доказательство.} Введём систему остаточных классов $\symbf{Z}_m$. В ней есть {\ital m} классов: $r_0,r_1,\ldots,r_{m-1}$.

Пусть множество $\Phi$ содержит в себе $\phi(m)$ остатков, взаимно простых с {\ital m}. Домножим каждый элемент на {\ital a} и образуем новое множество $\Phi_a$. Заметим, что:

\begin{column*}{t}{.55\linewidth}
{\ital Элементы $\Phi_a$ из разных классов.}\par
Допустим, это не так. Тогда:
$$ar_k\equiv ar_l\implies r_k\equiv r_l$$
Но $r_k\nequiv r_l\implies ar_k\nequiv ar_l\qedw$
\end{column*}
% ---
\begin{column*}{t}{.43\linewidth}
{\ital $\Phi$ и $\Phi_a$ конгруэнтны.}\par
Пусть $ar_k\equiv r_l,\ r_l\in\symbf{Z}_m$.\par
Так как $m\nmid ar_k$, то:
$$r_l\in\Phi\implies\Phi\equiv\Phi_a\qedw$$
\end{column*}

Перемножим элементы множеств $\Phi$ и $\Phi_a$:
% ---
\begin{align*}
r_0r_1\ldots r_{\phi(m)}&\equiv ar_0ar_1\ldots ar_{\phi(m)}\implies\\
r_0r_1\ldots r_{\phi(m)}&\equiv a^{\phi(m)}r_0r_1\ldots r_{\phi(m)}\implies
a^{\phi(m)}\equiv 1\qedb
\end{align*}
% ---
\begin{theorem}
{\bold Следствие.} Пусть $a\in\mathbb{Z},\ b\in\mathbb{N},\ (m,a)=1$. Тогда:
% ---
$$a^b\equiv a^{b\modn{\phi(m)}}\modb{m},\quad a\not\equiv 0\modb{m}$$
\end{theorem}
% ---
{\bold Доказательство.} Представим {\ital b} в арифметическом виде:
% ---
$$b=\phi(m)\left\lfloor\frac{b}{\phi(m)}\right\rfloor+b\modn{\phi(m)}$$

\begin{itemize}[itemindent=18mm]
\item[$\phi(m)$] --- модуль деления.\\
\item[$\lfloor b/\phi(m)\rfloor$] --- целое частное.\\
\item[$b\modn{\phi(m)}$] --- остаток.
\end{itemize}

Подставим полученное выражение:
% ---
$$a^{\phi(m)\lfloor b/\phi(m)\rfloor+b\modn{\phi(m)}}=(a^{\phi(m)})^{\lfloor b/\phi(m)\rfloor}a^{b\modn{\phi(m)}}$$
% ---
Так как $a^{\phi(m)}\equiv 1$, получается $a^b\equiv a^{b\modn{\phi(m)}}.\qedb$

\subsection{Алгоритм Евклида}

Пусть $a,b\in\mathbb{N}^0$ $(a\greater b)$, тогда:
% ---
$$(a,b)=(a\modn{b},b)$$
% ---
{\bold Доказательство.} Пусть $m\mid a-b,b$:
% ---
$$+\begin{cases*}
&a-b\equiv 0 &\modb{m}\\
&b\equiv 0 &\modb{m}
\end{cases*}\implies
\begin{cases*}
&a\equiv 0 &\modb{m}\\
&b\equiv 0 &\modb{m}
\end{cases*}$$
% ---
Значит, любой общий делитель $a-b,b$ имеется у $a,b$.

По определению НОД:
% ---
$$(a,b)=(a-b,b)$$
% ---
По определению конкруэнтных чисел:
% ---
$$(a,b)=(a\modn{b},b)$$

\subsection{Мультипликативная инверсия}

Пусть $ab\equiv 1\modb{m}$ --- линейное сравнение, где $b$ --- {\bold мультипликативная инверсия} числа $a$ по модулю $m$:
% ---
$$b\equiv a^{-1}\equiv\frac{1}{a}\modb{m},\quad(a,m)=1$$
% ---
«Дробные» числа можно {\ital складывать}, {\ital перемножать} и {\ital сокра"=щать} как рациональные:
% ---
$$\begin{cases}\begin{aligned}
\frac{a}{b}+\frac{c}{d}&\equiv\frac{ad+bc}{cd} &\modb{m}\\
\frac{a}{b}\times\frac{c}{d}&\equiv\frac{ac}{bd} &\modb{m}\\
\frac{eg}{fg}&\equiv\frac{e}{f} &\modb{m}
\end{aligned}\end{cases}$$

\subsection{Линейное сравнение}

{\ital Линейное} сравнение вида $ax\equiv b\modb{m}$ разрешимо отно"=сительно $x$, когда $(m,a)\mid b$. {\ital\color[HTML]{888888} (по соотношению Безу)}

План решения:
% ---
\begin{list*}
\item упростить линейное сравнение
\item рассчитать $(m,a)$ по алгоритму Евклида
\item выразить $(m,a)$ через полученные остатки
\item домножить соотношение Безу на $b$
\end{list*}
% ---
{\bold Пример.} Решить линейное сравнение: $4x\equiv 4\modb{6}$.\par
Упростим сравнение:
\begin{align*}
4x&\equiv 4\modb{6}\ |\cdot 1/2\\
2x&\equiv 2\modb{3}
\end{align*}
Применим {\ital алгоритм Евклида} в алгебраическом виде:\par
\begin{column*}{t}{.47\linewidth}
{\ital «Прямой» алгоритм:}
\begin{align*}
3&=2\cdot 1+1\\
2&=\symbf{1}\cdot 2+0
\end{align*}
\end{column*}
% ---
\begin{column*}{t}{.4\linewidth}
{\ital «Обратный» алгоритм:}
\begin{align*}
1&=3\cdot \symbf{1}+2\cdot (\symbf{-1})\ |\cdot 2\\
2&=3\cdot \symbf{2}+2\cdot (\symbf{-2})
\end{align*}
\end{column*}\par
Итак, коэффициенты Безу найдены: $x=-2,\ y=2$.\par
{\ital Ответ:} $x=-2$.\par

\subsection{Китайская теорема об остатках}

Сравнения можно объединять в {\ital систему}:
% ---
$$\begin{cases}
\begin{aligned}
x&\equiv a_1\modb{m_1}\\
&\dots\\
x&\equiv a_n\modb{m_n}
\end{aligned}
\end{cases}$$
% ---
Она разрешима относительно {\ital x} по модулю $[m_1,\ldots,m_n]$, когда разрешима каждая пара сравнений, в частности $(m_1,m_2)\mid a_1-a_2$.

{\bold Доказательство.} Рассмотрим пару сравнений из системы:
% ---
$$\begin{cases}
x\equiv a_1\modb{m_1}\\
x\equiv a_2\modb{m_2}
\end{cases}
\hspace*{-12pt}\iff
\begin{cases}
\begin{aligned}
x&=a_1+m_1j,& j&\in\mathbb{Z}\\
x&=a_2-m_2k,& k&\in\mathbb{Z}
\end{aligned}
\end{cases}
\hspace*{-12pt}\iff$$
$$m_1j+m_2k=a_2-a_1$$
% ---
Данное соотношение Безу имеет целые коэффициенты $j$, $k$, когда $(m_1,m_2)\mid(a_1-a_2)$.$\qedw$

По индукции, система будет разрешима относительно {\ital x}, когда будет разрешима каждая пара сравнений.

Допустим, $x\equiv y\equiv a_i\modb{m_i},\ i\in\{i\}_{i=1}^n$ --- решение всей системы. Значит, $m_i\mid x-y\implies [m_1,\ldots,m_n]\mid x-y\iff$\\$x\equiv y\modb{[m_1,\ldots,m_n]}.\qedb$

План решения каждой пары сравнений:
% ---
\begin{itemize}
\item[---]упростить линейные сравнения;
\item[---]преобразовать их в соотношения Безу, приравнять их;
\item[---]решить полученное выражение как линейное сравнение.
\end{itemize}
% ---
{\bold Пример.} Решить систему сравнений:
% ---
$$\begin{cases}
\begin{aligned}
&x\equiv 2 &\modb{3}\\
&x\equiv 2 &\modb{4}\\
&2x\equiv -3 &\modb{5}
\end{aligned}
\end{cases}$$
% ---
Упростим последнее сравнение:
% ---
$$2x\equiv -3\modb{5}\iff x\equiv 1\modb{5}$$
% ---
Преобразуем первую пару сравнений в соотношения Безу:
% ---
$$\begin{cases}
\begin{aligned}
x\equiv 2 &\modb{3}\\
x\equiv 2 &\modb{4}
\end{aligned}
\end{cases}
\hspace*{-12pt}\iff
\begin{cases}
\begin{aligned}
x&=2+3j,& j&\in\mathbb{Z}\\
x&=2+4k,& k&\in\mathbb{Z}
\end{aligned}
\end{cases}$$
% ---
Приравняем их и решим как сравнение:
% ---
$$2+3j=2+4k\iff 2\equiv 2+k\modb{3}\iff k\equiv 0\modb{3}$$
% ---
Значит, $x=2+4k\equiv 2\modb{12}$ --- решение первой пары.

Аналогично решив следующую {\ital\color{desc} (и последнюю)} пару, получим решение всей системы: $x\equiv 26\modb{60}$.

{\ital Ответ:} $x\equiv 26\modb{60}$.

\subsection{Сравнение по составному модулю}

Пусть $f\in\mathbb{Z}[x]$. Тогда для $m=p_1^{\alpha_1}\dots p_r^{\alpha_r}$ разрешимо
% ---
$$f(x)\equiv 0\modb{m},$$
% ---
если разрешимы $f(x)\equiv 0\modb{p_i^{\alpha_i}},\ i\in[1;r]\cap\mathbb{Z}$.

{\bold Доказательство $\implies$.} Пусть $x\in\mathbb{Z}$ --- решение
% ---
$$f(x)\equiv 0\modb{m},\ p_i^{\alpha_i}\mid m\implies f(x)\equiv 0\modb{p_i^{\alpha_i}}.\qedb$$
% ---
{\bold Доказательство $\impliedby$.} Пусть $x_i$ --- решение
% ---
$$f(x_i)\equiv 0\modb{p_i^{\alpha_i}}$$
% ---
По китайской теореме об остатках:
% ---
$$\forall i_1,i_2\in[1;r],\ i_1\neq i_2\ (p_{i1}^{\alpha_{i1}},p_{i2}^{\alpha_{i2}})=1\implies$$
$$\exists x\colon x\equiv x_i\modb{p_i^{\alpha_i}}\implies f(x)\equiv 0\modb{[p_1^{\alpha_1},\dots,p_r^{\alpha_r}]}\implies$$
$$f(x)\equiv 0\modb{m}\qedb$$ 

\subsection{Сравнение по степени простого модуля}

Пусть $f\in\mathbb{Z}[x]$. Тогда для простого $p$ разрешимо
% ---
$$f(x)\equiv 0\modb{p^\alpha},$$
% ---
если разрешимы $f(x)\equiv 0\modb{p^i},\ i\in[1;\alpha]\cap\mathbb{Z}$.

{\bold Доказательство.} Аналогично прошлому пункту.

\subsection{Лемма Гензеля}

Пусть для $f\in\mathbb{Z}[x]$ верно {\ital\color{desc}(К. Гензель)}:
% ---
$$f(a)\equiv 0\modb{p^\alpha},\quad f'(a)\not\equiv 0\modb{p}$$
% ---
Тогда существует такое уникальное $t$, что:
% ---
$$f(a+tp^\alpha)\equiv 0\modb{p^{\alpha+1}}$$
% ---
{\bold Доказательство.} Пусть $a$ --- решение $f(x)\equiv 0\modb{p^\alpha}$, которое можно представить в виде $x=a+tp^\alpha$.

По теореме Тейлора:
% ---
$$f(a+tp^\alpha)=f(a)+tp^\alpha f'(a)+t^2p^{2\alpha}f''(a)/2!+\dots$$
$$+t^np^{n\alpha}f^{(n)}(a)/n!\equiv f(a)+tp^\alpha f'(a)\modb{p^{\alpha+1}}\qedb$$
% ---
{\bold Следствие.} Пусть для $f\in\mathbb{Z}[x]$ верно
% ---
$$f(x_\alpha)\equiv 0\modb{p^\alpha},\quad f'(x_\alpha)\not\equiv 0\modb{p^\alpha}.$$
% ---
Тогда решение сравнения по модулю $p^{\alpha+1}$ имеет вид:
% ---
$$x_{\alpha+1}\equiv x_\alpha-\frac{f(x_\alpha)}{f'(x_\alpha)}\modb{p^{\alpha+1}}$$
% ---
{\bold Доказательство.} По лемме Гензеля:
% ---
\begin{align*}
f(x_\alpha)+tp^\alpha f'(x_\alpha)\equiv 0&\modb{p^{\alpha+1}}\iff\\
tp^\alpha\equiv-\frac{f(x_\alpha)}{f'(x_\alpha)}&\modb{p^{\alpha+1}}\iff\\
x_\alpha+tp^\alpha\equiv x_{\alpha+1}\equiv x_\alpha-\frac{f(x_\alpha)}{f'(x_\alpha)}&\modb{p^{\alpha+1}}\qedb
\end{align*}
