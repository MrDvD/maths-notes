\section{Предел последовательности}

% ОПРЕДЕЛЕНИЕ ПОДПОСЛЕДОВАТЕЛЬНОСТИ

\subsection{Определение}

{\bold Предел} последовательности $\{x_n\}$ --- такое $a\in\bar{\mathbb{R}}$, что:
% ---
$$\forall\varepsilon\greater 0\ \exists N\colon\forall n\greater N\ x_n\in U_\varepsilon(a)$$
% ---
Обозначение:
% ---
$$\lim_{n\to\infty}x_n=a\text{ или }n\to\infty,\ x_n\to a$$
% ---
\begin{theorem}
{\bold Свойства.}
% ---
\begin{list*}
\item дистрибутивность относительно {\ital сложения} $+$
\item дистрибутивность относительно {\ital умножения} {\ital (деления)}
\end{list*}
\end{theorem}

{\ital По существованию предела} последовательность бывает:
% ---
\begin{list*}
\item {\bold сходящаяся}: существует {\ital конечный} предел
\item {\bold расходящаяся}: все остальные
\end{list*}

\begin{theorem}
Если у последовательности есть предел, то он {\ital единственный}.
\end{theorem}

{\bold Доказательство.} Пусть $\{x_n\}$ --- такая последова-ть, что:
% ---
$$\begin{cases*}
&\lim_{n\to\infty}x_n=a\\
&\lim_{n\to\infty}x_n=b
\end{cases*}\implies x\in U_\varepsilon(a)\cap U_\varepsilon(b)$$
% ---
Положим, что $a\neq b$:
% ---
$$\exists\varepsilon\greater 0\colon U_\varepsilon(a)\cap U_\varepsilon(b)=\emptyset$$
% ---
Значит, $x_n\in\emptyset$, что противоречит условию.$\qedb$

\subsection{Свойства}

\begin{theorem}
{\bold Критерий Коши.} Последовательность сходится $\implies$ она ограничена.
\end{theorem}
% ---
{\bold Доказательство.} По определению предела:
% ---
$$\sqsupset\lim_{n\to\infty}x_n=a\iff\forall\varepsilon\greater 0\ \exists N\colon\forall n\greater N\ x_n\in U_\varepsilon(a)$$
% ---
По «дистрибуции» модуля относительно сложения:
% ---
$$\abs{x_n}=\abs{x_n-a+a}\leq\abs{x_n-a}+\abs{a}\less\varepsilon+\abs{a}$$
% ---
Положим, что $\forall m\leq N\ L=\max(\abs{\{x_m\}},\varepsilon+\abs{a})\implies\abs{x_n}
\leq L$.$\qedb$

\begin{theorem}
{\bold Предельный переход.} Пусть $n\to\infty$, $x_n\to a$, $y_n\to b$. Тогда справедливо следствие:
% ---
$$x_n\leq y_n\text{ или }x_n\less y_n\implies a\leq b$$
\end{theorem}

{\bold Доказательство.} По определению предела:
% ---
$$\forall\varepsilon\greater 0\ \exists N\colon\forall n\greater N\ x_n\in U_
\varepsilon(a),\ y_n\in U_\varepsilon(b)$$
% ---
Следовательно,
% ---
$$+\begin{cases}
x_n\leq y_n\\
a-x_n\less\varepsilon\\
y_n-b\less\varepsilon
\end{cases}\hspace*{-12pt}\iff
\begin{cases}
y_n-x_n\geq 0\\
y_n-x_n\less 2\varepsilon+b-a
\end{cases}\hspace*{-12pt}\iff
\frac{a-b}{2}\less\varepsilon$$
% ---
Так как $\varepsilon$ --- сколь угодно малое положительное число, то $a-b\leq 0\iff a\leq 
b$.$\qedw$

При $x_n\less y_n$ доказательство аналогично.$\qedb$

\begin{theorem}
{\bold Теорема о зажатой послед-ти.} Пусть $n\to\infty$,\\ $x_n,y_n\to a$. Тогда справедливо следствие:
% ---
$$\forall\{z_n\}\colon x_n\leq z_n\leq y_n\implies z_n\to a$$
\end{theorem}

{\bold Доказательство.} По определению предела:
% ---
$$\forall\varepsilon\greater 0\ \exists N\colon\forall n\greater N\ x_n,y_n\in U_\varepsilon(a)$$
% ---
Следовательно:
% ---
$$a-\varepsilon\less x_n\leq z_n\leq y_n\less a+\varepsilon\implies z_n\in U_
\varepsilon(a)\implies\lim_{n\to\infty}z_n=a\qedb$$

\subsection{Фундаментальность}

{\bold Фундаментальная последова-ть} удовлетворяет {\ital условию Коши}:
% ---
$$\forall\varepsilon\greater 0\ \exists N\colon\forall n,m\greater N\ \abs{x_n-x_m}\less
\varepsilon$$
% ---
Например, бесконечная десятичная последовательность вещественного числа {\ital фундаментальна}.

Для {\bold эквивалентных последова-тей} верно:
% ---
$$\{x_n\}\sim\{y_n\}\iff\{x_n\}-\{y_n\}\text{ --- б.м.}$$
% ---
Отношение $\sim$ обладает всеми свойствами {\ital эквиваленции}.
% ---
\begin{theorem}
Фундаментальная последовательность {\ital ограничена}.
\end{theorem}
% ---
{\bold Доказательство.} По условию Коши:
% ---
$$\forall\varepsilon\greater 0\ \exists N\colon\forall n,m\greater N\ \abs{x_n-x_m}\less
\varepsilon$$
% ---
По «дистрибуции» модуля относительно сложения:
% ---
$$\begin{cases}
\abs{x_n-x_m}\less\varepsilon\\
\abs{x_n}=\abs{x_n-x_m+x_m}
\end{cases}\hspace*{-12pt}\iff
\begin{cases}
\abs{x_n-x_m}\less\varepsilon\\
\abs{x_n}\leq\abs{x_n-x_m}+\abs{x_m}
\end{cases}\hspace*{-12pt}\iff$$
$$\abs{x_n}\less\varepsilon+\abs{x_m}$$
% ---
Положим, что $\forall k\leq N\ L=\max(\abs{\{x_k\}},\varepsilon+\abs{x_m})\implies
\abs{x_n}\leq L$.$\qedb$

\begin{theorem}
Последовательность сходится $\iff$ она фундаментальная.
\end{theorem}
% ---
{\bold Доказательство $\implies$.} По определению предела:
% ---
$$\forall\varepsilon\greater 0\ \exists N\colon\forall n\greater N\ x_n\in U_
{\varepsilon/2}(a)$$ 
% ---
$$\forall n,m\greater N\ \abs{x_n-x_m}=\abs{(x_n-a)+(a-x_m)}$$
% ---
По «дистрибуции» модуля относительно сложения:
% ---
$$\abs{x_n-x_m}\leq\abs{x_n-a}+\abs{x_m-a}\less \varepsilon/2+\varepsilon/2=\varepsilon
\qedb$$
% ---
{\bold Доказательство $\impliedby$.} Пусть $\{x_n\}$ фундаментальна $\implies$ она 
ограничена $\implies$ по лемме Больцано-Вейерштрасса она частично сходится к $c$
$\implies$ по условию Коши и по лемме Больцано-Вейерштрасса она сходится к $c$.$\qedb$

\subsection{Теорема Вейерштрасса}

\begin{theorem}
{\bold Теорема.} {\ital\color{desc} (О монотонной последовательности)}

\begin{list*}[][\#]
\item Монотонная ограниченная послед-ть {\ital сходится}.
\item Монотонная неограниченная послед-ть {\ital бесконечно большая}.
\end{list*}
% ---
$$\begin{gathered}
\forall\{x_n\}\kern-4pt\nearrow\lim_{n\to\infty}x_n=\sup\{x_n\}\\
\forall\{y_n\}\kern-4pt\searrow\lim_{n\to\infty}y_n=\inf\{y_n\}
\end{gathered}$$
\end{theorem}
% ---
{\bold Доказательство.} По определению точной верхней границы:
% ---
$$\forall n\in\mathbb{N}\ x_n\leq\sup\{x_n\}$$
% ---
Так как последовательность неубывает, то
% ---
$$\forall\epsilon\greater 0\ \exists N\colon\forall n\greater N\ x_n\in U_\epsilon(\sup
\{x_n\})\implies$$
$$\lim_{n\to\infty}x_n=\sup\{x_n\}.\qedw$$
% ---
Для $\{y_n\}\kern-4pt\searrow$ доказательство аналогично.$\qedb$

\subsection{Принцип Кантора}

\begin{theorem}
{\bold Теорема.} Последовательность вложенных отрезков содержит точки $\xi$, которые принадлежат им всем:
% ---
$$\forall n\in\mathbb{N}\ \exists\xi\in[a_n;b_n]\subset[a_{n-1};b_{n-1}]$$
% ---
Если $n\to\infty$, $(b_n-a_n)\to 0$, то $\xi$ единственна:
% ---
$$\lim_{n\to\infty}a_n=\sup\{a_n\}=\lim_{n\to\infty}b_n=\inf\{b_n\}=\xi$$
\end{theorem}
% ---
{\bold Доказательство.} По теореме Вейерштрасса:
% ---
$$\lim_{n\to\infty}a_n=\sup\{a_n\}\quad\quad\lim_{n\to\infty}b_n=\inf\{b_n\}$$
% ---
$$\forall n\in\mathbb{N}\colon\xi\in[\sup\{a_n\};\inf\{b_n\}]\implies\xi\in[a_n;b_n]\qedw$$
% ---
Если $\inf\{b_n\}=\sup\{a_n\}$, то $\xi$ единственна:
% ---
$$0=\inf\{b_n\}-\sup\{a_n\}=\lim_{n\to\infty}b_n-\lim_{n\to\infty}a_n=\lim_{n\to\infty}(b_n-a_n)\qedb$$

\subsection{Лемма Больцано-Вейерштрасса}

{\bold Частичный предел} последовательности --- предел её подпоследовательности.

\begin{theorem}
{\bold Лемма.} {\ital\color{desc}(О предельной точке)}
% ---
\begin{list*}[][\#]
\item У ограниченной последовательности есть частичный {\ital конечный} предел.
\item У неограниченной последовательности есть частичный {\ital бесконечный} предел.
\end{list*}
% ---
$$\forall\{x_n\}\in[a;b]\ \exists\{n_k\}\kern-4pt\uparrow\colon\lim_{k\to\infty}x_{n_k}=\xi$$
\end{theorem}
% ---
{\bold Доказательство.} По принципу Кантора:
% ---
$$\forall k\in\mathbb{N}\ \exists!\xi\in[a_k;b_k]\subset[a_{k-1};b_{k-1}]\iff$$
$$\lim_{k\to\infty}a_k=\lim_{k\to\infty}b_n=\xi$$
% ---
Образуем подпоследовательность: {\ital\color{desc} (метод Больцано)}
% ---
$$\{x_{n_k}\mid \{n_k\}\kern-4pt\uparrow,\ x_{n_k}\in[a_k;b_k]\}$$
% ---
По теореме о зажатой последовательности:
% ---
$$a_k\leq x_{n_k}\leq b_k\implies \lim_{k\to\infty}x_{n_k}=\xi\qedb$$
% ---
\begin{theorem}
{\bold Теорема.} Частичный предел фундаментальной\\ последовательности равен её пределу:
% ---
$$\lim_{k\to\infty}x_{n_k}=\lim_{n\to\infty}x_n=a$$
\end{theorem}
% ---
{\bold Доказательство.} Пусть $\{x_n\}$ фундаментальна $\implies$ она ограничена.

По лемме Больцано-Вейерштрасса $\lim_{k\to\infty}x_{n_k}=a$.

По условию Коши:
% ---
$$\forall\varepsilon/2\greater 0\ \exists N\colon\forall n,m\greater N\ \abs{x_n-x_m}
\less\varepsilon/2$$
% ---
Зафиксируем $n$. При $x_m=x_{n_k}\greater N$ перейдём к пределу:
% ---
$$\abs{x_n-a}\leq\varepsilon/2\less\varepsilon\iff\lim_{k\to\infty}x_n=a\qedb$$
