\section{Теория функций}

\subsection{Монотонность}

\begin{theorem}
{\bold Встречная монотонность.} Да.
\end{theorem}
% ---
\begin{theorem}
{\bold Теорема.} Выражение вида $f(f(x))$...
\end{theorem}

\begin{theorem}
{\bold Метод рационализации.} Пусть $f$ --- монотонно возрастающая функция. Тогда справедливо:
% ---
$$f(a)-f(b)\lor 0\implies a-b\lor 0$$
% ---
Его можно применять к отдельному {\ital множителю} или в~составе {\ital дроби}.
\end{theorem}
