\section{Тригонометрия}

\subsection{Основные функции}

{\bold Единичной} называется окружность, которая задаётся урав"=нением $x^2+y^2=1$.\par

Тригонометрические функции соотносят {\ital координаты} точки единичной окружности
и {\ital градусную меру дуги}, образуемой ей с начальным радиусом.\par

{\bold Синус} --- нечётная функция с периодом $2\pi$; график --- {\ital синусоида}:
\\[-13pt]
% ---
$$\sin\colon\mathbb{R}\xrightarrow{\alpha\mapsto y}[-1;1]$$\\[-1pt]

Обратная нечётная функция к $\sin\vert_{[-\pi/2;\pi/2]}$ --- {\bold арксинус}:
% ---
$$\sin^{-1}=\arcsin\colon[-1;1]\xrightarrow{\alpha\mapsto y}[-\pi/2;\pi/2]$$
% ---
{\bold Косинус} --- чётная функция с периодом $2\pi$; график --- {\ital синусоида}
со смещением влево на $\pi/2$ {\ital\color{desc} («косинусоида»)}:\\[-6pt]
% ---
$$\cos\colon\mathbb{R}\xrightarrow{\alpha\mapsto x}[-1;1]$$\\[-1pt]

Обратная функция к $\cos\vert_{[0;\pi]}$ --- {\bold арккосинус}:
% ---
$$\cos^{-1}=\arccos\colon[-1;1]\xrightarrow{\alpha\mapsto x}[0;\pi]$$
% ---
{\bold Тангенс} --- нечётная функция с периодом $\pi$; график --- {\ital тангенсоида}:
\\[-13pt]
% ---
$$\tg\colon\mathbb{R}\backslash\{\pi/2+\pi n\mid n\in\mathbb{Z}\}\xrightarrow
{\alpha\mapsto y/x}\mathbb{R}$$\\[-1pt]

Обратная нечётная функция к $\tg\vert_{(-\pi/2;\pi/2)}$ --- {\bold арктангенс}:
% ---
$$\tg^{-1}=\arctg\colon\mathbb{R}\xrightarrow{y/x\mapsto\alpha}(-\pi/2;\pi/2)$$
% ---
{\bold Котангенс} --- нечётная функция с периодом $\pi$; график --- {\ital тангенсоида}
с симметрией относительно оси $Ox$ и смеще"=нием вправо на $\pi/2$ {\ital\color{desc} 
(«котангенсоида»)}:\\[-7pt]
% ---
$$\ctg\colon\mathbb{R}\backslash\{\pi n\mid n\in\mathbb{Z}\}\xrightarrow{\alpha\mapsto
x/y}\mathbb{R}$$\\[-1pt]

Обратная функция к $\ctg\vert_{(0;\pi)}$ --- {\bold арккотангенс}:
% ---
$$\ctg^{-1}=\arcctg\colon\mathbb{R}\xrightarrow{x/y\mapsto\alpha}(0;\pi)$$

\subsection{Основные тождества}

Из определений тригонометрических функций следует:
% ---
$$\begin{aligned}
&\sin^2\alpha+\cos^2\alpha=1 & \arccos x&=\arcsin(\sqrt{1-x^2})\\
&1+\tg^2\alpha=1/\cos^2\alpha & \arccos x&=\arctg(\sqrt{1-x^2}/x)\\
&1+\ctg^2\alpha=1/\sin^2\alpha & \arcsin y&=\arcctg(\sqrt{1-y^2}/y)
\end{aligned}$$

\subsection{Сумма и разность двух углов}

Из скалярного произведения векторов следует:
% ---
$$\begin{gathered}
\cos(\alpha\pm\beta)=\cos\alpha\cos\beta\mp\sin\alpha\sin\beta\\
\sin(\alpha\pm\beta)=\sin\alpha\cos\beta\pm\sin\beta\cos\alpha\\
\begin{aligned}
\tg(\alpha\pm\beta)=\frac{\tg\alpha\pm\tg\beta}{1\mp\tg\alpha\tg\beta} &\qquad
\ctg(\alpha\pm\beta)=\frac{\ctg\alpha\ctg\beta\mp 1}{\ctg\alpha\pm\ctg\beta}
\end{aligned}
\end{gathered}$$
% ---
{\bold Доказательство.} Пусть $\vec{A}=\langle\cos\alpha;\sin\alpha\rangle,\ 
\vec{B}=\langle\cos\beta;\sin\beta\rangle$.\par

Рассмотрим их скалярное произведение:
% ---
$$\begin{gathered}
+\begin{cases*}
&\vec{A}\cdot\vec{B}=\cos\alpha\cos\beta+\sin\alpha\sin\beta\\
&\vec{A}\cdot\vec{B}=\norm{\vec{A}}\ \norm{\vec{B}}\cos(\alpha-\beta)=\cos(\alpha-\beta)
\end{cases*}\implies\\
\cos(\alpha-\beta)=\cos\alpha\cos\beta+\sin\alpha\sin\beta\qedw
\end{gathered}$$
% ---
Затем полезно применить эти четыре формулы:
% ---
$$\begin{gathered}
\begin{aligned}
\alpha+\beta&=\alpha-(-\beta)\\
\sin(\alpha-\beta)&=\cos((\pi/2-\alpha)+\beta)
\end{aligned}\\
\tg\alpha=\sin\alpha/\cos\alpha\quad\quad\ctg\alpha=\cos\alpha/\sin\alpha\qedb
\end{gathered}$$

\subsection{Двойной угол}

Из формул суммы и разности двух углов следует:
% ---
$$\begin{gathered}
\begin{aligned}
\begin{aligned}
\cos 2\alpha&=\cos^2\alpha-\sin^2\alpha\\
\tg 2\alpha&=\frac{2\tg\alpha}{1-\tg^2\alpha}
\end{aligned} &\qquad
\begin{aligned}
\sin 2\alpha&=2\sin\alpha\cos\alpha\\
\ctg 2\alpha&=\frac{\ctg^2\alpha-1}{2\ctg\alpha}
\end{aligned}
\end{aligned}\\
(\sin\alpha\pm\cos\alpha)^2=1\pm\sin 2\alpha
\end{gathered}$$

\subsection{Формулы приведения}

\begin{column*}{r}{78mm}
Из формул суммы и разности двух углов следуют {\ital формулы приведения}, которые
имеют вид:
% ---
$$f(\pi n/2\pm\alpha)={\color{desc}\pm co}f(\alpha),\ n\in\mathbb{Z}$$
% ---
Конечная функция и её знак опреде"=ляются по графику; стрелками обозна"=чены места
смены функции на~{\ital кофункцию}.
\end{column*}
% ---
\begin{column*}{r}{124pt}
\begin{tikzpicture}
\draw [->, line width=2.5pt, desc, name path=x-axis] (-2.1,0) -- (2.1,0) node [below, black] {$x$};
\draw [->, line width=2.5pt, desc, name path=y-axis] (0,-2.1) -- (0,2.1) node [below right=3pt, yshift=6pt, black] {$y$};
\draw [dashed, line width=2.5pt, desc, name path=circle] (0,0) circle (1.5);
\draw [fill=black, name intersections={of=circle and x-axis}]
  (intersection-1) circle (3pt) node [above right] {$0$}
  (intersection-2) circle (3pt) node [above left] {$\pi$};
\draw [fill=black, name intersections={of=circle and y-axis}]
  (intersection-1) circle (3pt) node [above left, yshift=-3pt] {$\pi/2$}
  (intersection-2) circle (3pt) node [below right] {$3\pi/2$};
\draw [<->, line width=2.75pt, black] ([shift={(0,0)}] 55:1.5) arc [start angle=55, end angle=125, radius=1.5];
\draw [<->, line width=2.75pt, black] ([shift={(0,0)}] -55:1.5) arc [start angle=-55, end angle=-125, radius=1.5];
\node [anchor=south west, align=left, scale=.8] at (-1pt,-1pt) {{\bold I}\\[-4pt]$+$ sin\\[-4pt]$+$ cos\\[-4pt]$+$ (c)tg};
\node [anchor=south east, align=right, scale=.8] at (1pt,-1pt) {{\bold II}\\[-4pt]sin $+$\\[-4pt]cos $-$\\[-4pt](c)tg $-$};
\node [anchor=north east, align=right, scale=.8] at (1pt,1pt) {sin $-$\\[-4pt]cos $-$\\[-4pt](c)tg $+$\\[0pt]{\bold III}};
\node [anchor=north west, align=left, scale=.8] at (-1pt,1pt) {$-$ sin\\[-4pt]$+$ cos\\[-4pt]$-$ (c)tg\\[0pt]{\bold IV}};
\end{tikzpicture}
\end{column*}

{\bold Следствие.} Для обратных функций верно:
% ---
$$\begin{aligned}
\begin{aligned}
\arcsin x&=\pi/2-\arccos x\\
\arctg x&=\pi/2-\arcctg x
\end{aligned} &\qquad
\begin{aligned}
\arccos(-x)&=\pi-\arccos x\\
\arcctg(-x)&=\pi-\arcctg x
\end{aligned}
\end{aligned}$$

\subsection{Формулы понижения степени}

Из формул двойного угла и основного тригонометрического тождества следует:
% ---
$$\begin{aligned}
\begin{aligned}
2\cos^2\frac{\alpha}{2}&=1+\cos\alpha\\
2\sin^2\frac{\alpha}{2}&=1-\cos\alpha
\end{aligned} &\qquad
\begin{aligned}
\tg^2\frac{\alpha}{2}&=\frac{1-\cos\alpha}{\cos\alpha+1}\\
\ctg^2\frac{\alpha}{2}&=\frac{1-\cos\alpha}{\cos\alpha+1}
\end{aligned}
\end{aligned}$$
% ---
Из них легко выводятся формулы {\ital половинного угла}.

\subsection{Сумма и разность двух функций}

Из формул суммы и разности двух углов следует:
% ---
$$\begin{aligned}
\sin\alpha\pm\sin\beta&=2\sin\frac{\alpha\pm\beta}{2}\cos\frac{\alpha\mp\beta}{2}\\
\cos\alpha+\cos\beta&=2\cos\frac{\alpha+\beta}{2}\cos\frac{\alpha-\beta}{2}\\
\cos\alpha-\cos\beta&=-2\sin\frac{\alpha+\beta}{2}\sin\frac{\alpha-\beta}{2}\\
\end{aligned}$$
% ---
Из них можно вывести формулы {\ital произведения двух функций}.

{\bold Доказательство.} Рассмотрим сумму синусов:
% ---
$$\sin(x+y)+\sin(x-y)=$$
$$\sin x\cos y+\sin y\cos x+\sin x\cos y-\sin y\cos x=2\sin x\cos y$$
% ---
Введём обозначения:
% ---
$$\begin{cases*}
&x+y=\alpha\\
&x-y=\beta
\end{cases*}\iff
\begin{cases*}
&2x=\alpha+\beta\\
&2y=\alpha-\beta
\end{cases*}\iff
\begin{cases*}
&x=\frac{\alpha+\beta}{2}\\
&y=\frac{\alpha-\beta}{2}
\end{cases*}$$
% ---
Таким образом:
% ---
$$\sin\alpha+\sin\beta=2\sin\frac{\alpha+\beta}{2}\cos\frac{\alpha-\beta}{2}\qedw$$
% ---
Остальные формулы доказываются аналогично.$\qedb$

\begin{theorem}
Для $c=\sqrt{a^2+b^2}$ справедливо:
% ---
$$\begin{gathered}
a\sin\alpha+b\cos\alpha=c\sin(\alpha+\phi)=c\cos(\alpha-\phi)\\
\abs{a\sin\alpha+b\cos\alpha}\leq c
\end{gathered}$$
\end{theorem}

{\bold Доказательство.} Рассмотрим синус суммы двух углов:
% ---
$$c\sin(\alpha+\phi)=c\sin\alpha\cos\phi+c\sin\phi\cos\alpha$$
% ---
Обозначим $a=c\cos\phi$, $b=c\sin\phi$ и найдём сумму квадратов:
% ---
$$a^2+b^2=c^2(\sin^2\phi+\cos^2\phi)=c^2\iff c=\sqrt{(a^2+b^2)}\qedw$$
% ---
Случай с косинусом доказывается аналогично.$\qedb$

\subsection{Подстановка Вейерштрасса}

Тригонометрические функции от $2\alpha$ можно выразить через тангенс от $\alpha$
{\ital\color{desc}(К. Вейерштрасс)}:
% ---
$$\sin 2\alpha=\frac{2\tg\alpha}{1+\tg^2\alpha}\qquad
\cos 2\alpha=\frac{1-\tg^2\alpha}{1+\tg^2\alpha}$$
% ---
{\bold Доказательство.} Распишем каждую функцию:
% ---
$$\begin{gathered}
\sin 2\alpha=\frac{2\sin\alpha\cos\alpha}{\sin^2\alpha+\cos^2\alpha}=\frac{2\tg\alpha}{1+\tg^2\alpha}\qedw\\
\cos 2\alpha=\frac{\cos^2\alpha-\sin^2\alpha}{\sin^2\alpha+\cos^2\alpha}=\frac{1-\tg^2\alpha}{1+\tg^2\alpha}\qedb
\end{gathered}$$
