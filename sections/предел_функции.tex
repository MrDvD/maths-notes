\section{Предел функции}

\subsection{Определение}

{\bold Предел} функции $f$ в точке $x_0$ --- такое $a$, что {\ital\color{desc} (О.Л. Коши)}
% ---
$$\forall\varepsilon\greater 0\ \exists\delta\greater 0\colon\underbrace{
\forall x\in\overset{\circ}{U}_\delta(x_0)}_{\text{\tinyt I}},\ \underbrace{f(x)\in U_\varepsilon(a)}_{\text{\tinyt II}}.$$\\
% ---
\begin{tabularc}{0pt}{0pt}{>{\raggedleft\arraybackslash}p{.03\linewidth} @{ --- } 
>{\raggedright\arraybackslash}p{.91\linewidth}}{n}
I & функция $f$ определена в какой-либо проколотой $\delta$-окрестности точки $x_0$;
\\[18pt]
II & функция $f$ имеет образ в какой-либо проколотой $\varepsilon$-окрестности точки 
$a$.  
\end{tabularc}

{\bold Предел} функции $f$ в точке $x_0$ --- такое
$a$, что {\ital\color{desc} (Э. Гейне)}
% ---
$$\forall\{x_n\}\in D_f\colon\lim_{n\to\infty}x_n=x_0\ (x_n\neq x_0)\implies\lim_{n\to
\infty}f(x_n)=a.$$
% ---
Упрощённая запись $\lim_{x\to x_0}f(x)=a$ или $x\to x_0$, $f(x)\to a$.

\subsection{Бесконечно малая и большая}

{\bold Бесконечно малой} {\ital\color{desc}(б.м.)} называется такая функция $\alpha(x)$ при $x\to x_0$, что:
% ---
$$\lim_{x\to x_0}\alpha(x)=0$$
% --
\begin{theorem}
{\bold Связь предела и б.м.} Если функция $f$ имеет предел $\lim_{x\to x_0}f(x)=a$, то справедливо:
% ---
$$f(x)=a+\underset{x\to x_0}{\alpha(x)},\ \alpha\text{ --- б.м.}$$
\end{theorem}

{\bold Бесконечно большой} {\ital\color{desc}(б.б.)} называется такая функция $y(x)$ при $x\to x_0$, что:
% ---
$$\lim_{x\to x_0}y(x)=\infty$$
% ---
\begin{theorem}
{\bold Связь бесконечно малой и большой.} Верен факт: 
% ---
\begin{align*}
\underset{x\to x_0}{\alpha(x)}\text{ --- б.м., }\forall x\in\overset{\circ}{U}_{x_0}\ \alpha(x)\neq 0\iff\frac{1}{\alpha}\text{ --- б.б.}
\end{align*}
\end{theorem}

\subsection{Композиция функций}

\begin{theorem}
Пусть $f,g$ --- функции. Тогда:
% ---
$$\begin{cases}
\lim_{x\to x_0}f(x)=y_0\\
\lim_{y\to y_0}g(x)=z_0
\end{cases}\hspace*{-12pt}\implies\begin{cases}
\lim_{x\to x_0}(g\circ f)(x)=z_0\\
f(x)\neq y_0
\end{cases}$$
\end{theorem}
% ---
{\bold Доказательство.} Пусть $g\circ f=\varphi$; по определению предела:\\[-14pt]
% ---
$$\begin{cases}
\forall\varepsilon\greater 0\ \exists\delta\greater 0\colon\forall y\in\overset{\circ}{U}_\delta(y_0)\subseteq D_g,\ f(x)\in U_\varepsilon(z_0)\\
\forall\delta\greater 0\ \exists\sigma\greater 0\colon\forall x\in\overset{\circ}{U}_\sigma(x_0)\subseteq D_f,\ g(x)\in U_\delta(y_0)
\end{cases}$$\\[-6pt]
% ---
Из $\overset{\circ}{U}_\delta(y_0)\cap U_\delta(y_0)=\overset{\circ}{U}_\delta(y_0)$ 
следует:\\[-14pt]
% ---
$$\begin{cases}
\forall\varepsilon\greater 0\ \exists\sigma\greater 0\colon\forall x\in\overset{\circ}{U}_\sigma(x_0)\subseteq D_f,\ \varphi(x)\in U_\varepsilon(z_0)\\
y\neq y_0\implies f(x)\neq y_0
\end{cases}\hspace*{-12pt}\iff$$
$$\lim_{x\to x_0}\varphi(x)=z_0,\ f(x)\neq y_0\qedb$$

\subsection{Односторонний предел}

{\bold Односторонним} {\ital\color{desc}(правым или левым)} называется предел функции, который определён в терминах односторонних окрестностей {\ital (монотонных последовательностей)}:
% ---
$$\lim_{x\to x_0+0}f(x)=a\quad\text{или}\quad x\to x_0+0,\ f(x)\to a$$
$$\lim_{x\to x_0-0}f(x)=a\quad\text{или}\quad x\to x_0-0,\ f(x)\to a$$
% ---
Сущестование предела равносильно существованию {\ital равных} односторонних пределов:
% ---
$$\lim_{x\to x_0}f(x)\iff\lim_{x\to x_0+0}f(x)=\lim_{x\to x_0-0}f(x)$$

\subsection{Асимптота}

{\bold Асимптота} --- прямая, к которой {\ital неограниченно} прибли"=жается кривая, но не сливается с ней.

{\ital Горизонтальная асимптота} для графика функции $f$ задаётся уравнением:
% ---
$$y=\lim_{x\to\infty}f(x)$$
% ---
{\ital Наклонная асимптота} для графика функции $f$ задаётся уравнением $y=kx+b$, где
% ---
\begin{align*}
k&=\lim_{x\to\infty}\frac{f(x)}{x},\\
b&=\lim_{x\to\infty}(f(x)-kx).
\end{align*}
% ---
{\ital Вертикальная асимптота} для графика функции $f$ задаётся уравнением $x=a$, где
% ---
$$\lim_{x\to a+0}f(x)=\infty\text{ или }\lim_{x\to a-0}f(x)=\infty.$$

\subsection{Непрерывность}

Пусть $f\colon X\to Y$ --- функция. Тогда:

\begin{tabularc}{0pt}{0pt}{r @{ --- } l}{n}
$x-x_0=:\Delta x$ & {\ital приращение аргумента} в точке $x_0$\\
$f(x)-f(x_0)=:\Delta f$ & {\ital приращение функции} в точке $x_0$
\end{tabularc}

Функция $f$ {\bold непрерывна} {\ital в точке} $x_0$, если
% ---
$$\lim_{x\to x_0}f(x)=f(x_0)\quad\text{или}\quad\Delta x\to 0,\ \Delta f\to 0.$$
% ---
{\bold Односторонняя непрерывность} в точке $x_0$ определяется через односторонние пределы.

Непрерывными в точке $x_0$ являются {\ital сумма}, {\ital произведение}, {\ital частное {\color{desc}(предел знаменателя не равен нулю)}} и~{\ital композиция} непрерывных в ней функций.

Функция $f$ {\bold непрерывна} {\ital на промежутке} $[a;b]$, если она непрерывна в каждой точке этого промежутка:
% ---
$$f\in\mathbb{C}[a;b]\text{ --- нотация}$$
% ---
\begin{theorem}
{\bold Предел под непрерывной функцией.} Пусть $f,g$ --- функции, $g$ непрерывна в точке $x_0$. Тогда:
% ---
$$\lim_{x\to x_0}f(x)=a\implies\lim_{x\to x_0}(g\circ f)(x)=g(\lim_{x\to x_0}f(x))$$
\end{theorem}
% ---
Доказательство схоже с теоремой о пределе {\ital композиции функций}.

\subsection{Замечательные пределы}

Когда-нибудь это будет пояснено {\ital (я надеюсь)}:
% ---
$$\begin{aligned}
\lim_{x\to 0}\frac{\sin x}{x}=1\qquad & \lim_{x\to\infty}\left(1-\frac{1}{x}\right)^x=e
\end{aligned}$$

\subsection{Теорема о промежуточном значении}

\begin{theorem}
Пусть $f\in\mathbb{C}[a;b]$. Тогда справедливо:
% ---
$$\forall c\in[f(a);f(b)]\ \exists\xi\in[a;b]\colon c=f(\xi)$$
\end{theorem}
% ---
{\bold Доказательство.} По принципу Кантора:
% ---
$$\forall n\in\mathbb{N}\ \exists\xi\in[a_n;b_n]\subset[a_{n-1};b_{n-1}]\subseteq X
\implies$$
$$n\to\infty,\ a_n,b_n\to\xi$$
% ---
По определению непрерывности функции на промежутке:
% ---
$$n\to\infty,\ f(a_n),f(b_n)\to f(\xi)$$
% ---
По теореме о промежуточной функции:
% ---
$$f(a_n)\leq c\leq f(b_n)\implies c=f(\xi)\qedb$$
% ---
\begin{theorem}
{\bold Метод бисекции.} Пусть $f\in\mathbb{C}[a;b]$. Тогда справедливо:
% ---
$$\sgnn f(a)\neq\sgnn f(b)\implies\exists c\in[a;b]\colon f(c)=0$$
\end{theorem}

Используется, если нужно найти {\ital примерный} нуль функции.

\subsection{Критерий Коши}

\begin{theorem}
Сходимость $\iff$ выполнение {\ital условия Коши}:\\[-8pt]
% ---
$$\forall\varepsilon\greater 0\ \exists\delta\greater 0\colon\forall x',x''\in\overset
{\circ}{U}_\delta(x_0)\ \abs{f(x')-f(x'')}\less\varepsilon$$
\end{theorem}
% ---
{\bold Доказательство $\implies$.} По определению предела:\\[-8pt]
% ---
$$\forall\varepsilon\greater 0\ \exists\delta\greater 0\colon\overset{\circ}{U}_\delta
(x_0)\subseteq D_f,\ U_{\varepsilon/2}(a)\cap E_f\neq\emptyset$$\\[-6pt]
% ---
Пусть $x',x''\in\overset{\circ}{U}_\delta(x_0)$; по неравенству треугольника:
% ---
$$\abs{f(x')-f(x'')}\leq\abs{f(x')-a}+\abs{f(x'')-a}\less\varepsilon/2+\varepsilon/2=
\varepsilon\qedb$$
% ---
{\bold Доказательство $\impliedby$.} По условию Коши:
% ---
$$\exists\{x_n\}\in D_f\colon\lim_{n\to\infty}x_n=x_0,\ x_n\neq x_0$$
% ---
Последовательности $\{f(x_n)\}$ фундаментальны $\implies$ сходятся.

По фундаментальности и сходимости к одной точке $x_0$:
% ---
$$\lim_{x\to x_0}f(x)=a\qedb$$

\subsection{Теорема Вейерштрасса}

\begin{theorem}
Пусть $f\in C[a;b]$. Тогда в некоторых точках отрезка функция достигает своих точных 
верхней и нижней границ на $[a;b]$.
\end{theorem}
% ---
{\bold Доказательство.} Пусть $\sup f([a;b])=:M$, $\inf f([a;b])=:m$.

По определению точных верхней и нижней границ:
% ---
$$\forall x\in[a;b]\ f(x)\in[m;M]$$
% ---
По принципу компактности отрезка:
% ---
$$\lim_{n\to\infty}f(x_n)=M\quad\quad\lim_{k\to\infty}x_{n_k}=\xi$$
% ---
По определению непрерывности:
% ---
$$\lim_{k\to\infty}f(x_{n_k})=f(\xi)\implies f(\xi)=M\qedb$$
