\section{Алгоритмы на строках}

\subsection{Префикс-функция}

{\bold Собственным} называется {\ital префикс}, который не совпадает со всей строкой.

{\bold Грань} --- собственный префикс строки, который равен её суффиксу.

{\bold Префикс-функция} строки $S$ --- массив, $i$-ый элемент которого равен длине максимальной грани подстроки $S[0\ldots i]$.

{\bold Алгоритм} реализации {\ital префикс-функции}:

\begin{code}{python}
prefixes = [0] * len(S)
def fill_prefixes(S, prefixes):
  for i in range(len(S) - 1):
    j = 0
    while j and S[i + 1] != S[j]: # I
      j = prefixes[j - 1]
    if S[i + 1] == S[j]:
      j += 1
    prefixes[i + 1] = j
\end{code}

\begin{tabularx}{\textwidth}{r|C|L}
\bold I &
\begin{tikzpicture}[baseline]
\draw [->, line width=2pt, desc, yshift=1pt] (1.4,0) to [in=90, out=90] (-.2,0);
\draw [->, line width=2pt] (-2,0) -- (2,0);
\draw [fill=black] (.6,0) circle (2.5pt) node [below] {$i$};
\draw [fill=black] (1.4,0) circle (2.5pt) node [below] {$i+1$};
\draw [fill=black] (-.2,0) circle (2.5pt) node [below] {$j$};
\draw [fill=black] (-1,0) circle (2.5pt) node [below] {$j-1$};
\end{tikzpicture} &
Цикл смены индекса $j$ идёт до равенства с $S(i+1)$ или до границ индексации.
\end{tabularx}

\subsection{Алгоритм КМП}

\begin{theorem}
{\bold Задача.} Даны строки {\ital text} и {\ital search}. Найти позиции всех вхождений {\ital search} в {\ital text}.
\end{theorem}

{\bold Идея.} Образовать новую строку конкатенацией:
% ---
$$search+\#+text$$
% ---
Вычислить {\ital префикс-функцию} от новой строки: индексы элементов, которые численно равны длине {\ital search}, будут ответом. 

Алгоритм {\ital онлайновый}...

\subsection{Вхождения префиксов}

\begin{theorem}
{\bold Задача.} Дана строка $S$. Посчитать для каждого префикса $S$, сколько раз он встречается в $S$.
\end{theorem}

{\bold Идея.} Вычислить префикс-функцию от $S$. Затем составить отдельный массив {\ital occur} по принципу:
% ---
\begin{list*}[][\#]
\item Посчитать для каждого значения $\pi$, сколько раз оно встречалось в $\pi$.
\item Посчитать {\ital динамику}: вхождения большего префикса $i$ добавить к наибольшему собственному суффиксу этого префикса $prefixes[i-1]$.
\end{list*}
% ---
\begin{code}{python}
fill_prefixes()
for i in range(len(S)):
  occur[prefixes[i]] += 1
for i in range(len(S) - 1, 0, -1):
  occur[prefixes[i - 1]] += occur[i]
\end{code}

\subsection{Учёт различных подстрок}

\begin{theorem}
{\bold Задача.} Дана строка $S$. Посчитать количество её различных подстрок.
\end{theorem}

{\bold Идея.} Добавим один символ в конец неполной строки: появятся новые подстроки, которые оканчиваются в новом символе. Нужно найти {\ital уникальные} среди них.

Инвертируем строку, чтобы символ стал префиксом. Тогда число уникальных подстрок равно числу элементов $\pi$ с нулевым значением:
% ---
$$\text{len}(S)-\pi_\text{\tinyt max}$$
% ---
Аналогично можно дописывать символы в начало или удалять символы с конца или с начала.

\subsection{Сжатие строки}

\begin{theorem}
{\bold Задача.} Дана строка $S$. Найти такую строку наименьшей длины, что $S$ можно представить в виде конкатенации её копий.
\end{theorem}

{\bold Идея.} Пусть $n=\text{len}(S)$, $k=n-\pi[n-1]$. Тогда верно:
% ---
$$k\mid n\implies k\text{ --- длина искомой строки}$$
% ---
Иначе $S$ {\ital невозможно} сжать.

\subsection{Z-функция}

{\bold Блок} строки $S$ --- подстрока $S$, которая равна её собственному префиксу.

{\bold Z-функция} строки $S$ --- массив, $i$-ый элемент которого равен длине максимального блока с началом в $i$.

{\bold Алгоритм} реализации {\ital z-функции}:

\begin{code}{python}
def get_block(S, l, r):
  idx = 0
  while S[l + idx] == S[r + idx]:
    if r + idx < len(S):
      break
    idx += 1
  return idx

blocks = [0] * len(S)
def fill_blocks(string, blocks):
  l, r = 0, 0
  for i in range(len(S)):
    if i > r: # I
      value = get_block(S, 0, i)
      if value:
        l, r = i, value
    else: # II
      k = i - l
      if blocks[k] < r - i + 1:
        blocks[i] = blocks[k]
      else:
        blocks[i] = r - i + 1
        l, q = i, get_block(S, blocks[i], r + 1)
        if q:
          blocks[i] += q
          r = i + blocks[i] - 1
\end{code}

\begin{tabularx}{\textwidth}{r|C|L}
\bold I &
\begin{tikzpicture}[baseline]
\draw [pattern=north east lines] (-2,0) rectangle (-1.3,.35);
\draw [pattern=north east lines] (1,0) rectangle (1.7,.35);
\draw [->, line width=2pt] (-2,0) -- (2,0);
\draw [fill=black] (-2,0) circle (2.5pt) node [below] {$0$};
\draw [fill=black] (-1,0) circle (2.5pt) node [below] {$l$};
\draw [fill=black] (0,0) circle (2.5pt) node [below] {$r$};
\draw [fill=black] (1,0) circle (2.5pt) node [below] {$i$};
\end{tikzpicture} &
Поиск нового блока с позиций $0$ и $i$, обновление $l$, $r$ при нахождении\\\hline
\multirow{2}{*}{\bold II} &
\begin{tikzpicture}[baseline]
\draw (-2.5,0) rectangle (-.9,.5);
\draw (.1,0) rectangle (1.7,.5);
\draw [pattern=north east lines] (-1.7,0) rectangle (-1.4,.35);
\draw [pattern=north east lines] (.9,0) rectangle (1.7,.35);
\draw [->, line width=2pt] (-2.5,0) -- (2.5,0);
\draw [fill=black] (-2.5,0) circle (2.5pt) node [below] {$0$};
\draw [fill=black] (-1.7,0) circle (2.5pt) node [below] {$k$};
\draw [fill=black] (.1,0) circle (2.5pt) node [below] {$l$};
\draw [fill=black] (.9,0) circle (2.5pt) node [below] {$i$};
\draw [fill=black] (1.7,0) circle (2.5pt) node [below] {$r$};
\end{tikzpicture} &
\multirow{2}{=}{Рассмотреть две ситуации}\\\cline{2-2}
& \begin{tikzpicture}[baseline]
\draw (-2.5,0) rectangle (-.9,.5);
\draw (.1,0) rectangle (1.7,.5);
\draw [pattern=north east lines] (-1.7,0) rectangle (-.4,.35);
\draw [pattern=north east lines] (.9,0) rectangle (1.7,.35);
\draw [pattern=north east lines, dashed] (1.7,0) rectangle (2.2,.35) node [midway, yshift=4pt] {$q$};
\draw [->, line width=2pt] (-2.5,0) -- (2.5,0);
\draw [fill=black] (-2.5,0) circle (2.5pt) node [below] {$0$};
\draw [fill=black] (-1.7,0) circle (2.5pt) node [below] {$k$};
\draw [fill=black] (.1,0) circle (2.5pt) node [below] {$l$};
\draw [fill=black] (.9,0) circle (2.5pt) node [below] {$i$};
\draw [fill=black] (1.7,0) circle (2.5pt) node [below] {$r$};
\end{tikzpicture} &\\
\end{tabularx}

\subsection{Редакционное расстояние}

К {\bold элементарным операциям редактирования} строки относятся:
% ---
\begin{list*}
\item {\ital удаление} символа;
\item {\ital вставка} символа;
\item {\ital замещение} символа.
\end{list*} 
% ---
{\bold Редакционное расстояние} между строками $S_1$ и $S_2$ --- минимальное количество {\ital элементарных операций редактирования}, которые нужно совершить, чтобы перевести $S_1$ в $S_2$.

{\bold Редакционный граф} для строк $S_1$ и $S_2$ --- орграф с вершинами-состояниями строк, у которых есть до трёх рёбер, {\ital эквивалентных} элементарным операциям редактирования:

{\centering
\begin{tikzpicture}
\node (start) [draw, line width=1pt, align=left] at (0,1) {$S_1[0\dots n]$\\$S_2[0\dots m]$};
\node (remove) [draw, line width=1pt, anchor=east, align=left] at (-1.5,-1) {$S_1[1\dots n]$\\$S_2[0\dots m]$};
\node (change) [draw, line width=1pt, align=left] at (0,-1) {$S_1[1\dots n]$\\$S_2[1\dots m]$};
\node (insert) [draw, line width=1pt, anchor=west, align=left] at (1.5,-1) {$S_1[0\dots n]$\\$S_2[1\dots m]$};
\draw [-{>[length=6pt]}, line width=.8pt] (start.west) -- (remove.north) node [midway, above, sloped] {De};
\draw [-{>[length=6pt]}, line width=.8pt] (start.south) -- (change.north) node [midway, right, yshift=2pt] {Re};
\draw [-{>[length=6pt]}, line width=.8pt] (start.east) -- (insert.north) node [midway, above, sloped] {In};
\end{tikzpicture}\par}

\subsection{Алгоритм Вагнера-Фишера}

{\bold Редакционное предписание} --- сертификат редакционного расстояния между двумя строками:

{\centering{\bold D}elete, {\bold I}nsert, {\bold R}eplace, {\bold M}atch.\par}

{\bold Алгоритм Вагнера-Фишера} находит редакционное предписание по {\ital матрице расстояний}, минимизируя выбор элементарных операций редактирования.

\subsection{Общая подпоследовательность}

\begin{theorem}
{\bold Задача.} Даны строки $S_1$ и $S_2$. Определить длину их наибольшей общей подпоследовательности $LCS(S_1,S_2)$.
\end{theorem}

{\bold Идея.} Используем принцип оптимальности на префиксе:

{\centering\begin{tikzpicture}
\node (start) [draw, line width=1pt, align=left] at (0,0.5) {$LCS(n,m)$};
\node (neql) [draw, line width=1pt, anchor=east, align=left] at (-2.3,-1) {$LCS(n-1,m)$};
\node (eq) [draw, line width=1pt, align=left] at (0,-1) {$LCS(n-1,m-1)$};
\node (neqr) [draw, line width=1pt, anchor=west, align=left] at (2.3,-1) {$LCS(n,m-1)$};
\draw [-{>[length=6pt]}, line width=.8pt] (neql.north) -- (start.west) node [midway, above, sloped, scale=.8] {$S_1[n]\neq S_2[m]$};
\draw [-{>[length=6pt]}, line width=.8pt] (eq.north) -- (start.south) node [midway, yshift=2pt, scale=.8, fill=white, fill opacity=.6, text opacity=1] {$S_1[n]=S_2[m]$};
\draw [-{>[length=6pt]}, line width=.8pt] (neqr.north) -- (start.east) node [midway, above, sloped, scale=.8] {$S_1[n]\neq S_2[m]$};
\end{tikzpicture}\par}

\begin{list*}[][\#]
\item $S_1[n]=S_2[m]\implies$инкрементируем прошлый узел
\item $S_1[n]\neq S_2[m]\implies$максимизируем прошлые узлы
\end{list*}

\subsection{Алгоритм Нидлмана-Вунша}

\begin{theorem}
{\bold Задача.} Даны две строки $S_1$ и $S_2$. Составить их оптимальное выравнивание.
\end{theorem}

{\bold Идея.} Составить {\ital матрицу схожести} с весовой функцией:

\begin{tabularx}{\textwidth}{r@{ --- }L}
$+1$ & вес совпадения;\\
$-\mu$ & штраф за замену;\\
$-\delta$ & штраф за удаление, вставку.
\end{tabularx}

Задача сводится к нахождению пути с {\ital максимальным весом}.

\subsection{Скобочная последовательность}

Когда-нибудь...

\newpage
\subsection{Алгоритм сортировочной станции}

{\bold Обратная польская нотация} --- форма записи математических выражений, в которой операторы расположены {\ital после} операндов.

\begin{theorem}
{\bold Задача.} Дано математическое выражение в инфиксной нотации. Перевести его в постфиксную нотацию.
\end{theorem}

{\bold Идея.} Сначала запарсить строку {\ital регулярным выражением}:

{\centering\textbackslash d+\textbackslash.?\textbackslash d*|[+\textbackslash$-$*/\^{}()]\par}
% ---
Затем ввести две структуры: {\ital строку} с ответом и {\ital стек} для операторов. Их заполнение происходит при считывании по токену за раз:
% ---
\begin{list*}[][\#]
\item Токен --- {\ital число}$\implies$добавить к строке.
\item Токен --- {\ital бинарный оператор}:
\begin{list*}[2]
\item если приоритет последнего элемента в стеке $\geq$, чем у~токена, то вытолкнуть его из стека в строку {\ital\color{desc} (повторить при необходимости)};
\item добавить токен в стек.
\end{list*}
\item Токен --- {\ital открывающая скобка}$\implies$добавить в стек.
\item Токен --- {\ital закрывающая скобка}$\implies$вытолкнуть все операторы до «$($» из стека в строку, а «$($» удалить.  
\end{list*}
% ---
Для поддержки унарных операторов ввести флаги {\ital next\_unary} и {\ital is\_unary}.

\subsection{DP по цифрам}

\begin{theorem}
{\bold Задача.} Найти количество двоичных последовательностей из $n$ элементов без $k$ единиц подряд.
\end{theorem}

{\bold Идея.} Пусть $f(x)$ --- количество двоичных последовательностей из $x$ элементов без $k$ единиц подряд.

Определим базовые случаи:
% ---
$$\begin{aligned}
\begin{aligned}
&f(0)=2^0\\
&f(1)=2^1
\end{aligned} &\qquad
\begin{aligned}
&f(k-1)=2^{k-1}\\
&f(k)=2^k-1
\end{aligned}
\end{aligned}$$
% ---
Рассмотрим случаи, когда к последовательности дописывается одна цифра:
% ---
$$\begin{gathered}
011\dots 010+0=\underbrace{011\dots 010}_{f(x-1)\text{ seq.}}0\\
011\dots 010+1=\underbrace{011\dots 010}_{f(x-1)\text{ seq.}}1\\
\end{gathered}$$
% ---
Но в конце предыдущей последовательности может находиться $k-1$ единиц, поэтому исключим её:
% ---
$$\underbrace{011\dots 010}_{f(x-k-1)\text{ seq.}}\hspace*{-5pt}0\underbrace{1\dots 1}_{k-1}1$$
% ---
Значит, конечная рекуррентная формула имеет вид:
% ---
$$f(x)=2f(x-1)-f(x-k-1)$$

\begin{theorem}
{\bold Задача.} Найти количество $n$-значных чисел, у которых сумма любых двух соседних цифр --- простое число.
\end{theorem}

{\bold Идея.} Пусть $f(x,k)$ --- количество $x$-значных чисел, у которых сумма любых двух соседних цифр --- простое число, причём они оканчиваются на $k$.

Определим базовые случаи:
% ---
$$f(0,i)=0\qquad f(1,j)=1\text{, но }f(1,0)=0$$
% ---
Тогда рекуррентная формула примет вид:
% ---
$$f(x,k)=\sum_{\mathclap{j+k\text{ --- простое}}}f(x-1,j)$$
% ---
\begin{theorem}
{\bold Задача.} Найти количество строк длины $n$, которые состоят из символов $a$, $b$, $c$ и не содержат подстроки $ab$.
\end{theorem}

{\bold Идея.} Пусть $f(x,k)$ --- количество строк длины $x$, которые оканчиваются символом $k$.

Определим базовые случаи:
% ---
$$f(0,k)=1\qquad f(1,k)=1$$
% ---
Тогда рекуррентная формула примет вид:
% ---
$$f(x,k)=\begin{cases*}
&f(x-1,a)+f(x-1,c), &&k=a\\
&f(x-1,a)+f(x-1,b)+f(x-1,c), &&k=b,c
\end{cases*}$$
% ---
Ответ --- сумма $f(n,a)+f(n,b)+f(n,c)$.

\begin{theorem}
{\bold Задача.} Назовём число {\ital интересным}, если его цифры идут в неубывающем порядке. Сколько интересных положительных чисел лежит в диапазоне $[L;R]$?
\end{theorem}

{\bold Идея.} Пусть $f(x,k)$ --- количество $x$-значных интересных чисел, которые оканчиваются на $k$ {\ital\color{desc}(ведущие нули допускаются)}.

Представим $R$ в виде $\overline{a_1\dots a_n}$. Пусть $y$ --- интересное число в~диапазоне $[1;R]$, причём $y=\overline{b_1\dots b_n}$:
% ---
\begin{list*}[][\#]
\item Если $b_i=a_i$, то $b_{i+1}\leq a_{i+1}$.
\item Если $b_i\less a_i$, то последующие цифры интересного числа --- любые.
\end{list*}

Когда-нибудь... возможно...

\begin{theorem}
{\bold Задача.} Назовём число {\ital гладким}, если его цифры идут в неубывающем порядке. Вывести $N$-ое гладкое число.
\end{theorem}

{\bold Идея.} Пусть $f(x,k)$ --- количество $x$-значных гладких чисел с~цифрой $k$ в $x$-ом разряде {\ital\color{desc}(отсчёт справа)}.

Определим базовый случай:
% ---
$$f(1,k)=1\text{ \ital\color{desc} (нуль считается)}$$
% ---
Тогда рекуррентная формула имеет вид:
% ---
$$f(x,k)=\sum_{j=k}^9f(n-1,j)$$
% ---
Определимся с числом разрядов у искомого числа:
% ---
$$\begin{cases*}
&f(n,0)\leq N\\
&f(n+1,0)\greater N
\end{cases*}\implies
n+1\text{ --- число разрядов}$$
% ---
Каждую цифру $N$-го числа найдём при помощи цикла:
% ---
\begin{list*}[][\#]
\item $count+f(x,k)\leq N\implies count\pluseq f(x,k)$
\item $count+f(x,k)\greater N\implies k$ --- $x$-ая цифра
\end{list*}

\begin{theorem}
{\bold Задача.} Назовём число {\ital плавным}, если две соседние цифры различаются не более, чем на $1$. Определить количество плавных натуральных чисел длины $n$.
\end{theorem}

{\bold Идея.} Когда-нибудь...

\begin{theorem}
{\bold Задача.} Дан диапазон чисел $[L;R]$. Посчитать сумму {\ital цифр} всех его натуральных чисел.
\end{theorem}

{\bold Идея.} Пусть $f(x)$ --- сумма цифр всех натуральных чисел из~диапазона $[1;x]$.

Посчитаем некоторые значения:
% ---
$$\begin{aligned}
f(9)&=1+\dots+9=45\\
f(99)&=f(9)+(10+f(9))+\dots+(90+f(9))\\
&=10f(9)+10\cdot 45\\
f(999)&=f(99)+(100+f(99))+\dots+(900+f(99))\\
&=10f(99)+100\cdot 45
\end{aligned}$$ 
% ---
Заметим, что $f(10^k-1)=10f(10^{k-1}-1)+10^{k-1}\cdot 45$.

Тут ещё добавить пример для $f(328)$...

Допустим, $k$ --- количество разрядов числа $x$, $m$ --- его {\ital MSD}. Тогда искомая величина складывается из двух диапазонов:
% ---
$$[1;m10^{k-1}-1]\cup[m10^{k-1};x]$$
% ---
Посчитаем суммы цифр чисел на них:
% ---
$$\begin{aligned}
f_1(x)&=mf(10^{k-1}-1)+\frac{m(m-1)}{2}10^{k-1}\\
f_2(x)&=m(x\modn{10^{k-1}}+1)+f(x\modn{10^{k-1}})
\end{aligned}$$
