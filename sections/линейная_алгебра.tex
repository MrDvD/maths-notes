\section{Линейная алгебра}

\subsection{Линейное пространство}

{\bold Пространство} $\mathbb{R}^n$ --- множество всех упорядоченных наборов:
% ---
$$(x_1,\dots,x_n)$$
% ---
{\bold Вектор} --- элемент пространства.

{\bold Компонента вектора} --- ...

{\bold Линейное {\ital (векторное)} пространство} --- ...

\subsection{Матрица}

{\bold Матрица} с коэффициентами из поля $\mathbb{K}$ --- ...

{\bold Единичная матрица} --- ...
% ---
$$.$$
% ---
{\bold Умножение матриц} --- ...
% ---
$$.$$
% ---
{\bold Транспонированная матрица} для $M\in\Mat_{\mathbb{K}}(n,m)$ --- такая $M^{\tau}\in\Mat_{\mathbb{K}}(m,n)$, что:
% ---
$$M^{\tau}=\norm{\tilde{a}_{ij}}\colon\tilde{a}_{ij}=a_{ji}$$
% ---
\begin{theorem}
{\bold Свойства.}
% ---
\begin{list*}[][\#]
\item Согласованность с матричным сложением:
% ---
$$\forall A,B\in\Mat_{\mathbb{K}}(n,m)\ (A+B)^\tau=A^\tau+B^\tau$$
\item Согласованность с умножением матрицы на скаляр:
% ---
$$\forall A\in\Mat_{\mathbb{K}}(n,m),\ \alpha\in\mathbb{K}\ (\alpha A)^\tau=\alpha A^\tau$$
\item Согласованность с матричным умножением:
% ---
$$\forall A,B\in\Mat_{\mathbb{K}}(n,m)\ (A\cdot B)^\tau=B^\tau\cdot A^\tau$$
\end{list*}
\end{theorem}
% ---
{\bold Определитель} --- числовая характеристика квадратной матрицы:
% ---
$$\det A=...$$
% ---
{\bold Минор матрицы} $n$ порядка $\mathcal{M}_{ij}$ --- определитель $n-1$ порядка, получающийся из определителя матрицы $A$ вычеркиванием $i$ строки и $j$ столбца.

{\bold Матрица миноров} --- ...

{\bold Алгебраическое дополнение} элемента $a_{ij}$ --- число, определяемое выражением:
% ---
$$\mathcal{A}_{ij}=(-1)^{i+j}\mathcal{M}_{ij}$$
% ---
{\bold Обратная матрица} --- ...
% ---
$$A^{-1}=\frac{\mathcal{M}^{\tau}}{\det A}$$

\subsection{Преобразования матрицы}

{\bold Элементарные преобразования матриц} --- преобразования вида:
% ---
\begin{list*}[][\#]
\item Прибавление к одной строке другой, умноженной на число.
\item Перестановка двух строк.
\item Умножение одной строки на число, отличное от нуля. 
\end{list*}

\begin{theorem}
{\bold Метод Крамера.} Да.
\end{theorem}
