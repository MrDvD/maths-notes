\section{Дифференциальное исчисление}

% ОПРЕДЕЛЕНИЕ МОНОТОННОСТИ ОКОЛО ТОЧКИ

\subsection{Дифференцируемость}

{\bold Дифференцируемой} {\ital{\color{desc}(«линейной в малом»)} в точке $x_0$} называется такая функция $f$, для которой справедливо:
% ---
$$\Delta f=(k+\underset{\Delta x\to 0}{\alpha(x)})\Delta x,\ \alpha\text{ --- б.м.}$$
% ---
{\ital Односторонняя дифференцируемость} в точке $x_0$ опреде"=ляется через односторонние пределы.

{\bold Дифференциал} {\ital функции} $f$ --- линейная часть $\Delta f$:
% ---
$$k\Delta x=:\diff f$$
% ---
{\bold Производная} {\ital в точке $x_0$} --- предел вида: {\ital\color{desc} (Ж.Л. Лагранж)}
% ---
$$k=\lim_{\Delta x\to 0}\frac{\Delta f}{\Delta x}=:f'(x_0)$$ 

\subsection{Свойства}

Таблица {\ital «дистрибуции»} производной:\par
% ---
\begin{tabularcx}{0pt}{0pt}{C @{\hspace*{-40pt}} C}{\textwidth}
{$\begin{aligned}
&(f+g)'=f'+g'\\
&(f\cdot g)'=f'g+fg'\\
&(f\circ g)'=(f'\circ g)g'
\end{aligned}$} &
{$\begin{aligned}
\left(\frac{f}{g}\right)'&=\frac{f'g-fg'}{g^2}\\
(kf)'&=kf',\ k=\text{const}
\end{aligned}$}
\end{tabularcx}

\begin{theorem}
Дифференцируемость $\implies$ непрерывность.
\end{theorem}

{\bold Доказательство.} По определению производной:
% ---
$$\lim_{\Delta x\to 0}\frac{\Delta f}{\Delta x}=f'(x_0)\iff \frac{\Delta f}{\Delta x}=f'(x_0)+\underset{\Delta x\to 0}{\alpha(x)}\Delta x\iff$$
$$\Delta f=\Delta x(f'(x_0)+\alpha(x)\Delta x)\implies\Delta x\to 0,\ \Delta f\to 0\qedb$$

\begin{theorem}
{\bold Производная обратной функции.} Пусть $y=f(x)$ --- дифференцируемая функция. Тогда справедливо:
% ---
$$f^{-1'}(y)=\frac{1}{f'(x)},\ f'(x)\neq 0$$
\end{theorem}

{\bold Доказательство.} По условию запишем тождество:
% ---
$$\frac{\Delta f}{\Delta x}=1\colon\frac{\Delta x}{\Delta f}$$
% ---
По предельному переходу и непрерывности функций:
% ---
$$\begin{gathered}\lim_{\Delta x\to 0}\frac{\Delta f}{\Delta x}=1\colon\lim_{\Delta f\to 0}\frac{\Delta x}{\Delta f}\overset{\text{\tinyt опр}}{\iff} f'(x)=1\colon f^{-1'}(y)\iff\\
\iff f^{-1'}(y)=\frac{1}{f'(x)},\ f'(x)\neq 0\qedb\end{gathered}$$
% ---
\begin{theorem}
{\bold Инвариантность.} Пусть $f(x)$, $g(x)$ --- дифференцируемые функции. Тогда верно:
% ---
$$\diff f=f'(x)\diff x\implies \diff(f\circ g)=f'(g(x))\diff g$$ 
\end{theorem}
% ---
{\bold Доказательство.} По правилам дифференцирования:
% ---
$$\diff(f\circ g)=f'(g(x))\diff x=f'(g(x))g'(x)\diff x=f'(g(x))\diff g\qedb$$

\subsection{Элементарные производные}

Таблица производных элементарных функций:
% ---
\begin{gather*}
\begin{aligned}
C'=0\qquad & (x^n)'=nx^{n-1},\ n\neq 0\qquad & \ln'x=1/x
\end{aligned}\\
\begin{align*}
\sin'\alpha&=\cos\alpha\hspace*{1cm} & \cos'\alpha&=-\sin\alpha\hspace*{2.05cm}\\
\tg'\alpha&=1/\cos^2\alpha & \ctg'\alpha&=-1/\sin^2\alpha\\
\arcsin'x&=1/\sqrt{1-x^2} & \arccos'x&=-1/\sqrt{1-x^2}\\
\arctg'x&=1/(1+x^2) & \arcctg'x&=-1/(1+x^2)
\end{align*}
\end{gather*}

\subsection{Касательная}

{\bold Касательная} к кривой в точке $x_0$ --- прямая, которая проходит через $x_0$ и представляет {\ital предельное} положение секущей при $x\to x_0$, или $\Delta x\to 0$.

\begin{theorem}
{\bold Геометрический смысл производной.} Угловой коэф"=фициент {\ital\color{desc}(тангенс)} касательной к графику функции $f$ равен {\ital производной} в этой точке:
% ---
$$k=\tg\alpha=f'(x_0)$$
\end{theorem}
% ---
{\bold Доказательство.} По определению касательной:
% ---
$$\lim_{x\to x_0}\frac{\Delta f}{\Delta x}=\tg\alpha=k$$
% ---
По определению производной:
% ---
$$f'(x_0)=\tg\alpha=k\qedb$$
% ---
\begin{theorem}
{\bold Уравнение касательной} к графику функции $f$ в точке $x_0$ имеет вид:
% ---
$$f'(x_0)=\frac{y-f(x_0)}{x-x_0}$$
\end{theorem}
% ---
{\bold Доказательство.} По уравнению секущей графика $f$:
% ---
$$\frac{x-x_0}{x_1-x_0}=\frac{y-f(x_0)}{f(x_1)-f(x_0)}\implies y-f(x_0)=\frac{f(x_1)-f(x_0)}{x_1-x_0}(x-x_0)$$
% ---
По определению касательной:
% ---
$$y-f(x_0)=\lim_{x_1\to x_0}\frac{f(x_1)-f(x_0)}{x_1-x_0}(x-x_0)=f'(x_0)(x-x_0)\qedb$$

\subsection{Нормаль}

{\bold Нормаль} к кривой в точке $x_0$ --- прямая, которая проходит через $x_0$ и образует с касательной в $x_0$ {\ital прямой угол}:
% ---
$$\vec{n}=\begin{pmatrix}
A\\B\end{pmatrix}\text{ для }Ax+By+C=0$$
% ---
{\bold Доказательство.} По скалярному произведению векторов:
% ---
$$\begin{gathered}(\vec{f}'\cdot\vec{n})=0\implies A(x-x_0)+B(y-y_0)=0\\
Ax-Ax_0+By-By_0=0
\end{gathered}$$
% ---
Пусть $C=-(Ax_0+By_0)$, тогда:
% ---
$$Ax+By+C=0\qedb$$
% ---
\begin{theorem}
{\bold Теорема.} Прямые $f_1$ и $f_2$ перпендикулярны, если:
% ---
$$(\vec{n}_1\cdot\vec{n}_2)=0\text{, или }k_1=-\frac{1}{k_2}$$
\end{theorem}
% ---
{\bold Доказательство.} По скалярному произведению векторов:
% ---
$$\begin{gathered}
(\vec{n_1}\cdot\vec{n}_2)=0\implies\cos\alpha=0\implies\alpha=\frac{\pi}{2}\qedw\\
(\vec{n_1}\cdot\vec{n}_2)=0\implies k_1k_2+1=0\implies k_1=-\frac{1}{k_2}\qedb
\end{gathered}$$
% ---
\begin{theorem}
{\bold Теорема.} Уравнение прямой по точке и нормальному вектору:
% ---
$$(\vec{n}\cdot\vec{l})=0\text{, или }A(x-x_0)+B(y-y_0)=0$$
\end{theorem}

\subsection{Промежутки монотонности}

\begin{theorem}
Если функция $f$ дифференцируема в точке $x_0$, то
% ---
$$\begin{cases}
f'(x_0)\greater 0\implies f\kern-4pt\uparrow\text{около }x_0\\
f'(x_0)\less 0\implies f\kern-4pt\downarrow\text{около }x_0
\end{cases}\hspace*{-12pt}.$$
\end{theorem}
% ---
{\bold Доказательство.} По определению производной:
% ---
$$f'(x_0)\greater 0\iff \lim_{\Delta x\to 0}\frac{\Delta f}{\Delta x}\greater 0\iff\frac
{\Delta f}{\Delta x}\greater o(\Delta x)$$
% ---
При достаточно малом $\Delta x$ верно:
% ---
$$\frac{\Delta f}{\Delta x}\greater 0\iff\begin{sqcases*}
&\Delta f,\Delta x\greater 0\\
&\Delta f,\Delta x\less 0
\end{sqcases*}\iff
f\kern-4pt\uparrow\text{около }x_0\qedw$$
% ---
Для $f'(x_0)\less 0$ доказательство аналогично.$\qedb$

\subsection{Экстремум}

{\ital Локальный} {\bold максимум} функции $f$ --- такая точка $x_0$, что:
% ---
$$\exists\delta\greater 0\colon\sup U_\delta(x_0)=f(x_0)$$ 
% ---
{\ital Локальный} {\bold минимум} функции $f$ --- такая точка $x_0$, что:
% ---
$$\exists\delta\greater 0\colon\inf U_\delta(x_0)=f(x_0)$$
% ---
Их объединяют в точки {\bold экстремума}.
 
{\bold Критической} называется такая точка $x_0$, в которой:
% ---
$$\begin{sqcases*}
&f'(x_0)=0\text{ \ital\color{desc}(стационарна)}\\
&f'(x_0)=\text{undefined}
\end{sqcases*}$$
% ---
\begin{theorem}
Экстремум --- {\ital критическая точка} первого порядка. {\ital\color{desc} (не~наоборот)}
\end{theorem}

{\bold Доказательство.} По определению локального максимума:
% ---
$$\exists\delta\greater 0\colon\forall x\in\overset{\circ}{U}_\delta(x_0)\ f(x_0)\greater 
f(x)$$
% ---
Производная в точке $x_0$ либо существует, либо нет.$\qedw$

Допустим, она существует; по определению производной:
% ---
$$\lim_{x\to x_0}\frac{\Delta f}{\Delta x}=f'(x_0)$$
% ---
По предельному переходу:
% ---
$$\begin{sqcases*}
&\Delta x\greater 0\implies\Delta f/\Delta x\less 0\implies f'(x_0)\leq 0\\
&\Delta x\less 0\implies\Delta f/\Delta x\greater 0\implies f'(x_0)\geq 0
\end{sqcases*}\iff$$
$$0\leq f'(x_0)\leq 0\iff f'(x_0)=0\qedw$$
% ---
Для локального минимума доказательство аналогично.$\qedb$
% ---
\begin{theorem}
{\bold Условие.} Критическая точка --- {\ital экстремум}, если в~ней первая производная {\ital меняет знак}.
\end{theorem}
% ---
{\bold Доказательство.} По определению критической точки:
% ---
$$\begin{sqcases*}
&f'(x_0)=0\\
&f'(x_0)=\text{undefined}
\end{sqcases*}$$
% ---
Допустим для определённости:
% ---
$$\begin{cases}
\exists\delta\greater 0\colon\forall x\in\overset{\circ}{U}_{\delta-}(x_0)\ f'(x)\greater 
0\\
\exists\delta\greater 0\colon\forall x\in\overset{\circ}{U}_{\delta+}(x_0)\ f'(x)\less 0
\end{cases}$$
% ---
По промежуткам монотонности:
% ---
$$\begin{cases}
f\kern-4pt\uparrow\text{на }U_{\delta-}(x_0)\\
f\kern-4pt\downarrow\text{на }U_{\delta+}(x_0)
\end{cases}\hspace*{-12pt}\iff x_0\text{ --- локальный максимум}\qedw$$
% ---
Для локального минимума доказательство аналогично.$\qedb$

\begin{theorem}
{\bold Условие.} Критическая точка --- {\ital экстремум}, если в ней вторая производная {\ital ненулевая}, причём:
% ---
$$\begin{aligned}
&f''(x_0)\less 0\implies x_0\text{ --- локальный максимум}\\
&f''(x_0)\greater 0\implies x_0\text{ --- локальный минимум}
\end{aligned}$$
\end{theorem}

\subsection{Выпуклость}

Кривая $f$ {\bold выпукла вверх} в точке $M$, если в окрестности точки $f$ лежит {\ital ниже} своей касательной в этой точке.

Кривая $f$ {\bold выпукла вниз} в точке $M$, если в окрестности точки $f$ лежит {\ital выше} своей касательной в этой точке.

Кривая $f$ {\bold выпукла} на интервале $(a;b)$, если она выпукла в~{\ital каждой} её точке.

\begin{theorem}
{\bold Условие.} Пусть $f$ дважды дифференцируема на $(a;b)$:
% ---
$$\begin{aligned}
&\forall x\in(a;b)\ f''(x)\leq 0\implies f\text{ выпукла вверх}\\
&\forall x\in(a;b)\ f''(x)\geq 0\implies f\text{ выпукла вниз}
\end{aligned}$$
\end{theorem}

В {\bold точке перегиба} происходит смена {\ital характера выпуклости} функции.

\begin{theorem}
Точка перегиба --- {\ital критическая точка} второго порядка. {\ital\color{desc} (не наоборот)}
\end{theorem}

\begin{theorem}
{\bold Условие.} Критическая точка --- {\ital точка перегиба}, если в~ней вторая производная {\ital меняет знак}.
\end{theorem}

\newpage
\subsection{Теоремы о среднем}

\begin{theorem}
{\bold Теорема Ролля.} Пусть $f\in\mathbb{C}[a,b]$, дифференцируема на $(a,b)$:\labeltext{Теорема Ролля}{sec:rolletheorem}
% ---
$$f(a)=f(b)\implies\exists\xi\in(a;b)\colon f'(\xi)=0$$
\end{theorem}
% ---
{\bold Доказательство.} По \nameref{sec:weierstrasscontinuity} для непрерывной функции на компакте:
% ---
$$f(m)=\inf f([a;b])\quad\quad f(M)=\sup f([a;b])$$
% ---
По условию существования экстремума:
% ---
$$f(a)=f(b)=f(m)\implies f'(M)=0\qedw$$
% ---
При $f(m)=f(M)$ функция --- константа на $[a;b]$, произ"=водная которой равна нулю.$\qedb$

\begin{theorem}
{\bold Теорема Лагранжа.} Пусть $f\in\mathbb{C}[a,b]$, дифференцируема на $(a,b)$:\labeltext{Теорема Лагранжа}{sec:lagrangetheorem}
% ---
$$\exists\xi\in(a;b)\colon f'(\xi)=\frac{f(b)-f(a)}{b-a}$$
\end{theorem}
% ---
{\bold Доказательство.} Пусть $\varphi(x):=f(x)-\lambda x$.

Подберём $\lambda$ так, чтобы $\varphi(a)=\varphi(b)$:
% ---
\begin{gather*}
f(a)-\lambda a=f(b)-\lambda b\iff (b-a)\lambda=f(b)-f(a)\iff\\
\iff\lambda=\frac{f(b)-f(a)}{b-a}
\end{gather*}
% ---
По \ref{sec:rolletheorem}:
% ---
\begin{gather*}
\exists\xi\in(a;b)\colon\varphi'(\xi)=0\iff f'(\xi)-\lambda=0\iff\\
\iff\lambda=f'(\xi)\implies f'(\xi)=\frac{f(b)-f(a)}{b-a}\qedb
\end{gather*}

\subsection{Постоянство функции}

\begin{theorem}
Пусть $f\in\mathbb{C}[a;b]$ и дифференцируема на $(a,b)$:
% ---
$$f'((a,b))=0\iff f([a;b])=C$$
\end{theorem}
% ---
{\bold Доказательство.} По \ref{sec:lagrangetheorem}:
% ---
$$\forall x',x''\in[a;b]\ \exists\xi\in(x';x'')\colon f'(\xi)=\frac{f(x'')-f(x')}{x''-x'}
$$
% ---
По определению стационарной точки:
% ---
$$f'(\xi)=0\implies \frac{f(x'')-f(x')}{x''-x'}=0\iff f(x'')=f(x')\qedb$$
% ---
Пусть $f,g\in\mathbb{C}[a;b]$ и $f'=g'$. Тогда:
% ---
$$\forall x\in[a;b]\ f(x)-g(x)=C$$
% ---
{\bold Доказательство.} Пусть $\varphi:=f-g$; по условию:
% ---
$$\forall x\in(a;b)\ \varphi'(x)=f'(x)-g'(x)=0$$
% ---
По условию постоянства функции:
% ---
$$\varphi'(x)=0\iff\varphi(x)=C\iff f(x)-g(x)=C\qedb$$

\subsection{Частные производные}

{\bold Функция нескольких переменных} $x_1,\dots,x_n$ задана соответствием вида:
% ---
$$f\colon\mathbb{R}^n\xrightarrow{x_1,\dots,x_n\to y}\mathbb{R}$$
% ---
{\bold Частная производная} функции $f(x_1,\dots,x_n)$ по $x_i$ --- производная $f$ с переменной $x_i$ и др. {\ital фикс. аргументами}:
% ---
$$\frac{\partial f}{\partial x_i}\text{ --- нотация}$$
% ---
{\bold Вторая} частная производная по $x$:
% ---
$$\frac{\partial^2f}{\partial x^2}\text{ --- нотация}$$
% ---
{\bold Смешанная} частная производная по $x$, $y$:
% ---
$$\frac{\partial f}{\partial x\partial y}\text{ --- нотация}$$
% ---
\begin{theorem}
{\bold Условие.} Критическая точка $M$ --- {\ital экстремум} $f(x,y)$, если верно:
% ---
$$\begin{gathered}
B^2-AC\less 0\\
A=\left.\frac{\partial^2f}{\partial x^2}\right|_M\quad B=\left.\frac{\partial^2f}{\partial x\partial y}\right|_M\quad C=\left.\frac{\partial^2f}{\partial y^2}\right|_M
\end{gathered}$$
% ---
Причём:
% ---
$$\begin{aligned}
&A\less 0\implies M\text{ --- локальный максимум}\\
&A\greater 0\implies M\text{ --- локальный минимум}\\
\end{aligned}$$
\end{theorem}

\subsection{Неявная производная}

{\bold Неявной} называется функция...

\newpage
\subsection{Устойчивость}

Пусть дано дифференциальное уравнение:
% ---
$$\frac{\diff x}{\diff t}=f(x),\ x\in\mathbb{R}^n$$
% ---
{\bold Нейтрально устойчивым} называется такое {\ital стационарное} решение $\tilde{\varphi}(t)$ дифференциального уравнения, что:
% ---
$$\forall\varepsilon\greater 0\ \exists\delta\greater 0\colon\forall x_t\in U_\delta(x^*)\ x_t\in U_\varepsilon(x^*)$$
$$\forall\varepsilon\greater 0\ \exists\delta\greater 0\colon\forall x_t\in U_\delta(x^*)\ x_t\in U_\varepsilon(x^*)$$
% ---
\begin{theorem}
Малые отклонения от решения не выводят систему из~окрестности стационарного решения.
\end{theorem}
% ---
{\bold Асимптотически устойчивым} называется такое {\ital нейтрально устойчивое} решение $x^*$ дифференциального уравнения, что:
% ---
$$t\to+\infty,\ \abs{x_t-x^*}\to 0$$
% ---
\begin{theorem}
Малые отклонения от нейтрально устойчивого решения со временем затухают.
\end{theorem}
% ---
Решения дифференциального уравнения делятся на:
% ---
\begin{list*}
\item{\bold притягивающие} --- асимптотически устойчивые;
\item{\bold отталкивающие} --- неустойчивые.
\end{list*}
% ---
\begin{theorem}
{\bold Линеаризация Ляпунова.} Устойчивость стационарного состояния уравнения определяется знаком производной правой части в стационарной точке:
% ---
$$\frac{\diff x}{\diff t}=f(x)$$
\end{theorem}
