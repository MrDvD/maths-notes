\section{Дифференциальное исчисление}

% ОПРЕДЕЛЕНИЕ МОНОТОННОСТИ ОКОЛО ТОЧКИ

\subsection{Дифференцируемость}

{\bold Дифференцируемой} {\ital{\color{desc}(«линейной в малом»)} в точке $x_0$} называется такая функция $f$, для которой справедливо:
% ---
$$\Delta f=(k+\underset{\Delta x\to 0}{\alpha(x)})\Delta x,\ \alpha\text{ --- б.м.}$$
% ---
{\ital Односторонняя дифференцируемость} в точке $x_0$ опреде"=ляется через односторонние пределы.

{\bold Дифференциал} {\ital функции} $f$ --- линейная часть $\Delta f$:
% ---
$$k\Delta x=:\diff f$$
% ---
{\bold Производная} {\ital в точке $x_0$} --- предел вида: {\ital\color{desc} (Ж.Л. Лагранж)}
% ---
$$k=\lim_{\Delta x\to 0}\frac{\Delta f}{\Delta x}=:f'(x_0)$$ 

\subsection{Свойства}

Таблица {\ital «дистрибуции»} производной:\par
% ---
\begin{tabularcx}{0pt}{0pt}{C @{\hspace*{-40pt}} C}{\textwidth}
{$\begin{aligned}
&(f+g)'=f'+g'\\
&(f\cdot g)'=f'g+fg'\\
&(f\circ g)'=(f'\circ g)g'
\end{aligned}$} &
{$\begin{aligned}
\left(\frac{f}{g}\right)'&=\frac{f'g-fg'}{g^2}\\
(kf)'&=kf',\ k=\text{const}
\end{aligned}$}
\end{tabularcx}

\begin{theorem}
Дифференцируемость $\implies$ непрерывность.
\end{theorem}

{\bold Доказательство.} По определению производной:
% ---
$$\lim_{\Delta x\to 0}\frac{\Delta f}{\Delta x}=f'(x_0)\iff \frac{\Delta f}{\Delta x}=f'(x_0)+\underset{\Delta x\to 0}{\alpha(x)}\Delta x\iff$$
$$\Delta f=\Delta x(f'(x_0)+\alpha(x)\Delta x)\implies\Delta x\to 0,\ \Delta f\to 0\qedb$$

\begin{theorem}
{\bold Производная обратной функции.} Пусть $y=f(x)$ --- дифференцируемая функция. Тогда справедливо:
% ---
$$f^{-1'}(y)=\frac{1}{f'(x)},\ f'(x)\neq 0$$
\end{theorem}

{\bold Доказательство.} По условию запишем тождество:
% ---
$$\frac{\Delta f}{\Delta x}=1\colon\frac{\Delta x}{\Delta f}$$
% ---
По предельному переходу и непрерывности функций:
% ---
$$\lim_{\Delta x\to 0}\frac{\Delta f}{\Delta x}=1\colon\lim_{\Delta f\to 0}\frac{\Delta x}{\Delta f}\overset{\text{\tinyt опр}}{\iff} f'(x)=1\colon f^{-1'}(y)\iff$$
$$\iff f^{-1'}(y)=\frac{1}{f'(x)},\ f'(x)\neq 0\qedb$$

\subsection{Элементарные производные}

Таблица производных элементарных функций:
% ---
\begin{gather*}
\begin{aligned}
C'=0\qquad & (x^n)'=nx^{n-1},\ n\neq 0\qquad & \ln'x=1/x
\end{aligned}\\
\begin{align*}
\sin'\alpha&=\cos\alpha\hspace*{1cm} & \cos'\alpha&=-\sin\alpha\hspace*{2.05cm}\\
\tg'\alpha&=1/\cos^2\alpha & \ctg'\alpha&=-1/\sin\trsp^2\alpha\\
\arcsin'x&=1/\sqrt{1-x^2} & \arccos'x&=-1/\sqrt{1-x^2}\\
\arctg'x&=1/(1+x^2) & \arcctg'x&=-1/(1+x^2)
\end{align*}
\end{gather*}

\subsection{Касательная}

{\bold Касательная} --- прямая, которая проходит через точку $x_0$ кривой и представляет {\ital предельное} положение секущей при $x\to x_0$, или $\Delta x\to 0$.

\begin{theorem}
{\bold Геометрический смысл производной.} Угловой коэф"=фициент {\ital\color{desc}(тангенс)} касательной к графику функции $f$ равен {\ital производной} в этой точке:
% ---
$$k=\tg\alpha=f'(x_0)$$
\end{theorem}
% ---
{\bold Доказательство.} По определению касательной:
% ---
$$\lim_{x\to x_0}\frac{\Delta f}{\Delta x}=\tg\alpha=k$$
% ---
По определению производной:
% ---
$$f'(x_0)=\tg\alpha=k\qedb$$
% ---
\begin{theorem}
{\bold Уравнение касательной} к графику функции $f$ в точке $x_0$ имеет вид:
% ---
$$y-f(x_0)=f'(x_0)(x-x_0)$$
\end{theorem}
% ---
{\bold Доказательство.} По уравнению секущей графика $f$:
% ---
$$\frac{x-x_0}{x_1-x_0}=\frac{y-f(x_0)}{f(x_1)-f(x_0)}\implies y-f(x_0)=\frac{f(x_1)-f(x_0)}{x_1-x_0}(x-x_0)$$
% ---
По определению касательной:
% ---
$$y-f(x_0)=\lim_{x_1\to x_0}\frac{f(x_1)-f(x_0)}{x_1-x_0}(x-x_0)=f'(x_0)(x-x_0)\qedb$$

\subsection{Промежутки монотонности}

\begin{theorem}
Если функция $f$ дифференцируема в точке $x_0$, то
% ---
$$\begin{cases}
f'(x_0)\greater 0\implies f\kern-4pt\uparrow\text{около }x_0\\
f'(x_0)\less 0\implies f\kern-4pt\downarrow\text{около }x_0
\end{cases}\hspace*{-12pt}.$$
\end{theorem}
% ---
{\bold Доказательство.} По определению производной:
% ---
$$f'(x_0)\greater 0\iff \lim_{\Delta x\to 0}\frac{\Delta f}{\Delta x}\greater 0\iff\frac
{\Delta f}{\Delta x}\greater o(\Delta x)$$
% ---
При достаточно малом $\Delta x$ верно:
% ---
$$\frac{\Delta f}{\Delta x}\greater 0\iff\begin{sqcases}
\Delta f,\Delta x\greater 0\\
\Delta f,\Delta x\less 0
\end{sqcases}\hspace*{-12pt}\iff
f\kern-4pt\uparrow\text{около }x_0\qedw$$
% ---
Для $f'(x_0)\less 0$ доказательство аналогично.$\qedb$

\subsection{Условие существования экстремума}

\begin{theorem}
Точка локального экстремума $\implies$ критическая точка.
\end{theorem}

{\bold Доказательство.} По определению локального максимума:
% ---
$$\exists\delta\greater 0\colon\forall x\in\overset{\circ}{U}_\delta(x_0)\ f(x_0)\greater 
f(x)$$
% ---
Производная в точке $x_0$ либо существует, либо нет.$\qedw$

Допустим, она существует; по определению производной:
% ---
$$\lim_{x\to x_0}\frac{\Delta f}{\Delta x}=f'(x_0)$$
% ---
По предельному переходу:
% ---
$$\begin{sqcases}
\Delta x\greater 0\implies\Delta f/\Delta x\less 0\implies f'(x_0)\leq 0\\
\Delta x\less 0\implies\Delta f/\Delta x\greater 0\implies f'(x_0)\geq 0
\end{sqcases}\hspace*{-12pt}\iff$$
$$0\leq f'(x_0)\leq 0\iff f'(x_0)=0\qedw$$
% ---
Для локального минимума доказательство аналогично.$\qedb$
% ---
\begin{theorem}
Если в критической точке производная меняет знак, она является локальным экстремумом.
\end{theorem}
% ---
{\bold Доказательство.} По определению критической точки:
% ---
$$\begin{sqcases}
f'(x_0)=0\\
f'(x_0)=\text{undefined}
\end{sqcases}$$
% ---
Допустим для определённости:
% ---
$$\begin{cases}
\exists\delta\greater 0\colon\forall x\in\overset{\circ}{U}_{\delta-}(x_0)\ f'(x)\greater 
0\\
\exists\delta\greater 0\colon\forall x\in\overset{\circ}{U}_{\delta+}(x_0)\ f'(x)\less 0
\end{cases}$$
% ---
По промежуткам монотонности:
% ---
$$\begin{cases}
f\kern-4pt\uparrow\text{на }U_{\delta-}(x_0)\\
f\kern-4pt\downarrow\text{на }U_{\delta+}(x_0)
\end{cases}\hspace*{-12pt}\iff x_0\text{ --- локальный максимум}\qedw$$
% ---
Для локального минимума доказательство аналогично.$\qedb$

\subsection{Теорема Ролля}

\begin{theorem}
Пусть $f$ дифференцируема на $[a;b]$. Тогда:
% ---
$$f(a)=f(b)\implies\exists\xi\in(a;b)\colon f'(\xi)=0$$
\end{theorem}
% ---
{\bold Доказательство.} По теореме Вейерштрасса:
% ---
$$f(m)=\inf f([a;b])\quad\quad f(M)=\sup f([a;b])$$
% ---
По условию существования экстремума:
% ---
$$f(a)=f(b)=f(m)\implies f'(M)=0\qedw$$
% ---
При $f(m)=f(M)$ функция --- константа на $[a;b]$, произ"=водная которой равна нулю.$\qedb$

\subsection{Теорема Лагранжа}

\begin{theorem}
Пусть $f$ дифференцируема на $[a;b]$. Тогда верно:
% ---
$$\exists\xi\in(a;b)\colon f'(\xi)=\frac{\Delta f}{\Delta x}$$
\end{theorem}
% ---
{\bold Доказательство.} Пусть $\varphi(x):=f(x)-\lambda x$.

Подберём $\lambda$ так, чтобы $\varphi(a)=\varphi(b)$:
% ---
\begin{gather*}
f(a)-\lambda a=f(b)-\lambda b\iff (b-a)\lambda=f(b)-f(a)\iff\\
\iff\lambda=\frac{f(b)-f(a)}{b-a}
\end{gather*}
% ---
По теореме Ролля:
% ---
\begin{gather*}
\exists\xi\in(a;b)\colon\varphi'(\xi)=0\iff f'(\xi)-\lambda=0\iff\\
\iff\lambda=f'(\xi)\implies f'(\xi)=\frac{f(b)-f(a)}{b-a}=\frac{\Delta f}{\Delta x}\qedb
\end{gather*}

\subsection{Условие постоянства функции}

\begin{theorem}
Пусть $f$ непрерывна на $[a;b]$ и состоит из стационарных точек на $(a;b)$. Тогда 
$f([a;b])=C$.
\end{theorem}
% ---
{\bold Доказательство.} По теореме Лагранжа:
% ---
$$\forall x',x''\in[a;b]\ \exists\xi\in(x';x'')\colon f'(\xi)=\frac{f(x'')-f(x')}{x''-x'}
$$
% ---
По определению стационарной точки:
% ---
$$f'(\xi)=0\implies \frac{f(x'')-f(x')}{x''-x'}=0\iff f(x'')=f(x')\qedb$$
% ---
Пусть $f,g\in\mathbb{C}[a;b]$ и $f'=g'$. Тогда:
% ---
$$\forall x\in[a;b]\ f(x)-g(x)=C$$
% ---
{\bold Доказательство.} Пусть $\varphi:=f-g$; по условию:
% ---
$$\forall x\in(a;b)\ \varphi'(x)=f'(x)-g'(x)=0$$
% ---
По условию постоянства функции:
% ---
$$\varphi'(x)=0\iff\varphi(x)=C\iff f(x)-g(x)=C\qedb$$
