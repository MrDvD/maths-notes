\section{Алгоритмы на графах}

\subsection{Обход в глубину}

{\bold Обход в глубину} {\ital (Depth-First Search, DFS)} --- метод, при котором граф обходят сначала по детям, потом по сиблингам.

{\bold Алгоритмы:}
% ---
\begin{list*}
\item поиск эйлерового пути, цикла
\item тест ацикличности
\item поиск мостов, точек сочленения
\item топологическая сортировка
\item построение компонент связности:
\begin{list*}[2]
\item обыкновенных {\ital\color{desc} (грядки, водостоки)}
\item сильных {\ital\color{desc} (алгоритм Косараджу, конденсация)}
\end{list*}
\item 2-SAT
\item алгоритм Куна
\end{list*}

\begin{theorem}
{\bold Маркировка.} Когда нужно найти {\ital циклы}, вершины маркируются в зависимости от типа графа: 
% ---
\begin{list*}
\item неорграф $\implies$ {\ital бинарные} метки
\item орграф $\implies$ {\ital тернарные} метки
\end{list*}
\end{theorem}

\subsection{Обход в ширину}

{\bold Обход в ширину} {\ital (Breadth-First Search, BFS)} --- метод, при котором граф обходят сначала по сиблингам, потом по детям.

{\bold Алгоритмы:}
% ---
\begin{list*}
\item алгоритм Кана
\item проверка графа на двудольность
\item поиск кратчайших рёберных путей
\end{list*}

Нет циклов $\implies$ не требуется массив {\ital visited}.

Также были задачи на потоки. Рассмотреть их!

\subsection{Поиск эйлерового пути}

{\bold Алгоритм} поиска эйлерового {\ital цикла}:

\begin{code}{python}
if is_eulerian(graph): #1
  ecirc = graph.get_ecirc(v) #2
else:
  ecirc = -1

def get_ecirc(v): # DFS
  ecirc = list()
  for n in edges[v]:
    if edges[v][n] > 0:
      edges[v][n] -= 1
      ecirc += get_ecirc(n)
  return ecirc + [v] #3
\end{code}

\begin{list*}[][\#]
\item Проверить граф на {\ital эйлеровость}.
\item Начать с любой вершины $v$.
\item Если дан орграф, ответ нужно {\ital инвертировать}.
\end{list*}

{\bold Алгоритм} поиска эйлерового {\ital пути}:

\begin{code}{python}
if is_semi_eulerian(graph): #1
  epath = graph.get_epath(v) #2
else:
  epath = -1

def get_epath(v): # DFS
  epath = list()
  for n in edges[v]:
    if edges[v][n] > 0:
      edges[v][n] -= 1
      epath += get_epath(n)
  return epath + [v] #3
\end{code}

\begin{list*}[][\#]
\item Проверить граф на {\ital полуэйлеровость}.
\item Начать с вершины нечётной степени $v$.

{\bold НО:} если дан орграф, выбрать вершину большей {\ital исхо"=дящей} валентности.
\item Если дан орграф, ответ нужно {\ital инвертировать}.
\end{list*}

Применяется в {\bold задачах}:

\begin{list*}
\item китайский почтальон;
\item домино, восстановление строки.
\end{list*}

\subsection{Топологическая сортировка}

{\bold Топологическая сортировка} --- упорядочивание вершин ориентированного леса согласно {\ital частичному порядку}, который задан рёбрами орграфа. 

{\bold Алгоритм Тарьяна} --- реализация через {\ital DFS}:

\begin{code}{python}
def topo_sort(v): # DFS
  vertices[v].visited = 1
  stack = list()
  for n in vertices[v].adjacent:
    if not vertices[n].visited:
      stack += topo_sort(n)
  return stack + [v]
\end{code}

{\bold Алгоритм Кана} --- реализация через {\ital BFS}:

\begin{code}{python}
def topo_sort(queue=deque()): # BFS
  for v in vertices: #1
    if vertices[v].indegree == 0:
      queue.append(v)
  order = list()
  while queue:
    q = queue.popleft()
    order.append(q)
    for n in vertices[q].adjacent_out:
      vertices[n].indegree -= 1
      if vertices[n].indegree == 0: #2
        queue.append(n)
  return order 
\end{code}

\begin{list*}[][\#]
\item Найти {\ital корни} графа, добавить их в очередь.
\item Если какая либо из соседних вершин стала {\ital корнем}.
\end{list*}

Применяется в {\bold задачах}:

\begin{list*}
\item лексикографическая сортировка;
\item топологическое маркирование;
\item тест ацикличности.
\end{list*}

\begin{theorem}
{\bold Задача.} В игре есть $N$ уровней, соединённых $M$ телепортами. Сколько есть способов добраться от первого уровня до $N$-го? {\ital\color{desc}(телепорты не образуют циклов)}
\end{theorem}

{\bold Идея.} Представим уровни как вершины, а телепорты как рёбра графа.

Пусть каждая вершина маркирована числом путей, исходящих от первой вершины {\ital\color{desc}(оно не всегда равно входящей валентности)}.

Запустим алгоритм Кана из первой вершины, и при обработке ребёнка очередной вершины будем добавлять к его маркировке родителькую.

Ответ к задаче --- значение маркировки вершины $N$.

\begin{theorem}
{\bold Задача.} Требуется выполнить $N$ курсов. Есть $M$ требований вида {\ital «курс $a$ должен быть выполнен до курса $b$»}. Составить, если возможно, порядок прохождения курсов. 
\end{theorem}

{\bold Идея.} Представим курсы как вершины, а требования как рёбра графа.

Проверим, что граф {\ital ацикличен}: иначе решение составить невозможно.

Запустим алгоритм Кана из корневой вершины --- полученная последовательность удовлетворяет условию.

\begin{theorem}
{\bold Задача.} Леви нужно добраться от города $1$ до $N$, но он хочет это сделать через наибольшее количество промежуточных городов. Составить, если возможно, такой маршрут.
\end{theorem}

{\bold Идея.} Представим города как вершины, а рейсы как рёбра графа.

Проверим, что граф {\ital ацикличен}: иначе решение составить невозможно.

Пусть каждая вершина маркирована максимальным путём, исходящих от первой вершины.

Запустим алгоритм Кана из первой вершины, и при обработке ребёнка очередной вершины будем добавлять к его маркировке максимальный путь из родительских с инкрементом.

Ответ к задаче --- значение маркировки вершины $N$.

\subsection{Алгоритм Косараджу}

{\bold Алгоритм} поиска компонент сильной связности:

\begin{code}{python}
reverse = graph.transpose() #1
order = reverse.topo_sort() #2
mark = 0
for v in order[::-1]: #3
  if not graph.vertices[v].visited:
    mark_component(v, mark)
    mark += 1

def mark_component(v, mark): # DFS
  vertices[v].visited = 1
  vertices[v].mark = mark
  for n in vertices[v].adjacent:
    if not vertices[n].visited:
      mark_component(n, mark)
\end{code}

\begin{list*}[][\#]
\item Построить {\ital транспонированный} граф {\ital reverse}.
\item Применить {\ital топологическую сортировку} к {\ital reverse}.
\item Применить {\ital DFS} на {\ital graph} в обратном порядке топологи"=ческой сортировки:
% ---
$$\text{цикл поиска в глубину}\equiv\text{сильная компонента связности}$$
\end{list*}

\subsection{2-SAT}

{\bold Алгоритм} решения:

\begin{list*}[][\#]
\item Перевести CNF в INF.
\item Построить граф импликаций.
\item Найти компоненты сильной связности в графе.
\item Проверить, что для любой вершины $x$ графа справедливо:
$$c[x]\neq c[\lnot x]$$
\item Если требуется вывести ответ, то воспользоваться формулой:
$$x=\left\{\begin{aligned}
&\text{true}, &c[x]\less c[\lnot x]\\
&\text{false}, &c[x]\greater c[\lnot x]
\end{aligned}\right.$$
\end{list*}

\subsection{Поиск мостов}

{\bold Алгоритм} поиска мостов на неорграфе:

\begin{code}{python}
def get_bridges(v, parent=-1): # DFS
  vertices[v].visited = 1
  time += 1
  vertices[v].ord = time #1
  vertices[v].min = time
  bridges = set()
  for n in vertices.adjacent:
    if n == parent: #2
      continue
    if vertices[n].visited: #3
      vertices[v].min = min(vertices[v].min, \
                            vertices[n].ord)
    else: #4
      bridges |= get_bridges(n, v) #5
      vertices[v].min = min(vertices[v].min, \
                            vertices[n].min)
      if vertices[n].min > vertices[v].ord and \
         parent != -1: #6
           bridges.add((v, n))
  return bridges
\end{code}

\begin{list*}[][\#]
\item Запись текущего порядка захода в вершину.
\item Если $\langle v,n\rangle$ --- ребро в {\ital обратную сторону}.
\item Если $\langle v,n\rangle$ --- {\ital обратное ребро}.
\item Если $\langle v,n\rangle$ --- {\ital ребро дерева}.
\item Мост может быть отмечен {\ital несколько раз} за алгоритм.
\item Критерий моста:
% ---
$$[\langle v,neigh\rangle\text{ --- мост}]=\left\{\begin{aligned}
&\text{true}, &\text{min}[neigh]\greater\text{ord}[v]\\
&\text{false}, &\text{min}[neigh]\leq\text{ord}[v]
\end{aligned}\right.$$
\end{list*}

\subsection{Поиск точек сочленения}

{\bold Алгоритм} поиска точек сочленения на неорграфе:

\begin{code}{python}
def get_cuts(v, parent=-1): # DFS
  vertices[v].visited = 1
  time += 1
  vertices[v].ord = time #1
  vertices[v].min = time
  children, cuts = 0, set()
  for n in vertices.adjacent:
    if n == parent: #2
      continue
    if vertices[n].visited: #3
      vertices[v].min = min(vertices[v].min, \
                            vertices[n].ord)
    else: #4
      children += 1 
      cuts |= get_cuts(n, v) #5
      vertices[v].min = min(vertices[v].min, \
                            vertices[n].min)
      if vertices[n].min >= vertices[v].ord and \
         parent != -1: #6
           cuts.add(v)
  if children > 1 and parent == -1:
    cuts.add(v)
  return cuts
\end{code}

\begin{list*}[][\#]
\item Запись текущего порядка захода в вершину.
\item Если $\langle v,n\rangle$ --- ребро в {\ital обратную сторону}.
\item Если $\langle v,n\rangle$ --- {\ital обратное ребро}.
\item Если $\langle v,n\rangle$ --- {\ital ребро дерева}.
\item Вершина может быть отмечена {\ital несколько раз} за алгоритм.
\item Критерий точки сочленения:
% ---
$$[v\text{ --- т. сочленения}]=\left\{\begin{aligned}
&\text{true}, &\text{min}[neigh]\geq\text{ord}[v]\\
&\text{false}, &\text{min}[neigh]\less\text{ord}[v]
\end{aligned}\right.$$
\end{list*}

\subsection{Алгоритм Куна}

{\bold Алгоритм} поиска максимального паросочетания на двудольном графе:

\begin{code}{python}
for v in vertices.part: #1
  if has_ichain(v):
    null_all_visited(vertices)

matching = dict() # empty == None
def has_ichain(v): # DFS
  if vertices[v].visited:
    return False
  vertices[v].visited = 1
  for n in vertices[v].adjacent:
    if matching[n] == None or \
       has_ichain(matching[n]): #2
         matching[n] = v
         return True
  return False
\end{code}

\begin{list*}[][\#]
\item Рассматриваются все вершины {\ital одной доли} двудольного графа.
\item Если пары нет или если есть {\ital увеличивающаяся цепь}.
\end{list*}

\subsection{Алгоритм Дейкстры}

{\bold Алгоритм Дейкстры} ищёт кратчайшие пути из заданной вершины до всех остальных вершин взвешенного графа {\ital\color{desc} (вес рёбер неотрицательный)}.

{\bold Наивная} реализация за $\mathcal{O}(n^2+m)$:

\begin{code}{python}
def dijkstra(root): # tabulation
  # base (default = INF)
  verts[root].min = 0
  while True:
    # minimization - O(N)
    v = -1
    for i in verts:
      if not verts[i].visited and \
         (v == -1 or verts[i].min < verts[v].min):
           v = i
    if v == -1: # reachable vertices are visited
      break
    verts[v].visited = 1
    # relaxation
    for n in verts[v].adjacent:
      curr_cost = verts[v].min + edges[n][v].cost
      if curr_cost < verts[n].min:
        verts[n].min = curr_cost
        verts[n].parent = v # certificate
\end{code}

{\bold Кучная} реализация за $\mathcal{O}(m\log n)$:

\begin{code}{python}
def dijksrta(root):
  # base (default = INF)
  verts[root].min = 0
  queue = [(0, root)]
  heapq.heapify(queue)
  while queue:
    # minimization - O(log N)
    dist, v = heapq.heappop(queue)
    if dist > verts[v].min:
      continue
    # relaxation
    verts[v].visited = True
    for n in verts[v].adjacent:
      if verts[n].visited:
        continue
      curr_cost = verts[v].min + edges[v][n].cost
      if curr_cost < verts[n].min:
        verts[n].min = curr_cost
        verts[n].parent = v # certificate
        heapq.heappush((verts[n].min, n))
\end{code}

\subsection{Алгоритм A*}

{\bold Алгоритм A*} ищет кратчайший путь между двумя заданными вершинами взвешенного графа на основе {\ital эвристики} {\ital\color{desc} (вес рёбер неотрицательный)}.

{\bold Кучная} реализация алгоритма за $\mathcal{O}(m\log n)$:

\begin{code}{python}
def a_star(start, goal):
  # base (default = INF)
  verts[start].min = 0
  queue = [(eur(start), start)]
  heapq.heapify(queue)
  while queue:
    # minimization - O(log N)
    heur, v = heapq.heappop(queue)
    if v == goal:
      break
    if heur != verts[v].heur:
      continue
    # relaxation
    verts[v].visited = True
    for n in verts[v].adjacent:
      if verts[n].visited:
        continue
      curr = verts[v].min + edges[v][n].cost
      if curr < verts[n].min:
         verts[n].min = curr
         verts[n].heur = verts[n].min + heur(n)
         verts[n].parent = v # certificate
         heapq.heappush(queue, (verts[n].heur, n))
\end{code}

\subsection{Кратчайший путь}

{\bold Волновой алгоритм} --- да.

{\bold Алгоритм Беллмана-Форда} --- да.

\subsection{Алгоритм Прима}

{\bold Алгоритм} поиска {\ital миностова}:

\begin{code}{python}
def prim(start): # greedy
  # base (default = INF)
  verts[start].min = 0
  while True:
    # minimization
    v = -1
    for i in verts:
      if not verts[i].visited and \
         (v == -1 or verts[i].min < v[1]):
           v = i
    if v == -1: # reachable vertices are visited
      break
    verts[v].visited = 1
    if verts[v].parent != -1: # got safe edge
      add_safe_edge(v, verts[v].parent)
    # relaxation
    for n in verts[v].adjacent:
      curr_cost = edges[v][n].cost
      if curr_cost < verts[n].min:
        verts[n].min = curr_cost
        verts[n].parent = v # certificate
\end{code}

Структура идентична {\ital алгоритму Дейкстры}.

\subsection{Disjoint Set Union}

{\bold Система непересекающихся множеств} {\ital (СНМ, или DSU)} --- дерево, которое обладает операциями:
% ---
\begin{list*}
\item\cdesc{make\_set(v)} --- создать новое множество за $\mathcal{O}(1)$
\item\cdesc{find\_set(v)} --- найти множество элемента за $\mathcal{O}(\log n)$
\item\cdesc{union(v, u)} --- объединить два множества за $\mathcal{O}(\log n)$
\end{list*}

{\bold Создание} нового множества:
% ---
\begin{code}{python}
def make_set(v):
  prev[v] = v # tree root
  rank[v] = 0 # tree depth (~log n)
\end{code}
% ---
{\bold Определение} множества элемента:
% ---
\begin{code}{python}
def find_set(v): # path compression heurisitc
  if v == prev[v]:
    return v
  return find_set(prev[v])
\end{code}
% ---
{\bold Объединение} двух множеств:
% ---
\begin{code}{python}
def unite(v, u):
  v, u = find_set(v), find_set(u)
  if v != u:
    if rank[v] < rank[u]: # rank heurisitc
      v, u = u, v
    prev[u] = v
    if rank[v] == rank[u]:
      rank[v] += 1
\end{code}

\begin{theorem}
{\bold Задача.} Дан взвешенный неорграф $G\langle N, M\rangle$. Цена пути между вершинами --- вес его максимального ребра. Найти число пар с мин. ценой пути между ними, равной $X$.
\end{theorem}

{\bold Идея.} Добавим «рёбра» весом $\less X$ в СНМ, метрикой которого является {\ital количество вершин}.

Искомое число --- суммарная {\ital разница} числа новых и старых связей* при добавлении рёбер весом $X$ {\ital (так исключатся их внутренние связи)}.

* --- число пар всех вершин в $n$-компоненте:
% ---
$$S_n=\frac{n(n-1)}{2}$$

\subsection{Алгоритм Краскала}

{\bold Алгоритм Краскала} помогает составить {\ital миностов} на~основе {\ital DSU}.

В СНМ хранятся вершины графа, которые объединяются {\ital безопасными рёбрами} в порядке {\ital возрастания веса}.

{\bold Безопасным} называется ребро, которое соединяет {\ital разные} компоненты связности.

{\bold Алгоритм} поиска миностова:
% ---
\begin{code}{python}
def kraskal():
  span = list()
  edges.sort()
  for i in range(len(edges)):
    e = edges[i]
    if find_set(e.start) != find_set(e.end):
      union(e.start, e.end)
      span.append(i)
\end{code}

\begin{theorem}
{\bold Задача.} Дан архипелаг из $N$ островов с $M$ мостами. Стоимость постройки моста --- {\ital расстояние} между островами. Объединить архипелаг мостами за {\ital минимальную} стоимость. 
\end{theorem}

{\bold Идея.} Представим архипелаг как несвязный неорграф, который нужно сделать связным.

Сделаем граф полным, добавив все отсутствующие рёбра.

Ответ на задачу --- стоимость {\ital миностова}, который можно составить {\ital алгоритмом Краскала}.

\subsection{Дерево отрезков}

{\bold Дерево отрезков} {\ital\color{desc} (segment tree)} --- бинарное дерево, которое обладает операциями:
% ---
\begin{list*}
\item\cdesc{build(arr)} --- построить дерево на массиве {\ital arr} за $\mathcal{O}(n)$
\item\cdesc{get(l, r)} --- вернуть $f(arr[l:r])$ за $\mathcal{O}(\log n)$
\item\cdesc{get(i)} --- вернуть $f(arr[i])$ за $\mathcal{O}(\log n)$
\item\cdesc{update(i, val)} --- обновить $arr[i]$ за $\mathcal{O}(\log n)$
\item\cdesc{update(l, r, val)} --- обновить $arr[l:r]$ за $\mathcal{O}(\log n)$
\end{list*}

{\bold Требования} к операции $f$:
% ---
\begin{list*}
\item ассоциативность
\item наличие нейтрального элемента $\perp$
\end{list*}
% ---
{\bold Построение} дерева {\ital «сверху»}:
% ---
\begin{code}{python}
def build(root, tl, tr):
  if tl + 1 == tr: # we use [tl; tr)
    verts[root].value = arr[tl]
  else:
    tm = (tl + tr) // 2
    build(2 * root, tl, tm)
    build(2 * root + 1, tm, tr)
    verts[root].value = f(verts[2 * v].value, \
                          verts[2 * v + 1].value)
\end{code}

{\bold Получение} значения функции на отрезке:
% ---
\begin{code}{python}
def get(root, tl, tr, ql, qr):
  if $[t_l;t_r)\cap[q_l;q_r)=\emptyset$: # don't go further
    return $\perp$
  if $[t_l;t_r)\subseteq[q_l;q_r)$: # subtree is covered
    return verts[root].value
  tm = (tl + tr) // 2
  return f(get(2 * root, tl, tm, ql, qr), \
           get(2 * root + 1, tm, tr, ql, qr))
\end{code}

{\bold Обновление} элемента массива:
% ---
\begin{code}{python}
def update(root, tl, tr, i, val):
  if tl + 1 == tr:
    verts[root] = val
    return
  tm = (tl + tr) // 2
  if i < tm:
    update(2 * root, tl, tm, i, val)
  else:
    update(2 * root + 1, tm, tr, i, val)
  verts[root].value = f(verts[2 * root].value, \
                        verts[2 * root + 1].value)
\end{code}

\subsection{Несогласованные поддеревья}

{\bold Отложенной} ({\bold массовой}) называется операция, которая применяется к {\ital подотрезку} массива в дереве отрезков.

{\bold Несогласованным} называется поддерево, которое хранит в~своих вершинах {\ital частичный} результат выполнения отложенной операции.

{\bold Проталкивание} массовой операции $g$:
% ---
\begin{code}{python}
def push(root, tl, tr):
  if tl + 1 != tr:
    verts[2 * root] = g(verts[2 * root], \
                        verts[root].diff)
    verts[2 * root + 1] = g(verts[2 * root + 1], \
                            verts[root].diff)
  verts[root].diff = 0
\end{code}

{\bold Обновление} отрезка массива:
% ---
\begin{code}{python}
def update(root, tl, tr, ql, qr, val):
  push(root, tl, tr) # ???
  if $[t_l;t_r)\cap[q_l;q_r)=\emptyset$: # don't go further
    return $\perp$
  if $[t_l;t_r)\subseteq[q_l;q_r)$: # subtree is covered
    verts[root].diff = g(verts[root].diff, val)
    return
  push(root, tl, tr) # ???
  tm = (tl + tr) // 2
  update(2 * root, tl, tm, ql, qr)
  update(2 * root + 1, tm, tr, ql, qr)
  verts[root].val = f(g(verts[2 * root].val, \
                        verts[2 * root].diff), \
                      g(verts[2 * root + 1].val, \
                        verts[2 * root + 1].diff))
\end{code}

{\bold Получение} значения функции на отрезке:
% ---
\begin{code}{python}
def get(root, tl, tr, ql, qr):
  if qr <= tl or tr <= ql: # don't go further
    return $\perp$
  if ql <= tl and tr <= qr: # subtree is covered
    return g(verts[root].val, verts[root].diff)
  push(root, tl, tr)
  ans = f(get(2 * root, tl, tm, ql, qr), \
          get(2 * root + 1, tm, tr, ql, qr))
  verts[root].val = f(g(verts[2 * root].val, \
                        verts[2 * root].diff), \
                      g(verts[2 * root + 1].val, \
                        verts[2 * root + 1].diff))
  return ans
\end{code}

\subsection{Корневая декомпозиция}

Да.
