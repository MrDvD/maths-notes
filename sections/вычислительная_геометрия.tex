\section{Вычислительная геометрия}

\subsection{Деление отрезка в отношении}

Точка $C$ делит отрезок $AB$ в отношении $\lambda\in\mathbb{R}$, если:
% ---
$$\begin{cases}
C\in AB\\
\overrightarrow{AC}=\lambda\overrightarrow{CB}\\
\lambda\neq -1
\end{cases}$$
% ---
\begin{theorem}
{\bold Теорема.} Пусть $C$ делит $AB$ в отношении $\lambda\in\mathbb{R}$. Тогда координаты точки $C$ равны:
% ---
$$x_C=\frac{x_A+\lambda x_B}{1+\lambda}\qquad y_C=\frac{y_A+\lambda y_B}{1+\lambda}$$ 
\end{theorem}

{\bold Доказательство.} По условию:
% ---
$$\overrightarrow{AC}=\lambda\overrightarrow{CB}\iff\overrightarrow{OC}-\overrightarrow{OA}=\lambda(\overrightarrow{OB}-\overrightarrow{OC})$$
% ---
По теореме Фалеса:
% ---
$$\left\{\begin{aligned}
&x_C-x_A=\lambda(x_B-x_C)\\
&y_C-y_A=\lambda(y_B-y_C)
\end{aligned}\right.\iff
\left\{\begin{aligned}
&x_C=\frac{x_A+\lambda x_B}{1+\lambda}\\
&y_C=\frac{y_A+\lambda y_B}{1+\lambda}
\end{aligned}\right.\qedb$$

\subsection{Направляющий вектор}

{\bold Направляющим} называется вектор, параллельный данной прямой:
% ---
$$\vec{l}=\begin{pmatrix}
A\\B\end{pmatrix}\implies\begin{cases*}
&x=x_0+At\\
&y=y_0+Bt
\end{cases*}$$
% ---
\begin{theorem}
{\bold Теорема.} Уравнение прямой по точке и направляющему вектору:
% ---
$$\frac{A}{B}=\frac{x-x_0}{y-y_0}$$
\end{theorem}
% ---
Уравнение легко выводится из определения направляющего вектора.

\begin{theorem}
{\bold Свойство.} Связь угла между прямыми и их направляющими векторами:
% ---
$$\cos\angle(a,b)=\abs{\cos\angle(\overrightarrow{a},\overrightarrow{b})}$$
\end{theorem}

\subsection{Коллинеарность}

{\bold Коллинеарными} называются:
% ---
\begin{tabularcx}{3pt}{3pt}{@{--- } L}{\textwidth}
{\ital точки}, которые лежат на одной прямой;\\
{\ital векторы}, которые лежат на одной прямой или на~параллельных прямых.
\end{tabularcx}
% ---
\begin{theorem}
{\bold Критерий коллинеарности} двух векторов:
% ---
$$\overrightarrow{a}=\lambda\overrightarrow{b},\ \lambda\in\mathbb{R}\iff\begin{cases}
x_a=\lambda x_b\\
y_a=\lambda y_b
\end{cases}$$ 
\end{theorem}
% ---
В частности, нулевой вектор коллинеарен {\ital любому} вектору:
% ---
$$\overrightarrow{0}=0\overrightarrow{a}$$
% ---
\begin{theorem}
{\bold Уравнение секущей} по двум известным точкам:
% ---
$$A\langle x_a,y_a\rangle,\ B\langle x_b,y_b\rangle\implies\frac{x-x_a}{x_b-x_a}=\frac{y-y_a}{y_b-y_a}$$
\end{theorem}
% ---
{\bold Доказательство.} Пусть $\overrightarrow{AX},\ \overrightarrow{AB}$ --- коллинеарные векторы.

По критерию коллинеарности двух векторов:
% ---
$$\left\{\begin{aligned}
x-x_a=\lambda(x_b-x_a)\\
y-y_a=\lambda(y_b-y_a)
\end{aligned}\right.\iff
\lambda=\frac{x-x_a}{x_b-x_a}=\frac{y-y_a}{y_b-y_a}\qedb$$

\subsection{Скалярное произведение}

\begin{theorem}
{\bold Теорема.} Косинус угла $\alpha$ между векторами $\overrightarrow{a}$ и $\overrightarrow{b}$ равен:
% ---
$$\cos\alpha=\frac{x_ax_b+y_ay_b}{\abs{\overrightarrow{a}}\abs{\overrightarrow{b}}}$$
\end{theorem}
% ---
{\bold Доказательство.} Отложим векторы $\overrightarrow{AB}=\overrightarrow{a}$, $\overrightarrow{AC}=\overrightarrow{b}$ от начала координат.

По теореме косинусов:
% ---
$$\begin{gathered}
BC^2=AB^2+AC^2-2\abs{\overrightarrow{AB}}\abs{\overrightarrow{AC}}\cos\alpha\implies\\
\cos\alpha=\frac{AB^2+AC^2-BC^2}{2\abs{\overrightarrow{AB}}\abs{\overrightarrow{AC}}}
\end{gathered}$$
% ---
По теореме Пифагора:
% ---
$$\begin{gathered}
\left\{\begin{aligned}
&AB^2=x_a^2+y_a^2\\
&AC^2=x_b^2+y_b^2\\
&BC^2=(x_a-x_b)^2+(y_a-y_b)^2
\end{aligned}\right.\implies\\
AB^2+AC^2-BC^2=2(x_ax_b+y_ay_b)\implies\\
\cos\alpha=\frac{x_ax_b+y_ay_b}{\abs{\overrightarrow{a}}\abs{\overrightarrow{b}}}\qedb
\end{gathered}$$
% ---
{\bold Скалярное произведение} {\ital векторов} $\overrightarrow{a}$, $\overrightarrow{b}$ --- величина:
% ---
$$x_ax_b+y_ay_b=:\overrightarrow{a}\cdot\overrightarrow{b}$$
% ---
\begin{theorem}
{\bold Теорема.} Скалярное произведение двух векторов равно:
% ---
$$\overrightarrow{a}\cdot\overrightarrow{b}=\abs{\overrightarrow{a}}\abs{\overrightarrow{b}}\cos\alpha,$$
% ---
$\alpha$ --- угол между векторами.
\end{theorem}

\begin{theorem}
{\bold Следствие.} Если векторы $\overrightarrow{a}$ и $\overrightarrow{b}$ {\ital коллинеарны}, то:
% ---
\begin{list*}
\item$\overrightarrow{a}\cdot\overrightarrow{b}\greater 0\implies$ векторы {\ital сонаправлены}
\item$\overrightarrow{a}\cdot\overrightarrow{b}\less 0\implies$ векторы {\ital несонаправлены}
\item$\overrightarrow{a}\cdot\overrightarrow{b}=0\implies$ один из векторов {\ital нулевой}
\end{list*}
\end{theorem}

\subsection{Ориентированный угол}

{\bold Ориентированным} называется угол $\alpha$ между $\overrightarrow{a}$ и $\overrightarrow{b}$, на который нужно повернуть $\overrightarrow{a}$, чтобы он был сонаправлен с $\overrightarrow{b}$:
% ---
$$\angle(\overrightarrow{a},\overrightarrow{b})\text{ --- обозначение},\ \alpha\in\left(-\pi;\pi\right]$$
% ---
{\bold Знак} ориентированного угла:

\begin{list*}
\item{\bold пололжительный}, если поворот происходит в {\ital положи"=тельном} направлении системы координат;
\item{\bold отрицательный}, если поворот происходит в {\ital отрица"=тельном} направлении системы координат;
\item{\bold нуль}, если вектора {\ital сонаправлены}.
\end{list*}

\subsection{Косое произведение}

{\bold Косое произведение} {\ital векторов} $\overrightarrow{a}$, $\overrightarrow{b}$ --- величина:
% ---
$$x_ax_b-y_ay_b=:\overrightarrow{a}\land\overrightarrow{b}$$
% ---
\begin{theorem}
{\bold Теорема.} Косое произведение двух векторов равно:
% ---
$$\overrightarrow{a}\land\overrightarrow{b}=\abs{\overrightarrow{a}}\abs{\overrightarrow{b}}\sin\alpha,$$
% ---
$\alpha$ --- угол между векторами.
\end{theorem}

\begin{theorem}
{\bold Теорема.} Знак косого произведения векторов {\ital совпадает} со знаком ориентированного угла.
\end{theorem}

Доказательство вытекает из {\ital чётности} синуса угла между векторами.

\subsection{Взаимное расположение}

Расположение {\ital точки} $A$ относительно {\ital прямой, луча или отрезка} $BC$:

\begin{list*}
\item$\angle(\overrightarrow{BA},\overrightarrow{BC})\greater 0\implies A$ лежит в {\bold левой} полуплоскости;
\item$\angle(\overrightarrow{BA},\overrightarrow{BC})\less 0\implies A$ лежит в {\bold правой} полуплоскости;
\item$\angle(\overrightarrow{BA},\overrightarrow{BC})=0\implies A$ {\bold коллинеарна} прямой $BC$.
\end{list*}

Взаимное расположение {\ital двух отрезков или лучей} $AB$ и $CD$:

\begin{list*}
\item концы обоих отрезков лежат в {\ital разных} полуплоскостях относительно друг друга $\implies$ отрезки {\bold пересекаются};
\item концы одного отрезка лежат в {\ital одной} полуплоскости относительно другого $\implies$ отрезки {\bold не пересекаются};
\item концы одного отрезка лежат {\ital на} прямой другого отрезка:
\begin{list*}[2]
\item конец одного отрезка лежит {\ital на} другом $\implies$ отрезки имеют {\bold общий подотрезок};
\item концы одного отрезка {\ital не} лежат на другом $\implies$ отрезки {\bold не пересекаются}.
\end{list*}
\end{list*}

\subsection{Ориентированная площадь}

{\bold Ориентированной} называется площадь многоугольника, которая обладает знаком его ориентированных углов.

\begin{theorem}
{\bold Теорема.} Ориентированная площадь треугольника равна половине {\ital косого произведения} векторов ориентированного угла.
\end{theorem}

Ориентированная площадь --- {\ital аддитивная} величина, к~основным методам её расчёта относят:

\begin{list*}
\item метод трапеций;
\item метод треугольников.
\end{list*}

\subsection{Метод трассировки луча}

\begin{theorem}
{\bold Задача.} На плоскости даны многоугольник и точка. Решить вопрос о принадлежности точки многоугольнику.
\end{theorem}

{\bold Алгоритм} трассировки луча:

\begin{list*}[][\#]
\item Проверить принадлежность точки стороне многоуголь"=ника: если {\ital истина}, остановить алгоритм.
\item Выпустить из точки в случайном направлении луч.
\item Посчитать число $n$ пересечений луча со сторонами многоугольника:
% ---
$$\left\{\begin{aligned}
&n\equiv 0\modb{2}\implies\text{ точка снаружи}\\
&n\equiv 1\modb{2}\implies\text{ точка внутри}
\end{aligned}\right.$$
\end{list*}

\subsection{Метод заметающей прямой}

Да.

\newpage
\subsection{Уравнения кривых}

{\bold Алгебраическая кривая} --- это...

Кривые первого порядка --- прямая.

Уравнение ромба

Кривые второго порядка {\ital (коники)} --- эллипс, парабола, гипербола.

\subsection{ГМТ}

{\bold Биссектриса} --- это

{\bold Биссекторная плоскость} --- это...

Свойство равноудалённости, расстояния и пр..

Гомотетия также

\subsection{Геометрические тела}

Правильный тетраэдр, пирамида...

\subsection{Прямые в пространстве}

{\ital По взаимному расположению} прямые бывают трёх видов:
% ---
\begin{list*}
\item{\bold пересекающиеся}: имеют {\ital общую точку}
\item{\bold параллельные}: {\ital лежат} в одной плоскости и {\ital не~имеют} общих точек
\item{\bold скрещивающиеся}: {\ital не лежат} в одной плоскости
\end{list*}
% ---
\begin{theorem}
{\bold Свойство.} Угол между скрещивающимися прямыми равен углу между двумя {\ital пересекающимися} прямыми, которые соответственно им параллельны.
\end{theorem}

{\bold Сонаправленные} {\ital лучи} лежат в одной полуплоскости на параллельных прямых.

\begin{theorem}
{\bold Свойство.} Углы с сонаправленными сторонами {\ital равны}.
\end{theorem} 
