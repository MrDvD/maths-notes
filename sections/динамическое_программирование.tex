\section{Динамическое программирование}

\subsection{Модель динамики}

{\bold Целевой} называется функция, у которой нужно найти {\ital экстремум {\color{desc}(оптимальное значение)}}.

{\bold Состояние} системы зависит от конечного числа {\ital параметров {\color{desc} (часто от одного или двух)}}.
% ---
\begin{theorem}
{\bold Принцип оптимальности}. Оптимальное решение зависит лишь от текущего состояния и цели, а не от предыстории. {\ital\color{desc}(Р. Беллман)}
\end{theorem}
% ---
{\bold Сертификат} {\ital решения} --- последовательность управляющих шагов, которые оптимизируют целевую функцию.

{\ital Типы} задач на динамику:

\begin{list*}
\item оптимизация целевой функции;
\item подсчёт количества вариантов решения;
\item составление сертификата решения.
\end{list*}

\subsection{Подходы динамики}

{\bold Мемоизация} --- {\ital рекурсивный} подход динамики, при кото"=ром подсчитанные результаты {\ital кешируются} и используются повторно {\ital\color{desc}(вычисления отложены)}.

{\bold Табуляция} --- {\ital итеративный} подход динамики, при котором кеш заполняется сразу, на основе тривиальных подзадач.

Также пояснить про одномерный и двумерный кеш, определение кеша?

\newpage
\subsection{Задача о рюкзаке}

{\bold Задача о рюкзаке} --- множество задач {\ital комбинаторной оптимизации}, которые сводятся к выбору подмножества:
% ---
\begin{list*}
\item с максимальной {\ital стоимостью}
\item с соблюдением ограничения на {\ital вес}
\end{list*}

{\bold Типы} задач о рюкзаке:
% ---
\begin{list*}
\item 0/1 {\ital\color{desc} (каждый предмет в одном экземпляре)}
\item неограниченный {\ital\color{desc} (каждый предмет бесконечен)}
\item задача размена монет
\item разбиение $N$-множества {\ital (balanced/unbalanced)}
\end{list*}

\begin{theorem}
{\bold Задача.} Дан рюкзак вместимостью $C\leq 10^9$ и $N$ вещей, которые имеют {\ital вес} и {\ital стоимость}. Максимизировать стоимость рюкзака, если $\sum c_i\leq 10^4$.
\end{theorem}

{\bold Идея.} Из-за ограничений на вместимость {\ital обратим логику}.

Пусть $f(c,i)$ --- минимальный суммарный вес первых $i$ вещей, которые стоят не менее $c$.

Очевидно, что $f(0,i)=0$ для любых $i$.

Тогда верно рекуррентное соотношение:
% ---
$$f(c,i)=\min\{f(c,i-1),\ f(c-c_{i-1},i-1)+w_{i-1}\}$$
% ---
\begin{theorem}
{\bold Задача.} Дан набор $N$ гирек, которые имеют {\ital вес}. Можно ли разбить гирьки на две кучи, равные по {\ital количеству гирек} и {\ital массе}?
\end{theorem}

{\bold Идея.} Пусть $f(w,n,i)$ --- возможность составить кучу массой $w$ из $n$ гирек, используя первые $i$ гирек из набора.

Очевидно, что $f(0,0,i)=1$ для любых $i$.

Тогда верно рекуррентное соотношение:
% ---
$$f(w,n,i)=f(w,n,i-1)\lor f(w-w_{i-1},n-1,i-1)$$
% ---
Ответ будет лежать в $f(\frac{\sum w_i}{2},\frac{N}{2},N)$.

\begin{theorem}
{\bold Задача.} Дан набор $N$ гирек, которые имеют {\ital вес}. Можно ли разбить гирьки на три кучи, равные по {\ital массе}?
\end{theorem}

{\bold Идея.} Пусть $f(w_1,w_2,i)$ --- возможность составить две кучи массами $w_1$ и $w_2$, используя первые $i$ гирек из набора.

Очевидно, что $f(0,0,i)=1$ для любых $i$.

Тогда верно рекуррентное соотношение:
% ---
$$f(w_1,w_2,i)=\begin{sqcases*}
&f(w_1,w_2,i-1)\\
&f(w_1-w_{i-1},w_2,i-1)\\
&f(w_1,w_2-w_{i-1},i-1)\end{sqcases*}$$
% ---
Ответ будет лежать в $f(\frac{\sum w_i}{3},\frac{\sum w_i}{3},N)$.

\newpage
\subsection{Счастливые билеты}

\begin{theorem}
{\bold Задача.} Дано натуральное число $n$. Найти количество $2n$-значных счастливых билетов.
\end{theorem}

{\bold Идея.} Пусть $D_n^k$ --- количество $n$-значных чисел с суммой цифр $k$.

Легко проверить, что счастливых билетов ровно $D_{2n}^{9n}$:
% ---
$$\overline{a_1\dots a_nb_1\dots b_n}\mapsto\overline{a_1\dots a_n(9-b_1)\dots(9-b_n)}$$
% ---
Очевидно, что $D_0^0=1$, $D_0^{k}=0,\ k\greater 0$.

Тогда $D_n^k$ можно выразить через $(n-1)$-значное число, добавив любую цифру $j$:
% ---
$$D_n^k=\sum_{j=0}^9D_{n-1}^{k-j}$$
% ---
\begin{theorem}
{\bold Задача.} Пусть натуральное число {\ital красивое}, если сумма квадратов его цифр --- полный квадрат. Найти количество красивых чисел в диапазоне $[1;N]$.
\end{theorem}

{\bold Идея.} Пусть $D_n^k$ --- количество $n$-значных чисел с суммой квадратов цифр $k$.

Тогда верна рекуррентная формула:
% ---
$$D_{n}^{k+j^2}\pluseq D_{n-1}^k,\ j\in\{0,\dots,9\}$$
% ---
Заметим, что для фиксированного $n$ верно:
% ---
$$1\leq k\leq 81n$$
% ---
Для каждого {\ital полного квадрата} $k\in[1;81n]$ найдём все числа из диапазона $[1;N]$, опираясь на определение $D_n^k$:
% ---
\begin{list*}
\item в ответ пойдёт $D_{n-1}^{k-d^2}$, $d\less d_1$ --- первая цифра числа;
\item в ответ пойдёт $D_{n-2}^{k-d^2}$, $d\less d_2$ --- вторая цифра числа;
\item в ответ пойдёт $D_{n-i-1}^{k-d^2}$, $d\less d_i$ --- $i$-ая цифра числа.
\end{list*}

\subsection{LIS}

\begin{theorem}
{\bold Задача.} Дана последовательность целых чисел. Найти длину её наибольшей возрастабщей подпоследовательности {\ital\color{desc}(longest increasing subsequence, LIS)}.
\end{theorem}

{\bold Идея.} Что-то...

\subsection{Числа Фибоначчи}

\begin{theorem}
{\bold Задача.} На прямой дощечке вбиты гвозди. Соединить пары гвоздей нитками так, чтобы к каждому гвоздю была привязана хотя бы одна нитка, а суммарная длина всех ниток была минимальна. 
\end{theorem}

{\bold Идея.} Пусть $f(i)$ --- суммарная длина ниток для $1\dots i$ гвоздей.

Определим базовые случаи:
% ---
$$f(2)=x_2-x_1\qquad f(3)=x_3-x_1$$
% ---
Добавим один гвоздь к текущим:
% ---
\begin{list*}[][\#]
\item Оптимально соединяем первые $i-1$ гвоздей, а последний гвоздь --- с $i-1$-ым:
% ---
$$f(i-1)+x_i-x_{i-1}$$
% ---
\item Оптимально соединями первые $i-2$ гвоздей, а последний гвоздь --- с $i-1$-ым:

{\centering $f(i-2)+x_i-x_{i-1}$\par}
\end{list*}
% ---
Тогда значение $f(i)$ --- {\ital минимум} среди всех случаев.

\subsection{Наибольшая подматрица}

\begin{theorem}
{\bold Задача.} В прямоугольной таблице $N\times M$ клетки раскрашены в белый и чёрный цвета. Найти наибольшую по площади прямоугольную область белого цвета.
\end{theorem}

{\bold Идея.} Пусть $f(x,k)$ --- наибольший номер строки из отрезка $[-1;x]$, на которой в $k$-ом столбце есть клетка {\ital чёрного} цвета.

Для хранения этой динамики достаточно {\ital одномерного} массива длины $N$.

Так, для прямоугольника уже определены {\ital две границы}: верхняя и нижняя.

Пусть $\langle i,j\rangle$ --- правая нижняя граница прямоугольника.

По условию, нужно расширить прямоугольник влево до~{\ital первой} клетки $\langle k,j\rangle$, для которой верно:
% ---
$$f(k,j)\greater f(i,j)$$
% ---
За счёт линейного алгоритма поиска {\ital ближайшего большего} за константу конечный алгоритм станет {\ital линейным}.

\subsection{Двумерная динамика}

\begin{theorem}
{\bold Задача.} Дан массив $x$ прейскуранта кафе на $N$ дней. За~чек свыше $100$ единиц выдаётся купон на бесплатный обед. Минимизировать суммарный чек.
\end{theorem}

{\bold Решение.} Пусть $f(j,i)$ --- минимальная суммарная стоимость обедов на $i$-й день включительно при оставшихся $j$ купонах.

Очевидно, что $f(0,0)=0$.

Используем принцип оптимальности на префиксе:

{\centering\begin{tikzpicture}
\node (start) [draw, line width=1pt, align=left] at (0,0.5) {$f(j,i)$};
\node (ifgreat) [draw, line width=1pt, align=left] at (-1.6,-0.5) {$x_i\greater 100$};
\node (ifless) [draw, line width=1pt, align=right] at (1.6,-0.5) {$x_i\leq 100$};
\node (get) [draw, line width=1pt, anchor=east, align=left] at (-2.3,-2) {$f(j-1,i-1)+x_i$};
\node (use) [draw, line width=1pt, align=left] at (0,-2) {$f(j+1,i-1)$};
\node (buy) [draw, line width=1pt, anchor=west, align=left] at (2.3,-2) {$f(j,i-1)+x_i$};
\draw [-{>[length=6pt]}, line width=.8pt] (ifgreat.north) -- (start.west);
\draw [-{>[length=6pt]}, line width=.8pt] (ifless.north) -- (start.east);
\draw [-{>[length=6pt]}, line width=.8pt] (get.north) -- (ifgreat.south west);
\draw [-{>[length=6pt]}, line width=.8pt] (use.north) -- (ifgreat.south east);
\draw [-{>[length=6pt]}, line width=.8pt] (use.north) -- (ifless.south west);
\draw [-{>[length=6pt]}, line width=.8pt] (buy.north) -- (ifless.south east);
\end{tikzpicture}\par}

Минимизируя $f$, ответ будет лежать в $f(i,N)$ при определённом $i$.

\begin{theorem}
{\bold Задача.} Дана шахматная доска $N\times M$. Сколькими способами конь из $(1,1)$ может добраться до $(N,M)$, если он ходит из клетки $(j,i)$ в клетки $(j+2,i-1)$, $(j+2,i+1)$, $(j+1,i+2)$, $(j-1,i+2)$.
\end{theorem}

{\bold Решение.} Пусть $f(j,i)$ --- число способов добраться до клетки $(j,i)$ из начальной.

Очевидно, что $f(0,0)=1$.

Используем принцип оптимальности на префиксе:

{\centering\begin{tikzpicture}
\node (start) [draw, line width=1pt, align=left] at (0,0.5) {$f(j,i)$};
\node (c1) [draw, line width=1pt, anchor=east, align=left] at (-2.3,-0.5) {$f(j+1,i-2)$};
\node (c2) [draw, line width=1pt, anchor=east, align=left] at (-0.4,-2.1) {$f(j-1,i-2)$};
\node (c3) [draw, line width=1pt, anchor=west, align=left] at (0.4,-2.1) {$f(j-2,i+1)$};
\node (c4) [draw, line width=1pt, anchor=west, align=left] at (2.3,-0.5) {$f(j-2,i-1)$};
\draw [-{>[length=6pt]}, line width=.8pt] (c1.north) -- (start.west) node [midway, above, sloped, scale=.8, fill=white, fill opacity=.6, text opacity=1] {$j\less N,i\greater 2$};
\draw [-{>[length=6pt]}, line width=.8pt] (c2.north) -- (start.south west) node [midway, above, sloped, scale=.8, fill=white, fill opacity=.6, text opacity=1] {$j\greater 1,i\greater 2$};
\draw [-{>[length=6pt]}, line width=.8pt] (c3.north) -- (start.south east) node [midway, above, sloped, scale=.8, fill=white, fill opacity=.6, text opacity=1] {$j\greater 1,i\less M$};
\draw [-{>[length=6pt]}, line width=.8pt] (c4.north) -- (start.east) node [midway, above, sloped, scale=.8, fill=white, fill opacity=.6, text opacity=1] {$j\greater 2,i\greater 1$};
\end{tikzpicture}\par}

Суммируя $f$, ответ будет лежать в $f(N,M)$.

\begin{theorem}
{\bold Задача.} Очередь из $N$ человек покупает билеты в театр. $i$-ый человек покупает 1 билет за $A_i$ времени, 2 билета --- за $B_i$, 3 билета --- за $C_i$. Минимизировать время.
\end{theorem}

{\bold Решение.} Пусть $f(j,i)$ --- минимальное время, за которое обслужат $i$-го человека, причём у него останется $j$ билетов.

Очевидно, что $f(0,0)=0$.

Используем принцип оптимальности на префиксе:

{\centering\begin{tikzpicture}
\node (start) [draw, line width=1pt, align=center] at (0,0.5) {$f(j,i)$};
\node (zero) [draw, line width=1pt, align=right] at (-1.7,-0.5) {$j=0$};
\node (one) [draw, line width=1pt, align=center] at (0,-0.5) {$j=1$};
\node (two) [draw, line width=1pt, align=left] at (1.7,-0.5) {$j=2$};
\node (zeroo) [draw, line width=1pt, anchor=east, align=center] at (-2.3,-2.4) {min\\$\begin{gathered}f(0,i-1)+A_i\\f(1,i-1)\end{gathered}$};
\node (onee) [draw, line width=1pt, align=center] at (0,-2.4) {min\\$\begin{gathered}f(0,i-1)+B_i\\f(2,i-1)\end{gathered}$};
\node (twoo) [draw, line width=1pt, anchor=south west, align=left] at (2.3,-2.4) {$f(0,i-1)+C_i$};
\draw [-{>[length=6pt]}, line width=.8pt] (zero.north) -- (start.west);
\draw [-{>[length=6pt]}, line width=.8pt] (one.north) -- (start.south);
\draw [-{>[length=6pt]}, line width=.8pt] (two.north) -- (start.east);
\draw [-{>[length=6pt]}, line width=.8pt] (zeroo.north) -- (zero.south);
\draw [-{>[length=6pt]}, line width=.8pt] (onee.north) -- (one.south);
\draw [-{>[length=6pt]}, line width=.8pt] (twoo.north) -- (two.south);
\end{tikzpicture}\par}

Минимизируя $f$, ответ будет лежать в $f(0,N-1)$.
