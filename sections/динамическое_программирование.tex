\section{Динамическое программирование}

\subsection{Модель динамики}

{\bold Целевой} называется функция, у которой нужно найти {\ital экстремум {\color{desc}(оптимальное значение)}}.

{\bold Состояние} системы зависит от конечного числа {\ital параметров {\color{desc} (часто от одного или двух)}}.
% ---
\begin{theorem}
{\bold Принцип оптимальности}. Оптимальное решение зависит лишь от текущего состояния и цели, а не от предыстории. {\ital\color{desc}(Р. Беллман)}
\end{theorem}
% ---
{\bold Сертификат} {\ital решения} --- последовательность управляющих шагов, которые оптимизируют целевую функцию.

{\ital Типы} задач на динамику:

\begin{list*}
\item оптимизация целевой функции;
\item подсчёт количества вариантов решения;
\item составление сертификата решения.
\end{list*}

\subsection{Подходы динамики}

{\bold Мемоизация} --- {\ital рекурсивный} подход динамики, при кото"=ром подсчитанные результаты {\ital кешируются} и используются повторно {\ital\color{desc}(вычисления отложены)}.

{\bold Табуляция} --- {\ital итеративный} подход динамики, при котором кеш заполняется сразу, на основе тривиальных подзадач.

Также пояснить про одномерный и двумерный кеш, определение кеша?

\subsection{Задача о рюкзаке}

Когда-нибудь...

\newpage
\subsection{Счастливые билеты}

\begin{theorem}
{\bold Задача.} Дано натуральное число $n$. Найти количество $2n$-значных счастливых билетов.
\end{theorem}

{\bold Идея.} Пусть $D_n^k$ --- количество $n$-значных чисел с суммой цифр $k$.

Легко проверить, что счастливых билетов ровно $D_{2n}^{9n}$:
% ---
$$\overline{a_1\dots a_nb_1\dots b_n}\mapsto\overline{a_1\dots a_n(9-b_1)\dots(9-b_n)}$$
% ---
Очевидно, что $D_0^0=1$, $D_0^{k}=0,\ k\greater 0$.

Тогда $D_n^k$ можно выразить через $(n-1)$-значное число, добавив любую цифру $j$:
% ---
$$D_n^k=\sum_{j=0}^9D_{n-1}^{k-j}$$
% ---
\begin{theorem}
{\bold Задача.} Пусть натуральное число {\ital красивое}, если сумма квадратов его цифр --- полный квадрат. Найти количество красивых чисел в диапазоне $[1;N]$.
\end{theorem}

{\bold Идея.} Пусть $D_n^k$ --- количество $n$-значных чисел с суммой квадратов цифр $k$.

Тогда верна рекуррентная формула:
% ---
$$D_{n}^{k+j^2}\pluseq D_{n-1}^k,\ j\in\{0,\dots,9\}$$
% ---
Заметим, что для фиксированного $n$ верно:
% ---
$$1\leq k\leq 81n$$
% ---
Для каждого {\ital полного квадрата} $k\in[1;81n]$ найдём все числа из диапазона $[1;N]$, опираясь на определение $D_n^k$:
% ---
\begin{list*}
\item в ответ пойдёт $D_{n-1}^{k-d^2}$, $d\less d_1$ --- первая цифра числа;
\item в ответ пойдёт $D_{n-2}^{k-d^2}$, $d\less d_2$ --- вторая цифра числа;
\item в ответ пойдёт $D_{n-i-1}^{k-d^2}$, $d\less d_i$ --- $i$-ая цифра числа.
\end{list*}
