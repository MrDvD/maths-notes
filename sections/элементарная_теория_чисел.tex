\section{Элементарная теория чисел}
\subsection{Делимость}
Пусть $a,b\in\mathbb{Z}$. Тогда {\ital a} --- {\bold делитель} {\ital b}, когда
$$ax=b,\ x\in\mathbb{Z}\iff a\mid b\iff \lvert a\rvert\leq\lvert b\rvert$$
% ---
Отношение делимости {\ital транзитивно}, такое выражение можно {\ital перемножить}
с другим:
$$
\times\begin{cases}
\begin{aligned}
a&\mid b\\
c&\mid d
\end{aligned}
\end{cases}
\hspace*{-12pt}\implies
ac\mid bd
$$
% ---
Общий делитель чисел делит их {\ital линейную комбинацию}:
$$a\mid b,c\implies a\mid bx+cy,\quad x,y\in\mathbb{Z}$$
% ---
Заметим, что $a=bx+cy$\footnotes{*}{Такое уравнение называют {\ital соотношением Безу}, а
{\ital x} и {\ital y} --- {\ital коэффициентами Безу {\color{desc}(Э. Безу)}}.},
когда $(b,c)\mid a$.\par

{\bold Доказательство.} Пусть $d:=(b,c)$, тогда:
$$d\mid b,c\implies d\mid(bx+cy)\implies d\mid a\qedb$$
% ---
Коэффициенты Безу $(x,y)$ {\ital неуникальны} и легко выражаются
{\ital\color[HTML]{888888}(доказывается подстановкой в соотношение)}:
$$(x+mk,y-ak),\ k\in\mathbb{Z}$$

\newpage
\subsection{Наибольший общий делитель}

{\ital Наибольший общий делитель}\footnotes{*}{Сокращённо {\ital НОД}, или {\ital Greatest Common Divisor} ({\ital GCD}).} для $\{a_k\}_{k\in\mathbb{N}}$ --- такое $gcd(\{a_k\})$, что
% ---
$$\exists d\colon d\mid gcd(\{a_k\})\mid\{a_k\}.$$
% ---
Упрощённая запись $gcd(\{a_k\})=(\{a_k\})$.

Этот бинарный оператор {\ital коммутативен}, {\ital ассоциативен}\\ и {\ital дистрибутивен}.

\subsection{Наименьшее общее кратное}

{\ital Наименьшее общее кратное}\footnotes{**}{Сокращённо {\ital НОК}, или {\ital Least Common Multiple} ({\ital LCM}).} для $\{a_k\}_{k\in\mathbb{N}}$ --- такое $lcm(\{a_k\})$, что
% ---
$$\exists m\colon \{a_k\}\mid lcm(\{a_k\})\mid m.$$
% ---
Упрощённая запись $lcm(\{a_k\})=[\{a_k\}]$.

Этот бинарный оператор {\ital коммутативен} и {\ital ассоциативен}, однако {\ital не дистрибутивен}.

\subsection{Двойственность}

НОД и НОК {\ital двойственны} друг другу:
% ---
$$(a,b)\cdot[a,b]=ab$$
% ---
{\bold Доказательство.} Пусть $m:=[a,b]$, тогда:
% ---
$$a,b\mid m\iff ab\mid am,bm\iff ab\mid (am,bm)\iff ab\mid (a,b)m$$
% ---
Так как $(a,b)\mid [a,b]\mid ab$, то $ab/(a,b)\mid [a,b]$.

Значит, $ab/(a,b)\leq [a,b]$. Но $[a,b]$ --- {\ital наименьшее} общее кратное $a,b$. Следовательно, $ab/(a,b)\nless [a,b]$, поэтому:
% ---
$$ab/(a,b)=[a,b]\iff ab=(a,b)\cdot [a,b]\qedb$$
