\section{Теория множеств}

\subsection{Определение}

{\bold Множеством} называется объединение {\ital различных} объектов --- {\ital элементов} множества --- в единое целое.

\begin{theorem}
{\bold Способы задания множеств.}
% ---
\begin{list*}[][\#]
\item Перечислением {\ital (списком элементов)}
\item Порождающей процедурой
\item Разрешающей процедурой
\end{list*}
\end{theorem}

Множество $A$ есть {\bold подмножество} $B$, если все элементы $A$ являются элементами $B$:
% ---
$$A\subseteq B$$  
% ---
Множество $A$ есть {\bold надмножество} $B$, если все элементы $B$ являются элементами $A$:
% ---
$$A\supseteq B$$
% ---
Подмножество {\bold собственное} {\ital (строгое)}, если оно {\ital не равно} исходному множеству:
% ---
$$A\subseteq B\land A\neq B\implies A\subset B$$
% ---
\begin{theorem}
{\bold Булеан} множества ${\mathcal B}(A)$ --- множество всех его подмножеств:
% ---
$${\mathcal B}(A)=\{X\mid X\subseteq A\}$$
% ---
{\ital Мощность} булеана определяется формулой:
% ---
$$\abs{{\mathcal B}(A)}=2^{\abs{A}}$$
\end{theorem}
% ---
\begin{theorem}
{\bold Метод взаимного включения.} Множества $A$ и $B$ {\ital равны}, если они содержат одни и те же элементы:
% ---
$$A\subseteq B\land A\supseteq B$$
\end{theorem}

\begin{theorem}
{\bold Пустое множество} $\emptyset$ не содержит {\ital ни одного элемента} и есть подмножество любого множества.

{\bold Универсум} $\mathbb{U}$ --- широкое множество, которое состоит из всех элементов {\ital исследуемой области}.
\end{theorem}

{\bold Конечным} множество состоит из {\ital конечного} числа элементов, а {\bold бесконечное} множество --- из {\ital бесконечного}.

\begin{theorem}
{\ital Бесконечные} множества делятся на два вида:
% ---
\begin{list*}
\item {\bold cчётное}: равномощно множеству $\mathbb{N}$ {\ital\color{desc} (его можно пронумеровать)}
\item {\bold несчётное}: не равномощно $\mathbb{N}$ {\ital\color{desc} (его нельзя пронумеровать)}
\end{list*}
\end{theorem}

{\bold Мощность} {\ital конечного} множества --- число его элементов.

\subsection{Вектор}

{\bold Вектор} {\ital (кортеж, или упорядоченная n-ка)} --- упорядоченный набор элементов --- {\ital координат (компонент)} вектора.

{\bold Размерность} вектора --- число его координат.

\begin{theorem}
{\bold Декартово} {\ital (прямое)} {\bold произведение} множеств $X_1,\dots,X_n$ --- множество векторов вида:
% ---
$$X_1\times\dots\times X_n=\{(x_1,\dots,x_n)\mid x_1\in X_1,\dots,x_n\in X_n\}$$
% ---
Свойства:
% ---
\begin{list*}
\item дистрибутивность относительно {\ital пересечения} $\cap$
\item дистрибутивность относительно {\ital разности} $\backslash$
\item {\ital *не коммутативность}
\item {\ital *не ассоциативность}
\item $(A\cap B)\times(C\cap D)=(A\times C)\cap(B\times D)$
\end{list*}
\end{theorem}
% ---
{\bold Декартова степень} множества --- прямое произведение одинаковых множеств:
% ---
$$A^3=A\times A\times A$$
% ---
\begin{theorem}
{\bold Теорема.} Мощность декартова произведения конечных множеств равна {\ital произведению мощностей} данных множеств:
% ---
$$\abs{A_1\times\dots\times A_n}=\abs{A_1}\times\dots\times\abs{A_n}$$
\end{theorem}

\subsection{Отношение}

{\bold Отношение} $\rho$ между множествами $X_1,\dots,X_n$ --- подмножество их {\ital декартова произведения}.

\begin{theorem}
{\bold Бинарное} отношение включает {\ital два} множества, что можно упрощённо записать:
% ---
$$x\in X,\ y\in Y, \langle x, y\rangle\in\rho=:x\rho y$$
% ---
{\bold Унарное} отношение {\ital\color{desc} (свойство)} включает в себя...?
\end{theorem}

\begin{tabularc}{0pt}{0pt}{r @{ --- } l}{n}
$X\supseteq D_\rho$ & область определения {\bold (прообраз)} отношения\\
$Y\supseteq E_\rho$ & область значений {\bold (образ)} отношения.
\end{tabularc}
% ---
{\bold Наложение} {\ital (сюръекция)} --- бинарное отношение с $Y=E_\rho$:
% ---
$$\begin{gathered}
\forall y\in Y\ \exists x\in D_\rho\colon x\rho y\\
\rho\colon X\twoheadrightarrow Y
\end{gathered}$$
% ---
У {\bold частично определённого} бинарного отношения $X\neq D_\rho$:
% ---
$$\exists x\in X\ \forall y\in E_\rho\colon x\not\rho y$$

\subsection{Виды отношений}

{\bold Вложение} {\ital (инъекция, мономорфизм)} --- бинарное отношение $\rho$ вида:
% ---
$$\begin{gathered}
\forall x_1,x_2\in D_\rho\ \exists y\in E_\rho\colon x_1\rho y,\ x_2\rho y\iff x_1=x_2\\
\rho\colon X\hookrightarrow Y
\end{gathered}$$
% ---
{\bold Изоморфизм} {\ital (биекция)} --- бинарное отношение, которое является {\ital и вложением, и наложением}:
% ---
$$\rho\colon X\xrightarrow{\sim}Y$$
% ---
{\bold Функция} {\ital (отображение)} --- бинарное отношение $\rho$ вида:
% ---
$$\begin{gathered}
\forall x\in D_\rho\ \exists!y\in E_\rho\colon x\rho y\\
\rho\colon X\xrightarrow{x\mapsto y} Y
\end{gathered}$$
% ---
\begin{theorem}
{\bold Композиция} {\ital (суперпозиция)} бинарных отношений\\ $f\subseteq X\times Y,\ g\subseteq Y\times Z$ --- такое $h\subseteq X\times Z\iff f\circ g$, что:
% ---
$$\forall x\in X,\ z\in Z\exists y\in Y\ x(f\circ g)z\iff xfy\land ygz$$
% ---
Свойства:

\begin{list*}
\item ассоциативность
\item {\ital *не коммутативность}
\end{list*}
\end{theorem}

{\bold Тождестенное отображение} --- функция вида:
% ---
$$\Id_X\colon X\to X\land\Id_X(x)=x$$
% ---
\begin{theorem}
{\bold Обратное} бинарное отношение $\rho^{-1}$ получается переста"=новкой исходных множеств в декартовом произведении:
% ---
$$\rho\subseteq X\times Y\iff\rho^{-1}\subseteq Y\times X$$
% ---
Свойства:
% ---
\begin{list*}
\item инволютивность
\item дистрибутивность относительно {\ital пересечения} $\cap$
\item дистрибутивность относительно {\ital объединения} $\cup$
\item дистрибутивность относительно {\ital композиции} $\circ$:
% ---
$$(P\circ Q)^{-1}=Q^{-1}\circ R^{-1}$$
\end{list*}
\end{theorem}

\subsection{Свойства отношений}

Пусть $\ast\subseteq X^2$ --- произвольное бинарное отношение.

Отношение $\ast$ {\bold симметрично}, когда
% ---
$$\forall x,y\in X\ x\ast y=y\ast x.$$
% ---
Отношение $\ast$ {\bold антисимметрично}, когда
% ---
$$\forall x,y\in X\ x\ast y\land y\ast x\implies x=y$$
% ---
Отношение $\ast$ {\bold транзитивно}, когда
% ---
$$\forall x,y,z\in X\ x\ast y\land y\ast z\implies x\ast z$$
% ---
Отношение $\ast$ {\bold рефлексивно}, когда
% ---
$$\forall x\in X\ x\ast x$$
% ---
Отношение $\ast$ {\bold антирефликсивно}, когда
% ---
$$\forall x\in X\ \lnot (x\ast x)$$

\subsection{Эквиваленция}

{\bold Эквиваленция} $\sim$ --- бинарное отношение на $M$ со~свойствами:
% ---
\begin{list*}[][\#]
\item Рефлексивность.
\item Симметричность.
\item Транзитивность.
\end{list*}

{\bold Класс эквивалентности} --- множество вида:
% ---
$$[a]=\{b\in M\mid a\sim b\}$$
% ---
\begin{theorem}
{\bold Свойство.} Два класса эквивалентности либо совпадают, либо не пересекаются.
\end{theorem}
% ---
{\bold Фактормножество} множества $M$ --- множество, которое состоит из классов эквивалентности отношения $\sim$:
% ---
$$N\slash\sim\text{ --- обозначение}$$
% ---
{\bold Отображение факторизации} --- отображение вида:
% ---
$$N\xrightarrow{a\mapsto [a]} M\slash\sim$$
% ---
Эквиваленция {\bold согласована} с операцией $\ast$ в $M$, если:
% ---
$$\begin{gathered}
a\sim a'\land b\sim b'\implies a\ast b\sim a'\ast b'\\
[a]\ast[b]=[a\ast b]
\end{gathered}$$

\subsection{Ограничение и продолжение}

{\ital Ограничением} отображения $f\colon X\to Y$ на $S\subseteq D_f$ называется
такое $f\vert_S\colon S\to Y$, что
% ---
$$\forall s\in S\colon f\vert_S(s)=f(s).$$
% ---
В свою очередь, $f$ является {\ital продолжением} отображения $f\vert_S$.

\subsection{Промежутки числовой прямой}

{\bold Отрезок} --- множество вида:
% ---
$$[a,b]=\{x\in\mathbb{R}\mid a\leq x\leq b\}$$
% ---
{\bold Интервал} --- множество вида:
% ---
$$[a,b]=\{x\in\mathbb{R}\mid a\less x\less b\}$$
% ---
{\bold Полуинтервал} --- множества вида:
% ---
$$\begin{gathered}
[a,b)=\{x\in\mathbb{R}\mid a\leq x\less b\}\\
(a,b]=\{x\in\mathbb{R}\mid a\less x\leq b\}
\end{gathered}$$
% ---
{\bold Луч} --- множества вида:
% ---
$$\begin{gathered}
[a,+\infty)=\{x\in\mathbb{R}\mid x\geq a\}\\
(a,+\infty)=\{x\in\mathbb{R}\mid x\greater a\}\\
(-\infty,b]=\{x\in\mathbb{R}\mid x\leq a\}\\
(-\infty,b)=\{x\in\mathbb{R}\mid x\less a\}
\end{gathered}$$
% ---
\begin{theorem}
{\bold $\symbf{\varepsilon}$-окрестность} точки $x_0\in\mathbb{R}$ --- интервал вида:
% ---
$$U_\varepsilon(x_0)=\{x\in\mathbb{R}\colon\abs{x-x_0}\less\varepsilon\}=(x_0-\varepsilon,x_0+\varepsilon)$$

Особые случаи:
% ---
$$\begin{gathered}
U_\varepsilon(+\infty)=(1/\varepsilon,+\infty)\\
U_\varepsilon(-\infty)=(-\infty,-1/\varepsilon)
\end{gathered}$$
\end{theorem}
% ---
{\bold Проколотая} $\varepsilon$-окрестность точки $x_0$ не включает саму точку:\\[-8pt]
% ---
$$\overset{\circ}{U}_\varepsilon(x_0)=U_\varepsilon(x_0)\backslash\{x_0\}$$
% ---
{\bold Правосторонняя (левосторонняя)} $\varepsilon$-окрестность точки $x_0$
не содержит свою {\ital левую  (правую)} половину:
% ---
$$U_{\varepsilon+}(x_0)=[x_0;\varepsilon)\quad\quad U_{\varepsilon-}(x_0)=(\varepsilon;x_0]$$

\subsection{Ограниченное множество}

Множество $M$ {\bold ограничено сверху}, если:
% ---
$$\forall m\in M\ \exists C\in\mathbb{R}\colon m\leq C$$
% ---
{\bold Верхняя граница} множества $M$ --- такое $N\in\mathbb{R}$, что:
% ---
$$\forall x\in M\colon x\leq N$$
% ---
{\bold Наибольший элемент} {\ital (максимум)} множества $M$ --- такое $\max M\in M$, что:
% ---
$$\forall x\in M\ x\leq\max M$$
% ---
{\bold Супремум} {\ital (точная верхняя граница)} --- такая верхняя граница множества $M$ --- $\sup M$, что:
% ---
$$\begin{gathered}
\forall\varepsilon\greater 0\ \exists m\in M\colon m\in U_{\varepsilon-}(\sup M)\\
{\color{desc}(\forall s\less\sup M\ \exists m\in M\colon s\less m\leq\sup M)}
\end{gathered}$$
% ---
\begin{theorem}
{\bold Принцип точной грани.} Если непустое множество $M$ ограничено сверху, то существует {\ital единственный} $\sup M$.
\end{theorem}
% ---
\begin{theorem}
{\bold Связь супремума с максимумом.} Если у множества $M$ существует $\max M$, то:
% ---
$$\sup M=\max M$$ 
\end{theorem}
% ---
Множество $M$ {\bold ограничено снизу}, если:
% ---
$$\forall m\in M\ \exists C\in\mathbb{R}\colon m\geq C$$
% ---
{\bold Нижняя граница} множества $M$ --- такое $N\in\mathbb{R}$, что:
% ---
$$\forall x\in M\colon x\geq N$$
% ---
{\bold Наименьший элемент} {\ital (минимум)} множества $M$ --- такое $\min N\in M$, что:
% ---
$$\forall x\in M\ x\geq\min M$$
% ---
{\bold Инфимум} {\ital (точная нижняя граница)} --- такая нижняя граница множества $M$ --- $\inf M$, что:
% ---
$$\begin{gathered}
\forall\varepsilon\greater 0\ \exists m\in M\colon m\in U_{\varepsilon+}(\inf M)\\
{\color{desc}(\forall s\greater\inf M\ \exists m\in M\colon \inf M\leq m\less s)}
\end{gathered}$$
% ---
\begin{theorem}
{\bold Принцип точной грани.} Если непустое множество $M$ ограничено снизу, то существует {\ital единственный} $\inf M$.
\end{theorem}
% ---
\begin{theorem}
{\bold Связь инфимума с минимумом.} Если у множества $M$ существует $\min M$, то:
% ---
$$\inf M=\min M$$ 
\end{theorem}
% ---
{\bold Ограниченное} множество $M$ ограничено {\ital и сверху, и снизу}:
% ---
$$\exists N\in\mathbb{R}\colon\forall x\in M\ \abs{x}\leq N$$
% ---
\begin{theorem}
{\bold Принцип Архимеда.} Пусть $x\in\mathbb{R}^+$. Тогда справедливо:
% ---
$$\forall y\in\mathbb{R}\ \exists!k\in\mathbb{Z}\colon (k-1)x\less y\leq kx$$
% ---
{\bold Следствие.} Для $x\in\mathbb{R}$ существует такое $k\in\mathbb{Z}$, что:
% ---
$$k\leq x\less k+1\qquad{\color{desc} (k=[x])}$$
\end{theorem}
