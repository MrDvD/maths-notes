\section{Теория множеств}

\subsection{Открытое множество}

{\ital $\varepsilon$-окрестность} точки $x_0\in X$ метрического пространства
$\langle X,d\rangle$ --- такое множество точек $x\in X$, что $d(x_0,x)\less\varepsilon$.

Упрощённая запись $\{x\mid d(x_0,x)\less\varepsilon\}=:U_\varepsilon(x_0)$.

Особые случаи:
% ---
\begin{align*}
U_\varepsilon(+\infty)&:=(1/\varepsilon;+\infty)\\
U_\varepsilon(-\infty)&:=(-\infty;-1/\varepsilon)
\end{align*}
% ---
{\ital Проколотой} называется $\varepsilon$-окрестность точки $x_0$ без неё:\\[-8pt]
% ---
$$\overset{\circ}{U}_\varepsilon(x_0):=U_\varepsilon(x_0)\backslash\{x_0\}$$
% ---
{\ital Правосторонней (левосторонней)} называется $\varepsilon$-окрестность точки $x_0$
без левой (правой) половины:
% ---
$$U_{\varepsilon+}(x_0):=[x_0;\varepsilon)\quad\quad U_{\varepsilon-}(x_0):=(\varepsilon;x_0]$$

\subsection{Ограниченное множество}

Множество $M$ ограничено {\ital сверху}, если
% ---
$$\forall m\in M\ \exists C\in\mathbb{R}\colon m\leq C.$$
% ---
{\bold Точной} {\ital (минимальной}, англ. {\ital supremum)} называется такая
{\ital верхняя} граница множества $M$ --- $\sup M$, что
% ---
$$\forall\varepsilon\greater 0\ \exists m\in M\colon m\in U_{\varepsilon-}(\sup M).$$
% ---
Множество $M$ ограничено {\ital снизу}, если
% ---
$$\forall m\in M\ \exists C\in\mathbb{R}\colon m\geq C.$$
% ---
{\bold Точной} {\ital (максимальной}, англ. {\ital infimum)} называется такая
{\ital нижняя} граница множества $M$ --- $\inf M$, что
% ---
$$\forall\varepsilon\greater 0\ \exists m\in M\colon m\in U_{\varepsilon+}(\inf M).$$

\subsection{Принцип Кантора}

Последовательность вложенных отрезков содержит точки $\xi$, которые принадлежат им всем:
% ---
$$\forall n\in\mathbb{N}\ \exists\xi\in[a_n;b_n]\subset[a_{n-1};b_{n-1}]$$
% ---
Если $n\to\infty$, $(b_n-a_n)\to 0$, то $\xi$ единственна:
% ---
$$\lim_{n\to\infty}a_n=\sup\{a_n\}=\lim_{n\to\infty}b_n=\inf\{b_n\}=\xi$$
% ---
{\bold Доказательство.} По теореме Вейерштрасса:
% ---
$$\lim_{n\to\infty}a_n=\sup\{a_n\}\quad\quad\lim_{n\to\infty}b_n=\inf\{b_n\}$$
% ---
Значит, $\forall(n\in\mathbb{N},\ \xi\in[\sup\{a_n\};\inf\{b_n\}])\ \xi\in[a_n;b_n]$.
$\qedw$

Если $\inf\{b_n\}=\sup\{a_n\}$, то $\xi$ единственна:
% ---
$$0=\inf\{b_n\}-\sup\{a_n\}=\lim_{n\to\infty}b_n-\lim_{n\to\infty}a_n=\lim_{n\to\infty}
(b_n-a_n)\qedb$$
% ---
\subsection{Определение}

{\bold Множеством} называется объединение {\ital различных} объектов --- {\ital элементов} множества --- в единое целое.

\begin{theorem}
{\bold Способы задания множеств.}
% ---
\begin{list*}[][\#]
\item Перечислением {\ital (списком элементов)}
\item Порождающей процедурой
\item Разрешающей процедурой
\end{list*}
\end{theorem}

Множество $A$ есть {\bold подмножество} $B$, если все элементы $A$ являются элементами $B$:
% ---
$$A\subseteq B$$  
% ---
Множество $A$ есть {\bold надмножество} $B$, если все элементы $B$ являются элементами $A$:
% ---
$$A\supseteq B$$
% ---
Подмножество {\bold собственное} {\ital (строгое)}, если оно {\ital не равно} исходному множеству:
% ---
$$A\subseteq B\land A\neq B\implies A\subset B$$
% ---
\begin{theorem}
{\bold Булеан} множества ${\mathcal B}(A)$ --- множество всех его подмножеств:
% ---
$${\mathcal B}(A)=\{X\mid X\subseteq A\}$$
% ---
{\ital Мощность} булеана определяется формулой:
% ---
$$\abs{{\mathcal B}(A)}=2^{\abs{A}}$$
\end{theorem}
% ---
\begin{theorem}
{\bold Метод взаимного включения.} Множества $A$ и $B$ {\ital равны}, если они содержат одни и те же элементы:
% ---
$$A\subseteq B\land A\supseteq B$$
\end{theorem}

\begin{theorem}
{\bold Пустое множество} $\emptyset$ не содержит {\ital ни одного элемента} и есть подмножество любого множества.

{\bold Универсум} $\mathbb{U}$ --- широкое множество, которое состоит из всех элементов {\ital исследуемой области}.
\end{theorem}

{\bold Конечным} множество состоит из {\ital конечного} числа элементов, а {\bold бесконечное} множество --- из {\ital бесконечного}.

\begin{theorem}
{\ital Бесконечные} множества делятся на два вида:
% ---
\begin{list*}
\item {\bold cчётное}: равномощно множеству $\mathbb{N}$ {\ital\color{desc} (его можно пронумеровать)}
\item {\bold несчётное}: не равномощно $\mathbb{N}$ {\ital\color{desc} (его нельзя пронумеровать)}
\end{list*}
\end{theorem}

{\bold Мощность} {\ital конечного} множества --- число его элементов.

\subsection{Вектор}

{\bold Вектор} {\ital (кортеж, или упорядоченная n-ка)} --- упорядоченный набор элементов --- {\ital координат (компонент)} вектора.

{\bold Размерность} вектора --- число его координат.

\begin{theorem}
{\bold Декартово} {\ital (прямое)} {\bold произведение} множеств $X_1,\dots,X_n$ --- множество векторов вида:
% ---
$$X_1\times\dots\times X_n=\{(x_1,\dots,x_n)\mid x_1\in X_1,\dots,x_n\in X_n\}$$
% ---
Свойства:
% ---
\begin{list*}
\item дистрибутивность относительно {\ital пересечения} $\cap$
\item дистрибутивность относительно {\ital разности} $\backslash$
\item {\ital *не коммутативность}
\item {\ital *не ассоциативность}
\item $(A\cap B)\times(C\cap D)=(A\times C)\cap(B\times D)$
\end{list*}
\end{theorem}
% ---
{\bold Декартова степень} множества --- прямое произведение одинаковых множеств:
% ---
$$A^3=A\times A\times A$$
% ---
\begin{theorem}
{\bold Теорема.} Мощность декартова произведения конечных множеств равна {\ital произведению мощностей} данных множеств:
% ---
$$\abs{A_1\times\dots\times A_n}=\abs{A_1}\times\dots\times\abs{A_n}$$
\end{theorem}

\subsection{Отношение}

{\bold Отношение} $\rho$ между множествами $X_1,\dots,X_n$ --- подмножество их {\ital декартова произведения}.

\begin{theorem}
{\bold Бинарное} отношение включает {\ital два} множества, что можно упрощённо записать:
% ---
$$x\in X,\ y\in Y, \langle x, y\rangle\in\rho=:x\rho y$$
% ---
{\bold Унарное} отношение {\ital\color{desc} (свойство)} включает в себя...?
\end{theorem}

\begin{tabularc}{0pt}{0pt}{r @{ --- } l}{n}
$X\supseteq D_\rho$ & область определения {\ital (прообраз)} отношения;\\
$Y\supseteq E_\rho$ & область значений {\ital (образ)} отношения.
\end{tabularc} 

\subsection{Виды отношений}

Отношение $\rho$ {\ital инъективно}, когда
% ---
$$\forall x_1,x_2\in D_\rho\ \exists y\in E_\rho\colon x_1\rho y,\ x_2\rho y\iff
x_1=x_2.$$
% ---
Отношение $\rho$ {\ital функционально}, когда
% ---
$$\forall x\in D_\rho\ \exists! y\in E_\rho\colon x\rho y.$$
% ---
\begin{theorem}
Функциональное отношение называется {\bold отображением} {\ital (функцией)}
и обозначается:
% ---
$$\rho\colon X\xrightarrow{x\mapsto y} Y$$
\end{theorem}
% ---
Отношение $\rho$ {\ital сюръективно}, когда
% ---
$$\forall y\in Y\ \exists x\in D_\rho\colon x\mapsto y.$$
% ---
Отношение $\rho$ {\ital всюду определено}, когда
% ---
$$\forall x\in X\ \exists y\in E_\rho\colon x\mapsto y.$$
% ---
\begin{theorem}
{\bold Композиция} {\ital (суперпозиция)} бинарных отношений\\ $f\subseteq X\times Y,\ g\subseteq Y\times Z$ --- такое $h\subseteq X\times Z\iff f\circ g$, что:
% ---
$$\forall x\in X,\ z\in Z\exists y\in Y\ x(f\circ g)z\iff xfy\land ygz$$
% ---
Свойства:

\begin{list*}
\item ассоциативность
\item {\ital *не коммутативность}
\end{list*}
\end{theorem}

\begin{theorem}
{\bold Обратное} бинарное отношение $\rho^{-1}$ получается переста"=новкой исходных множеств в декартовом произведении:
% ---
$$\rho\subseteq X\times Y\iff\rho^{-1}\subseteq Y\times X$$
% ---
Свойства:
% ---
\begin{list*}
\item идемпотентность
\item дистрибутивность относительно {\ital пересечения} $\cap$
\item дистрибутивность относительно {\ital объединения} $\cup$
\item дистрибутивность относительно {\ital композиции} $\circ$:
% ---
$$(P\circ Q)^{-1}=Q^{-1}\circ R^{-1}$$
\end{list*}
\end{theorem}

\subsection{Свойства отношений}

Пусть $\ast\subseteq X^2$ --- произвольное бинарное отношение.

Отношение $\ast$ {\ital симметрично}, когда
% ---
$$\forall x,y\in X\ x\ast y=y\ast x.$$
% ---
Отношение $\ast$ {\ital антисимметрично}, когда
% ---
$$\forall x,y\in X\ x\ast y\land y\ast x\implies x=y$$
% ---
Отношение $\ast$ {\ital транзитивно}, когда
% ---
$$\forall x,y,z\in X\ x\ast y\land y\ast z\implies x\ast z$$
% ---
Отношение $\ast$ {\ital рефлексивно}, когда
% ---
$$\forall x\in X\ x\ast x$$
% ---
Отношение $\ast$ {\ital антирефликсивно}, когда
% ---
$$\forall x\in X\ \lnot (x\ast x)$$

\subsection{Ограничение и продолжение}

{\ital Ограничением} отображения $f\colon X\to Y$ на $S\subseteq D_f$ называется
такое $f\vert_S\colon S\to Y$, что
% ---
$$\forall s\in S\colon f\vert_S(s)=f(s).$$
% ---
В свою очередь, $f$ является {\ital продолжением} отображения $f\vert_S$.\par
