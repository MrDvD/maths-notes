\section{Материал к ЕГЭ}

\subsection{16 задание}

\begin{tblr}[label=none, presep=6pt, postsep=6pt]{colspec={Q[c, 40mm, m]X[l, m]}, row{1}={table}}
{\bold Величина} & {\bold Название}\\
$S$ & первоначальная сумма долга\\\hline
$B_i$ & размер долга на конец $i$-го периода\\\hline
$X_i$, $x$ & размер платежа в $i$-ый период\\\hline
$n$ & число платёжных периодов\\\hline
$r$ & учётная ставка {\ital\color{desc}(в \%)}\\\hline
$p=1+0.01r$ & повышающий коэффициент
\end{tblr}

{\bold Аннуитетный} платёж --- долг выплачивается {\ital равными платежами}.

\begin{theorem}
Уравнение аннуитетного платежа:
% ---
$$p^nS=x\left(\frac{p^n-1}{p-1}\right)$$
\end{theorem}
% ---
{\bold Дифференцированный} платёж --- долг {\ital уменьшается равномерно}, при этом платежи в каждый период {\ital разные}.

\begin{theorem}
Рекуррентные формулы дифференцированного платежа:
% ---
$$\begin{cases*}
&B_i=pB_{i-1}-X_i\\
&B_i=\frac{n-i}{n}S
\end{cases*}$$
% ---
$X_i$ образует {\ital арифметическую прогрессию}.
\end{theorem}
