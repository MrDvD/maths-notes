\documentclass[12pt]{article}

% remove: microtype, xcolor, empheq, float, nccmath, graphicx

% packages
\usepackage{cellspace}
\usepackage{adjustbox}

\usepackage[a5paper,margin=12mm]{geometry}                      % макет
\usepackage{polyglossia}                                        % язык
\usepackage[warnings-off={mathtools-colon,
                          mathtools-overbracket}]{unicode-math} % \setmathfont
\usepackage{titlesec}                                           % \titleformat
\usepackage[open,openlevel=1]{bookmark}                         % индексация pdf
\usepackage[shortlabels]{enumitem}                              % списки
\usepackage[table]{xcolor}                                      % \rowcolor
\usepackage{xargs}                                              % \newenvironmentx
\usepackage[most]{tcolorbox}                                    % блоки
\usepackage[makeroom]{cancel}                                   % сокращение (maths)
\usepackage{tabularx,multicol,multirow,makecell}                % таблицы
\usepackage{tikz}                                               % графика
\usetikzlibrary{intersections}                                  % 'name intersections'
\usetikzlibrary{patterns}                                       % 'pattern'
\usetikzlibrary{arrows.meta}                                    % arrows customization

% enables russian hyphenation
\setmainlanguage[babelshorthands=true]{russian}
\setotherlanguage{english}

% mainfont
\newfontfamily\bold{Century Schoolbook Bold}
\newfontfamily\ital{Century Schoolbook Italic}
\newfontfamily\boldital{Century Schoolbook Bold Italic}
% titlefont
\newfontfamily\gilroysub[Scale=1.31,UprightFont={*-Medium}]{Gilroy}
\newfontfamily\gilroy[Scale=1.59,UprightFont={*-UltraLight}]{Gilroy}
\newfontfamily\tinyt[Scale=.835]{Century Schoolbook}
% monofont
\newfontfamily\mono{Anonymous Pro Regular}
\newfontfamily\monobold{Anonymous Pro Bold}
\newfontfamily\monoital{Anonymous Pro Italic}

% font format
\setmainfont{Century Schoolbook}
\setmathfont{TeX Gyre Schola Math}

\titleformat*{\subsection}{\raggedright\gilroysub}
\titleformat*{\section}{\gilroy\centering}
\titlespacing*{\subsection}{0pt}{6pt}{0pt}
\titlespacing*{\section}{0pt}{6pt}{1pt}

% colors
\definecolor{desc}{HTML}{888888}
\definecolor{table}{HTML}{D9D9D9}

% disables numeration
\pagenumbering{gobble}
\setcounter{secnumdepth}{0}

% shorthands
\newcommand\en[1]{\foreignlanguage{english}{#1}}     % english hyphenation
\newcommand\qedb{\ \char"25A0}                       % q.e.d filled
\newcommand\qedw{\ \char"25A1}                       % q.e.d empty
\newcommand\abs[1]{\left|#1\right|}                  % abs function
\newcommand\modb[1]{\ \left(\mathrm{mod}\ #1\right)} % modulo w brackets
\newcommand\modn[1]{\ \mathrm{mod}\ #1}              % modulo w/o brackets
\newcommand\pluseq{\mathrel{+}=}                     % += operator

% cases env
\newenvironment{cases*}{
\left\{\begin{aligned}
}{\end{aligned}\right.}

% rcases env
\newenvironment{rcases*}{
\left.\begin{aligned}
}{\end{aligned}\right\}}

% sqcases env
\newenvironment{sqcases*}{
\left[\begin{aligned}
}{\end{aligned}\right.}

% itemize & enumerate env
\newenvironmentx{list*}[2][1=1,2=-]{
\if #2-
\def\listenv{itemize}
\ifnum #1=2
\begin{\listenv}[leftmargin=4.5mm,itemsep=-3pt,topsep=-3pt,label=>]
\else
\begin{\listenv}[leftmargin=6.5mm,itemsep=-3pt,topsep=0pt,label=---]
\fi
\fi
\if #2\#
\def\listenv{enumerate}
\begin{\listenv}[leftmargin=6.5mm,itemsep=-3pt,topsep=0pt,label=\theenumi)]
\fi
}{\end{\listenv}}

% theorem box
\newtcolorbox{theorem}{
enhanced,
boxrule=0mm,frame hidden,arc=0mm,
colback=white!0,
boxsep=0mm,left=3.5mm,
borderline west={2pt}{0pt}{black},
before skip=6pt,after skip=7pt,
halign=left,
parbox=false
}

% raises power of cosine functions
\protected\def\arccos{\qopname\relax o{\vphantom{i}arccos}}
\protected\def\cos{\qopname\relax o{\vphantom{i}cos}}


% Стили
\newcommand{\id}[1]{\text{id}_{#1}}
\newcommand{\norm}[1]{\lVert#1\rVert}
\newcommand{\sgnb}[1]{\text{sgn}\hspace*{-2pt}\left(#1\right)}
\newcommand{\sgnn}[1]{\text{sgn}\ #1}
\newcommand{\ren}[1]{\Re\text{e}\ #1}
\newcommand{\reb}[1]{\Re\text{e}\left(#1\right)}
\newcommand{\imn}[1]{\Im\text{m}\ #1}
\newcommand{\imb}[1]{\Im\text{m}\left(#1\right)}
\newcommand{\optm}[1]{\underset{\scriptscriptstyle #1}{\text{opt}}}
\newcommand{\optl}[1]{\text{opt}_{#1}}
\newcommand{\diff}{\text{d}}
\newcommand{\footnotes}[2]{$^{#1}$\hspace*{-1.1pt}\let\thefootnote\relax\footnote
{\hspace*{-18pt}#1 #2}}
\newcommand{\multiset}[2]{\left(\!\!\binom{#1}{#2}\!\!\right)}
\newcommand{\stirl}{\genfrac\{\}{0pt}{}}

% Смена шрифта для \mathcal
\DeclareMathAlphabet{\mathcal}{OMS}{cmsy}{m}{n}
\SetMathAlphabet{\mathcal}{bold}{OMS}{cmsy}{b}{n}

% Понижает высоту степени тригонометрических функций
\newcommand{\trsp}{\hspace*{-2.3pt}\phantom{}}

% Minipage environment
\newenvironment{column*}[2]{
\begin{minipage}[#1]{#2}
\setlength\baselineskip{13.4pt} % Строка
\setlength{\parskip}{6pt} % Абзац
\abovedisplayskip=6pt % Формула над
\belowdisplayskip=6pt % Формула под
\raggedright % Выравнивание
}{\end{minipage}}

% Tabular environment
\newenvironment{tabularc}[4]{
\def\center{c}
\if\center #4
\centering
\fi
\setlength\tabcolsep{#1}
\setlength\extrarowheight{#2}
\begin{tabular}{#3}
}{\end{tabular}\par}

% Tabularx environment
\newenvironment{tabularcx}[4]{
\setlength\tabcolsep{#1}
\setlength\extrarowheight{#2}
\tabularx{#4}{#3}
}{\endtabularx\par}

\newcolumntype{L}{>{\raggedright\arraybackslash}X}
\newcolumntype{R}{>{\raggedleft\arraybackslash}X}
\newcolumntype{C}{>{\centering\arraybackslash}X}

% framed codebox with language customization
\lstset{columns=fixed,basicstyle=\mono,basewidth=0.55em,keywordstyle=\monobold,commentstyle=\monoital\color[HTML]{888888}}

% emph={has_ichain,get_cuts},emphstyle=\monoital

\newtcblisting{code}[1]{
enhanced,
listing only,
breakable,
size=minimal,
boxrule=0mm,frame hidden,arc=0mm,
colback=white!0,
boxsep=0mm,left=3.5mm,
top=-1mm,bottom=-1mm,
borderline west={2pt}{0pt}{black},
before skip=6pt,after skip=7pt,
listing options={language=#1,showstringspaces=false}}

% sheet test environment
\newenvironmentx{sheet*}[2][1=,2=0]{
\setlength\cellspacetoplimit{3pt}
\setlength\cellspacebottomlimit{3pt}
\renewcommand\arraystretch{#2}
\tabularx{\textwidth}[m]{#1}
}{\endtabularx\par}
