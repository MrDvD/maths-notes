\RequirePackage{silence}
\WarningFilter{latexfont}{Font shape `TU/}
\WarningFilter{latexfont}{Size substitutions}
\RequirePackage[warnings-off={mathtools-colon,mathtools-overbracket}]{unicode-math}
% Ошибки ^
\documentclass[fleqn]{article}
\usepackage[table]{xcolor} % Цвет
\usepackage{amsmath,nccmath} % Математика
\usepackage[a5paper, margin=12mm]{geometry}

\usepackage{tikz}
\usepackage{cellspace}
\usepackage{adjustbox}
% Пакеты
\usepackage{polyglossia} % Язык
\setmainlanguage{russian}
\setotherlanguage{english}
\setkeys{russian}{babelshorthands=true}
\usepackage{fontspec} % Шрифты .otf
\usepackage{titlesec} % Заголовки
\usepackage{tabularx,multicol,multirow,array,makecell} % Таблицы
\usepackage[shortlabels]{enumitem} % Списки
\usepackage{xargs}
\usepackage{graphicx,float} % Графика
\usepackage{empheq}
\usepackage[makeroom]{cancel}
\usepackage{listings}
\usepackage[most]{tcolorbox} % Удобные рамки
\tcbuselibrary{listings}
\usepackage[open,openlevel=1]{bookmark} % Секции в pdf-файле
\graphicspath{{svg/}}
% Нумерация
\pagenumbering{gobble} % Страницы
\setcounter{secnumdepth}{0} % Заголовки
% Шрифты
\setmainfont[SizeFeatures={Size=12}]{Century Schoolbook}
\setmathfont[Scale=1.2]{TeX Gyre Schola Math}
\newfontfamily\bold[SizeFeatures={Size=12}]{Century Schoolbook Bold}
\newfontfamily\ital[SizeFeatures={Size=12}]{Century Schoolbook Italic}
\newfontfamily\boldital[SizeFeatures={Size=12}]{Century Schoolbook Bold Italic}
\newfontfamily\gilroysub[SizeFeatures={Size=16},UprightFont={*-Medium}]{Gilroy}
\newfontfamily\gilroy[SizeFeatures={Size=19},UprightFont={*-UltraLight}]{Gilroy}
\newfontfamily\tinyt[SizeFeatures={Size=10}]{Century Schoolbook}
\newfontfamily\mono[SizeFeatures={Size=12}]{Anonymous Pro Regular}
\newfontfamily\monobold[SizeFeatures={Size=12}]{Anonymous Pro Bold}
\newfontfamily\monoital[SizeFeatures={Size=12}]{Anonymous Pro Italic}
% Стили
\titleformat*{\subsection}{\raggedright\gilroysub}
\titleformat*{\section}{\gilroy\centering}
\titlespacing*{\subsection}{0pt}{6pt}{0pt}
\titlespacing*{\section}{0pt}{6pt}{1pt}
\definecolor{desc}{HTML}{888888}
\definecolor{table}{HTML}{D9D9D9}
\newcommand{\modb}[1]{\ \left(\mathrm{mod}\ #1\right)}
\newcommand{\modn}[1]{\ \mathrm{mod}\ #1}
\newcommand{\qedb}{\ \char"25A0}
\newcommand{\qedw}{\ \char"25A1}
\newcommand{\id}[1]{\text{id}_{#1}}
\newcommand{\norm}[1]{\lVert#1\rVert}
\newcommand{\abs}[1]{\left|#1\right|}
\newcommand{\sgnb}[1]{\text{sgn}\hspace*{-2pt}\left(#1\right)}
\newcommand{\sgnn}[1]{\text{sgn}\ #1}
\newcommand{\ren}[1]{\Re\text{e}\ #1}
\newcommand{\reb}[1]{\Re\text{e}\left(#1\right)}
\newcommand{\imn}[1]{\Im\text{m}\ #1}
\newcommand{\imb}[1]{\Im\text{m}\left(#1\right)}
\newcommand{\optm}[1]{\underset{\scriptscriptstyle #1}{\text{opt}}}
\newcommand{\optl}[1]{\text{opt}_{#1}}
\newcommand{\diff}{\text{d}}
\newcommand{\footnotes}[2]{$^{#1}$\hspace*{-1.1pt}\let\thefootnote\relax\footnote
{\hspace*{-18pt}#1 #2}}
\newcommand{\multiset}[2]{\left(\!\!\binom{#1}{#2}\!\!\right)}
\newcommand{\stirl}{\genfrac\{\}{0pt}{}}

% Смена шрифта для \mathcal
\DeclareMathAlphabet{\mathcal}{OMS}{cmsy}{m}{n}
\SetMathAlphabet{\mathcal}{bold}{OMS}{cmsy}{b}{n}

% Понижает высоту степени тригонометрических функций
\newcommand{\trsp}{\hspace*{-2.3pt}\phantom{}}

% Minipage environment
\newenvironment{column*}[2]{
\begin{minipage}[#1]{#2}
\setlength\baselineskip{13.4pt} % Строка
\setlength{\parskip}{6pt} % Абзац
\abovedisplayskip=6pt % Формула над
\belowdisplayskip=6pt % Формула под
\raggedright % Выравнивание
}{\end{minipage}}

% Tabular environment
\newenvironment{tabularc}[4]{
\def\center{c}
\if\center #4
\centering
\fi
\setlength\tabcolsep{#1}
\setlength\extrarowheight{#2}
\begin{tabular}{#3}
}{\end{tabular}\par}

% sheet test environment
\newenvironmentx{sheet*}[2][1=,2=0]{
\setlength\cellspacetoplimit{3pt}
\setlength\cellspacebottomlimit{3pt}
\renewcommand\arraystretch{#2}
\tabularx{\textwidth}[m]{#1}
}{\endtabularx\par}

% Tabularx environment
\newenvironment{tabularcx}[4]{
\setlength\tabcolsep{#1}
\setlength\extrarowheight{#2}
\tabularx{#4}{#3}
}{\endtabularx\par}

\newcolumntype{L}{>{\raggedright\arraybackslash}X}
\newcolumntype{R}{>{\raggedleft\arraybackslash}X}
\newcolumntype{C}{>{\centering\arraybackslash}X}

% Square cases environment
\makeatletter\newenvironment{sqcases}{%
  \matrix@check\sqcases\env@sqcases
}{%
  \endarray\right.%
}
\def\env@sqcases{%
  \let\@ifnextchar\new@ifnextchar
  \left\lbrack
  \def\arraystretch{1.2}%
  \array{@{}l@{\quad}l@{}}%
}\makeatother

% Theorem Tcolorbox
\newtcolorbox{theorem}{
enhanced,
boxrule=0mm,frame hidden,arc=0mm,
colback=white!0,
boxsep=0mm,left=3.5mm,
borderline west={2pt}{0pt}{black},
before skip=6pt,after skip=7pt,
halign=left,
parbox=false
}

% itemize X enumerate environment
\newenvironmentx{list*}[2][1=1,2=-]{
\if #2-
\def\listenv{itemize}
\ifnum #1=2
\begin{\listenv}[leftmargin=4.5mm,itemsep=1pt,topsep=-3pt,label=>]
\else
\begin{\listenv}[leftmargin=6.5mm,itemsep=1pt,label=---]
\fi
\fi
\if #2\#
\def\listenv{enumerate}
\begin{\listenv}[leftmargin=6.5mm,itemsep=-3pt,topsep=0pt,label=\theenumi)]
\fi
}{\end{\listenv}}

% framed codebox with language customization
\lstset{columns=fixed,basicstyle=\mono,basewidth=0.55em,keywordstyle=\monobold,commentstyle=\monoital\color[HTML]{888888}}

% emph={has_ichain,get_cuts},emphstyle=\monoital

\newtcblisting{code}[1]{
enhanced,
listing only,
breakable,
size=minimal,
boxrule=0mm,frame hidden,arc=0mm,
colback=white!0,
boxsep=0mm,left=3.5mm,
top=-1mm,bottom=-1mm,
borderline west={2pt}{0pt}{black},
before skip=6pt,after skip=7pt,
listing options={language=#1,showstringspaces=false}}
